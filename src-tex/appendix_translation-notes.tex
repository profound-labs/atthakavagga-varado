
\section*{A\~n\~na: auxiliary basis of attachment}\label{transl-auxiliary-basis}

\textit{Añña} means `other, different'. It occurs in various forms, the meaning of which is usually self-explanatory. But I have called \textit{aññena} (vv.789, 813, 908) and \textit{aññato} (v.790) `by means of an auxiliary basis of attachment'. 

The key for this choice is found in v.789, \textit{aññena so sujjhati sopadhīko}: `Then a person with one basis of attachment is intrinsically purified by means of another'. Norman translates this as `He who has acquisitions [which lead to rebirth] is purified by something else [than the noble path].

As for \textit{na hi aññamokkhā} (v.773), I take it to mean: there is no liberation except in relation to the bondage of desire.

\section*{Avijj\=a: Dhamma blindness}

\textit{Avijjā} is usually translated as `ignorance'. But ignorance means a lack of theoretical knowledge; for instance, consider Venerable Channa who, the sutta tells us, had theoretical knowledge of Dhamma but did not really see it. He had to ask Venerable \=Ananda to teach him how to see it (\textit{me tathā dhammaṃ deseyya yathāhaṃ dhammaṃ passeyyan'ti}) (S.3.132). This shows that attainment of insight is a kind of seeing, and not the attainment of theoretical knowledge. Thus, stream-entry is like being shown the way when one is lost, or having a lamp brought into a dark place, or attaining the pure and spotless Dhamma-eye (\textit{virajaṃ vītamalaṃ dhammacakkhuṃ udapādi}) (D.1.110). Enlightenment is like darkness being banished, and light arising (M.1.248); this occurs when one knows and sees (evaṃ jānato evaṃ passato) the four noble truths as clearly as if one were looking at fish in a crystal clear pond (D.1.84).

\section*{\=Asava: blinding tendencies}\label{transl-blinding-tendencies}

Regarding the word \textit{āsava}, the PED notes the `difficulty of translating the term'. Norman leaves it untranslated. Horner agrees that it has `always been a problem to translators', but says that the root \textit{sru} `suggests a flowing, discharge, leak, trickling, oozing, while the prefix ā-, especially with verbs of motion, means towards' (Middle Length Sayings, Vol.1, xxiii). The term occurs only once in the Octads (v.913). I have called it `blinding tendency' for the following reasons:

The first three fetters (\textit{saṃyojanāni}) are called āsavas to be forsaken by seeing (\textit{sakkāyadiṭṭhi vicikicchā sīlabbataparāmāso: ime vuccanti āsavā dassanā pahātabbā}, M.1.9). But as significant as the fetters themselves is the tendency to them, and one is not really free of the fetters until one is also free of the tendency to them (\textit{sakkāyadiṭṭhi sānusayā pahīyati \ldots{} vicikicchā sānusayā pahīyati \ldots{} sīlabbataparāmāso sānusayo pahīyati \ldots{} kāmarāgo sānusayo pahīyati \ldots{} byāpādo sānusayo pahīyati}) (M.1.434). So \textit{āsava} is synonymous with \textit{saṃyojana} together with \textit{anusayo}.

At A.4.127, `wearing away of the āsavas (\textit{āsavānaṃ khīṇaṃ}) is synonymous with `the fetters being weakened and easily rotting away' (\textit{appakasireneva saṃyojanāni paṭippassambhanti pūtikānī bhavantī'ti}). So again, \textit{āsavā} implies \textit{saṃyojanā}.

Āsavas are apparently the opposite of the inclination to Nibbāna. For while the āsavas might arise in one who does not develop the enlightenment factors (M.1.11), when a monk develops and cultivates the seven factors of enlightenment, he slants, slopes, and inclines towards Nibbāna (\textit{nibbānaninno nibbānapoṇo nibbānapabbhāro}) (S.5.75). So the āsavas are the inclination away from Nibbāna.

Āsavas are the inclination of one's mind (\textit{nati cetaso}) that results from wrong thoughts. For whatever a monk frequently thinks about, that will become the inclination of his mind (\textit{yaññadeva bhikkhave bhikkhu bahulamanuvitakketi anuvicāreti tathā tathā nati hoti cetaso}) (M.1.116). So a monk should not tolerate harmful and unskilful thoughts. For whereas āsavas might arise in one who does not remove harmful and unskilful thoughts, there are no āsavas in one who removes them (M.1.11). So āsavas are the harmful inclinations of the mind.

The three āsavas (\textit{kāmāsavo, bhavāsavo, avijjāsavo}: M.1.55) have their corresponding equivalents in the seven tendencies (\textit{kāmarāgānusayo, bhavarāgānusayo, avijjānusayo}: S.5.60). Thus each \textit{āsava} is equivalent to one of the tendencies:

\begin{itemize}
\item \textit{kāmāsavo} is equivalent to \textit{kāmarāgānusayo}: the tendency to infatuation with sensual pleasure
\item \textit{bhavāsavo} is equivalent to \textit{bhavarāgānusayo}: the tendency to infatuation with states of existence
\item \textit{avijjāsavo} is equivalent to \textit{avijjānusayo}: the tendency to Dhamma blindness
\end{itemize}

But the three anusayas that are āsavas are conditions for Dhamma blindness (\textit{āsavasamudayā avijjāsamudayo; āsavanirodhā avijjānirodho}: M.1.54). And these three anusayas are can therefore be called 'blinding tendencies'.

Āsavas, then, are the tendencies for the manifestation of three blinding mental objects:

\begin{itemize}
\item the tendency to infatuation with sensual pleasure (\textit{kāmāsavo});
\item the tendency to infatuation with states of existence (\textit{bhavāsavo});
\item the tendency to Dhamma blindness (\textit{avijjāsavo}) (S.5.56).
\end{itemize}

\section*{Khilo: hard heartedness}\label{transl-hard-heartedness}

The word \textit{khilo} occurs in verses 780 and 973. Both the PED and Norman call it `barrenness of mind'. I call it `hard heartedness'. This harmonises with \textit{khilo}'s other meaning: `waste or fallow land'. Verse 780 says that a monk who does not enter disputes does not incline to hard heartedness. Verse 973 says a monk should destroy the hard heartedness he might have for his fellows in the holy life -- (\textit{sabrahmacārīsu khilaṃ pabhinde}). Thus the turning point in Venerable Channa's practice was when he destroyed his hard heartedness and opened himself up (\textit{āvīakāsi khilaṃ pabhindi}). This led to him asking Venerable Ānanda for Dhamma instruction (S.3.134).

Whoever is angry with his fellows in the holy life, displeased with them, upset about them, become hard hearted (\textit{khilajāto}) does not incline to exertion (A.4.460). Whereas having a mind that is sympathetic for all living beings (\textit{akhilaṃ sabbabhutesu}) conduces to attaining rebirth in the Brahma realms (S.4.118).

\section*{Ta\d{n}h\=a: clinging}\label{transl-clinging}

\textit{Taṇha} is popularly called `craving' (i.e. strong desire) or thirst. But even Buddhas get thirsty and desire water (`Ānanda, bring me some water. I am thirsty (\textit{pipāsito'mhi}) and will [want to] drink (\textit{pivissāmī'ti})' - D.2.128). Therefore translating taṇhā as `strong desire' seems mistaken. In fact, the key to the meaning of \textit{taṇha} is not its strength, but its quality of ensnaring (\textit{jālinī}), attaching (\textit{visaṭā}) and clinging (\textit{visattikā}): ``I will teach you the clinging that ensnares, that flows, that attaches, that clings to one'' (A.2.212-3). The Buddha said of himself ``Within him, \textit{taṇhā} no longer lingers, entangling and binding, to lead him anywhere'' (\textit{yassa jālinī visattikā taṇhā natthi kuhiñci netave}) (S.1.107).

Clinging arises because of sensation (\textit{vedanāpaccayā taṇhā}). And because of clinging, there is ownership, which the suttas express in three ways: \textit{taṇhāpaccayā upādānaṃ} (clinging $\rightarrow$ ownership) (S.2.1); \textit{upadhi taṇhānidāno} (clinging $\rightarrow$ possession) (S.2.107-112); \textit{taṇhaṃ paṭicca pariyesanā pariyesanaṃ paṭicca lābho} (clinging $\rightarrow$ pursuit $\rightarrow$ acquisition) (A.4.400-1);

Clinging, the second Noble Truth, leads to renewed existence and is accompanied by enjoyment and love, seeking enjoyment here and there; that is, clinging to sensual pleasure, to existence and to becoming (\textit{yāyaṃ taṇhā ponobhavikā nandirāgasahagatā tatra tatrābhinandinī seyyathīdaṃ: kāmataṇhā bhavataṇhā vibhavataṇhā}). Clinging is what binds a man to \textit{saṃs\=ara} (\textit{taṇhādutiyo puriso dīghamaddhāna saṃsaraṃ}) (A.2.9). Clinging is called the seamstress; for it sews a man to this ever becoming birth (\textit{taṇhā hi naṃ sibbati tassa tasseva bhavassa abhinibbattiyā}) (A.3.399).

There is a sixfold body of clinging (\textit{chayime taṇhākāyā}):

\begin{itemize}
\item clinging to visible forms: \textit{rūpataṇhā}
\item clinging to sounds: \textit{saddataṇhā}
\item clinging to odours: \textit{gandhataṇhā}
\item clinging to tastes: \textit{rasataṇhā}
\item clinging to tactile objects: \textit{phoṭṭhabbataṇhā}
\item clinging to mental phenomena: \textit{dhammataṇhā}.
\end{itemize}

(S.2.3).

In the Octads \textit{taṇha} is either related to the clinging to existence (e.g. \textit{taṇhagataṃ bhavesu}, v.776 ), or it is used with no particular object. For instance, a peaceful person is `free of clinging ' (\textit{vītataṇho}: v.849 ).

\section*{Up\=ad\=ana: ownership or possession}

Possession or ownership arises from clinging (\textit{taṇhā}), and is fourfold:

\begin{itemize}
\item possession of sensual desire: \textit{kāmūpādānaṃ}
\item possession of views: \textit{diṭṭhūpādānaṃ}
\item possession of precepts and practices: \textit{sīlabbatūpādānaṃ}
\item possession of a theory of Self: \textit{attavādūpādānaṃ}
\end{itemize}

The suttas show that ownership is intrinsic to all types of \textit{upādāna}:

\begin{itemize}

\item Possession of sensual desire: sensual desire is gradually worn down, starting at stream-entry, because the stream-enterer is apparently ``not obsessed'' (\textit{pariyuṭṭhito}) by sensual love (\textit{kāmarāga}) or ill-will (\textit{byāpāda}) (M.1.321-5); the once-returner has ``attenuated'' lust, hatred and delusion (\textit{rāgadosamohānaṃ tanuttā}); the non-returner has ``destroyed'' the first five fetters (\textit{pañcannaṃ orambhāgiyānaṃ saṃyojanānaṃ parikkhayā}) (M.1.34). Although once-returners have attenuated lust, some lay people who attain this state maintain sexual relationships (A.5.137). Attachment to sexual pleasure means that states of greed, hatred and delusion (\textit{lobha, dosa, moha}) can still invade the mind and remain (\textit{cittaṃ pariyādāya tiṭṭhanti}) (S.5.369) even though one sees that sensuality (\textit{kāmā}) is of full of sorrow and danger (M.1.91; S.5.369). These unwholesome states are only overcome when one attains the rapture and bliss (\textit{pītisukhaṃ}) that is free of sensuality (\textit{aññatreva kāmehi}) and unskilful states (\textit{akusalehi dhammehi}) (M.1.91). At this stage, the once returner would presumably become a non-returner. This suggests that non-returners are free of the fourth and fifth fetters because they can enter \textit{jh\=ana}, and once-returners, at least some of them, cannot seem to do this.

\item Possession of views: If views are seen as not mine; not what I am; not my Self then they are abandoned (\textit{evametāsaṃ diṭṭhīnaṃ pahānaṃ hoti evametāsaṃ diṭṭhīnaṃ paṭinissaggo hoti}) (M.1.40).

\item Possession of precepts and practices: Religious practices that are not in accordance with the noble discipline (\textit{ariyassa vinaye}) are discarded by disciples when they first take refuge in the Buddha, and see that these practices are ineffective for spiritual progress. In their place, noble precepts are undertaken; thus the stream enterer is `possessed of the precepts dear to the noble ones' (\textit{ariyakantehi sīlehi samannāgato hoti}) which are perfectly fulfilled, but they are not grasped (\textit{aparāmaṭṭhehi}): a word that implies that one does not see things as `me' or `mine' or 'my Self' (S.2.94).

\item Possession of a theory of Self: The insight of stream-enterers and of arahants is identical: both have seen the five khandhas as they actually are with proper wisdom as being not personal (\textit{netaṃ mama, neso hamasmi, na meso attā'ti}) (M.1.234-5). In stream-enterers, however, the presumption of a `me' is still found (\textit{asmī'ti adhigataṃ}). But in relation to any particular one of the khandhas ``This is me'' is not found (\textit{ayamahamasmī'ti ca na samanupassāmī'ti}) (S.3.127-133).

Venerable Khemaka said the presumption of a Self was like the perfume of a lotus that could not be said to belong to any particular part of the flower, it belongs to the whole flower. In the same way, he said that though a noble disciple has abandoned the five lower fetters, still, in relation to the five aggregates subject to clinging, there lingers in him a residual tendency to think in terms of `me' that has not yet been uprooted (\textit{asmi'ti anusayo asamūhato}) (S.3.127-133). Venerable Khemaka explained that if the disciple dwells examining the rise and fall in the five aggregates (\textit{udayabbayānupassī viharati}), those residual tendencies are uprooted, just as the smell that remains in cloth that is cleaned with cowdung would eventually vanish if the cloth was left in a sweet-scented casket.

\end{itemize}

\section*{Di\d{t}\d{t}hi: fixed view}

In the Octads, \textit{diṭṭhi} usually implies attachment, and therefore I have often called it `fixed view' not simply `view'. A fixed view is synonymous with a conclusion (v.781); it is regarded as the highest Goal by the person grasping it (vv.796, 833). It is a source of confrontation (v.833) or offensive behaviour (v.847). It is regarded as belonging to oneself (v.846) and may lead grading others as equal, inferior or superior (v.799) or to think one is perfected (v.889).

It is not always necessary to call \textit{diṭṭhi} `fixed view'. For instance in verse 796, where it is already clear that attachment is involved: `If a person maintains that of views (`fixed views'), his is the highest Goal / Holding it as supreme in the world / And says that all other views (`fixed views') are contemptible / Then he has not gone beyond disputes'.

\section*{Dhammesu niccheyya samuggah\={\i}ta\d{m}: in regards to dogmatic religious teachings}\label{transl-dogmatic-religious-teachings}

This phrase occurs in four places, vv.785, 801, 837, 907. It can be analysed as follows:

\begin{itemize}
\item \textit{dhammesu}: locative plural of \textit{dhammā}, religious teachings: `in regards to religious teachings'.
\item \textit{niccheyya}: potential case of \textit{nicchināti}: fit to be (must be, ought to be, to be) discriminated, considered, investigated, ascertained (Duroiselle para 466).
\item \textit{samuggahītaṃ}: past participle of \textit{samuggaṇhāti}: seized, grasped, embraced.
\end{itemize}

Together, this means `in regards to religious teachings which must be ascertained [only] after having grasped them' which I have phrased as `in regards to dogmatic religious teachings'. Norman phrases it `grasped from among doctrines, after consideration'.

For example, Norman translates v.837: ``\thinspace`Māgandiya', said the Blessed One, `nothing has been grasped [by me] from among the doctrines, after consideration, [saying,] `I profess this'. Whereas I have said `In regards to dogmatic religious teachings, of none of them have I said `I proclaim this'\thinspace''.

\section*{Parama\d{m}: the highest Goal}\label{transl-highest-goal}

In the suttas, Nibbāna is called the highest (\textit{paramaṃ}) (e.g. Dh v.184); or the highest happiness (\textit{paramaṃ sukhaṃ}: Dh 203); or the highest purity (\textit{paramaṃ suddhiṃ} S 1.166); or the highest Nibbāna in this lifetime (\textit{paramadiṭṭhadhammaṃ nibbānaṃ} (A.5.64); or the highest Goal (\textit{paramattha}: Sn.v.68; v.219). In the Octads it is simply called \textit{paramaṃ}; but if one precisely translates that as `highest' it could lead to misunderstandings. For instance, v.796 would be translated `If a person maintains that of views, his view is the highest \ldots{} Then he has not gone beyond disputes'. Therefore I translate it 'If a person maintains that of views, his view is the highest Goal \ldots{} Then he has not gone beyond disputes'.

\section*{Yametamattha\d{m}: `which you mentioned'}\label{transl-which-you-mentioned}

\textit{Yametamatthaṃ} occurs in vv.838, 869, 870 -- where it introduces a question referring back to a word or phrase used by the Buddha. Norman calls it `that thing which is'. PED says the dependent and elliptic use of ya with a demonstrative pronoun represents a deictic or emphatic use, with reference to what is coming next or what forms the necessary compliment to what is just being said. Thus it introduces a general truth or definition, as we would say `just this', `namely', `that is'. PED says attha means `matter', `affair' or `thing'. So \textit{yametamatthaṃ} would mean `just this matter'. Fausbøll phrases it `which thou mentionest'.

\section*{Sa\~n\~na: fictitious perceptions \& Papa\~ncasa\.nkh\=a: the conception of `me'}\label{transl-fictitious-perceptions}\label{transl-conception-of-me}

One of the key themes of the Octads is \textit{sañña} -- popularly called `perception', though the PED says it can mean `conception, idea, notion'. The word occurs in seven verses. It has two varieties: good types and bad types. The single good-type occurrence is found in v.841, where Māgandiya was expected to have the simplest notion (\textit{aṇumpi saññaṃ}) of what the Buddha was talking about. The other occurrences are all of bad types.

Overcoming the bad types of \textit{saññaṃ} leads to dramatic results. The Octads says that the sage who understands \textit{saññaṃ} crosses the flood of sorrow (v.779), and says that if one is unattached to \textit{saññaṃ}, there are no bonds (\textit{ganthā}) (v.847). Verse 802 says the arahant does not concoct the slightest \textit{saññā} regarding what is seen, heard or cognised (\textit{diṭṭhe va sute mute vā pakappitā natthi aṇūpi saññā}). This shows that \textit{saññā} can hardly mean simply perception; after all, the Buddha was able to perceive the people he was talking to. Alternatively in the Octads, \textit{saññaṃ} has the same meaning as it does in the \textit{Madhupiṇḍika Sutta} (MN 18), where the Buddha said that \textit{saññā} do not linger in the arahant (\textit{saññā nānusenti}). But what does \textit{saññā} mean, then?

In the \textit{Madhupiṇḍika Sutta}, the Buddha said that the phrase `\textit{saññā} do not linger' means that if one does not enjoy, welcome or take hold of \textit{papañcasaññāsaṅkhā}, this puts an end to the seven latent tendencies. This statement also seems to say that if one does not take hold of \textit{papañcasaññāsaṅkhā}, then these \textit{papañcasaññāsaṅkhā} do not linger. Thus \textit{saññaṃ} seems equivalent to \textit{papañcasaññāsaṅkhā}. And \textit{papañcasaññāsaṅkhā} is likely to be the meaning of \textit{sañña} in the Octads, too. But if, in the Octads, \textit{saññaṃ} is a synonym of \textit{papañcasaññāsaṅkhā}, in what way does it differ from \textit{papañcasaṅkhā}, of which it is said to be the source (v.874)?

Verse 916 says the basis of \textit{papañcasaṅkhā} is the thought `I am' (\textit{mūlaṃ papañcasaṅkhāya mantā asmīti sabbamuparundhe}), and the Buddha said when there is the thought `I am' (\textit{asmīti sati}), there comes another seventeen thoughts : I am in this world; I am thus; I am otherwise; I am bad; I am good etc. (A.2.212-3). From this, it would seem that `I am' is the root of seventeen other thoughts, and that all these thoughts comprise \textit{papañcasaṅkhāya}. Therefore \textit{papañcasaṅkhā} could be called `the conception of me'. My translation leads to the following results in the two mentions of the word in the Octads:

\begin{itemize}
\item v.874: Fictitious perceptions are indeed the source of the conception of `me': \textit{saññānidānā hi papañcasaṅkhā}
\item v.916: A sage should completely restrain the basis of the conception of `me': the thought `I am': \textit{mūlaṃ papañcasaṅkhāya mantā asmīti sabbamuparundhe}
\end{itemize}

In the Octads, \textit{papañcasaṅkhā} is a further development of \textit{saññā}, because, says v.874, the source of \textit{papañcasaṅkhā} is \textit{saññā} (\textit{saññānidānā hi papañcasaṅkhā}). And if \textit{papañcasaṅkhā} is `the conception of me', the meaning of \textit{saññā} can be found in the \textit{Mūlapariyāya Sutta} where it says that ordinary person (\textit{assutavā puthujjano}) perceives earth as `earth' (\textit{pathaviṃ pathavito sañjānāti}). Having done so, he then conceives `earth is mine' (\textit{pathaviṃ meti maññati}). This is because he has not comprehended earth (\textit{apariññātaṃ tassā'ti}). If the expression `earth is mine' can be taken as equivalent to `the conception of me' (\textit{papañcasaṅkhā}), then it becomes clear that it is the perception of an ordinary person that leads to the conception of me. And this act of perception, called \textit{sañjānāti}, arises from non-comprehension (\textit{apariññātaṃ}). And such perception is not found in arahants. They do not perceive earth as earth. Rather, arahants insightfully know earth as earth (\textit{pathaviṃ pathavito abhijānāti}); and having done so, they do not conceive `earth is mine' (\textit{pathaviṃ meti na maññati}). Thus, in the Octads, \textit{saññā} is the perception of non-comprehension that gives rise to the conception of me. It is not perception with insight. In the Octads I call it `fictitious perception', because it involves perceiving what is not factual. This fictitious perception is therefore the source of the eighteen thoughts of `me'; in other words, the source of the conception of `me' is the fictitious perceptions.

This gives the following results:

\begin{itemize}
\item v.779) Having understood fictitious perceptions (\textit{saññaṃ pariññā}) \ldots{} The sage crosses the flood of sorrow.
\item v.792) A person bound to fictitious perceptions (\textit{saññasatto}) \ldots{} goes high and low.
\item v.802) Whoever does not concoct the slightest fictitious perception (\textit{aṇūpi saññā}) regarding what is seen, heard or cognised \ldots{} how could anyone have any doubts about him?
\item v.841) Asking questions that are based on a fixed view, you cannot apprehend the simplest notion (\textit{nāddakkhi aṇumpi saññaṃ}).
\item v.847) For one unattached to fictitious perceptions, there are no bonds (\textit{saññāvirattassa na santi ganthā}) \ldots{} Those attached to fictitious perceptions and to views / Roam the world offending people.
\item v.874) Fictitious perceptions are indeed the source of the conception of `me' (\textit{saññānidānā hi papañcasaṅkhā}).
\item v.886) Apart from the mere notion of it (\textit{aññatra saññāya}), there are not many and various eternal Truths in the world.
\item v.916) A sage should completely restrain the basis of the conception of `me': the thought `I am' (\textit{mūlaṃ papañcasaṅkhāya mantā asmīti sabbamuparundhe})
\end{itemize}

\section*{Sati: `attentiveness'}\label{transl-attentiveness}

\subsection*{Summary}

\textit{Sati} is popularly called mindfulness. But `mindful' means `bearing in mind', which is not the meaning of \textit{sati}. I have called it `attentiveness'. Various quotes in the \textit{Ānāpānasati Sutta} support this.

\subsection*{Outline}

The \textit{Satipaṭṭhānā Sutta} indicates that sati has five qualities.

\begin{enumerate}
\item it has four fields of activity: (1) body (\textit{kāye kāyānupassī}) (2) sensation (\textit{vedanāsu vedanā}) (3) mind (\textit{citte cittā}) and (4) the Buddha's teaching models (\textit{dhammesu dhammā}). \textit{Sati} involves observing various aspects within these four fields (\textit{anupassī viharati}).
\item one should observe enthusiastically (\textit{ātāpī})
\item one should observe fully conscious (\textit{sampajāno})
\item one should observe attentively (\textit{satimā})
\item one should observe having removed covetousness and distress for the world (\textit{vineyya loke abhijjhādomanassaṃ}) (S.5.141).
\end{enumerate}

Various other aspects of sati are further explained in the \textit{Ānāpānasati Sutta} (MN 118).

\subsection*{Aspects of sati in the \=An\=ap\=anasati Sutta}

\begin{itemize}

\item When a monk is practising attentiveness with breathing (\textit{ānāpānasati}) the Buddha called it \textit{kāye kāyānupassī} because `Whenever, Ānanda, a monk knows: `I breathe in/out long/short'; or trains himself `Experiencing/tranquillising the whole body, I will breathe in/out'; on that occasion the monk dwells observing \textit{kāye kāya}. For what reason? (\textit{taṃ kissa hetu}). I call this a certain aspect of body (\textit{kāyaññatara}), Ānanda, that is, breathing in and breathing out' (\textit{yadidaṃ assāsapassāsaṃ}). This shows that \textit{kāye kāyānupassī} means observing certain aspects of the body. This is usually -- unhelpfully -- called `seeing the body in the body'.

\item The \textit{Ānāpānasati Sutta} says that the second field of activity of \textit{sati} is called \textit{vedanāsu vedanānupassī} because `I call this a certain aspect of sensation (\textit{vedanaññatarāhaṃ}) Ānanda, that is, close attention (\textit{sādhukaṃ manasikāraṃ}) to breathing in and breathing out'.

\item The \textit{Ānāpānasati Sutta} says that the third field of activity of \textit{sati} is called \textit{citte cittānupassī} because `there is no development of \textit{samādhi} with \textit{ānāpānasati} if one's attentiveness is muddled (\textit{muṭṭhassatissa}), if one is not fully conscious' (\textit{asampajānassa}).

\item The \textit{Ānāpānasati Sutta} says that the fourth field of activity of \textit{sati} is called \textit{dhammesu dhammānupassī} because `having, with wisdom, seen the abandoning of covetousness and distress, a monk is one who looks on closely with equanimity' (\textit{so yaṃ taṃ hoti abhijjhādomanassānaṃ pahānaṃ taṃ paññāya disvā disvā sādhukaṃ ajjhupekkhitā hoti}).

\end{itemize}

In conclusion, \textit{sati} means the enthusiastic observation of various aspects of the body, sensation, the mind, or of various Dhammas, and involves close attention, a mind that is unmuddled, fully conscious, equanimous, and free of covetousness and distress. I call this state `attentiveness'.

\section*{Pajahati: `detach from'}

One of the most interesting words of the Octads is \textit{pajahati} (as well as three similar verbs: \textit{jahati, nissajjati}, and \textit{paṭinissajjati}). The PED calls them: give up, renounce, forsake, abandon, eliminate, let go, get rid of. In most verses, any of these would seem satisfactory. However, verse 900 would be problematic because it would imply that an arahant is ``one who has given up precepts'' (\textit{sīlabbataṃ vāpi pahāya}). This seems unlikely. 

What the Octads emphasises is good behaviour, as well as detachment. Venerable So\.na said the moral behaviour of an arahant is a natural expression of an enlightened mind. When he was accused of practising harmlessness due to blind attachment to rituals and asceticism (\textit{silabbataparāmāsaṃ}), he said no, the arahant is intent on non-harming because of the destruction of greed, hatred and delusion (\textit{khayā rāgassa vītarāgattā \ldots{} dosassa vītadosattā \ldots{} mohassa vītamohattā abyāpajjādhimutto hoti}) (Vin.1.183-5).

These four P\=a\d{l}i words therefore seem in fact to mean the opposite of grasping. `Let go' would have been the easy choice, but it has a range of dubious meanings, including that of not keeping to a moral standard. So I translate the four words as: `detach'. The arahant has therefore `detached' from precepts, rather than `let go' of them. This translation harmonises with verse 798 which says a monk should `not be tethered' (\textit{na nissayeyya}) to precepts. It also harmonises with verse 791 which describes a monkey releasing one branch in order to seize another (\textit{purimaṃ pahāya aparaṃ sitāse}). A slight exception occurs with the word \textit{vippajahe} at v.926, which I call ``he should abandon'' (laziness, deception, merriment etc), rather than ``he should detach'' from these things. 

\section*{Instrumental and ablative cases as `intrinsic'}\label{transl-intrinsic}

One of the challenges of the Octads is not just in discriminating grammatical case endings, but in discriminating sense vs. nonsense for any particular case. For example, if someone claims that purity is on account of one's views, does it mean that

\begin{itemize}
\item purity is \emph{spoken} of on account of one's views?
\item Or, purity is \emph{achieved} on account of one's views?
\end{itemize}

The Buddha's conversation with M\=agandiya at verses 835-841 revolves around this dilemma. M\=agandiya asks the Buddha: 

\begin{verse}
This inner peace, whatever it is,\\
How is it explained by the wise?
\end{verse}

The Buddha replies:

\begin{verse}
They do not say that purity is on account of one's views\\
Learning, knowledge, or precepts and practices;\\
Nor on account of one's lack of views,\\
Learning, knowledge, precepts and practices.\\
But by detaching from these,\\
Not grasping them,\\
At peace, untethered,\\
One no longer hungers for existence.

\textit{Na diṭṭhiyā na sutiyā na ñāṇena sīlabbatenāpi na suddhimāha\\
Adiṭṭhiyā assutiyā añāṇā asīlatā abbatā nopi tena\\
Ete ca nissajja anuggahāya santo anissāya bhavaṃ na jape.}
\end{verse}

The first phrase of the Buddha's reply to Māgandiya in the instrumental case, the second, the ablative case. But the translation of both is probably identical -- and Norman agrees. He translates it `not by view \ldots{} not by absence of view'.

Duroiselle confirms this. He says:

\begin{enumerate}

\item the ablative case can be translated as `on account of' or `by reason of' and he says the same for the instrumental case -- see next. Thus he says \textit{sīlato naṃ pasaṃsanti} means `they praise him for [i.e. on account of] his virtue' (Practical Grammar of the Pali Language: paragraph 600, xi).

\item the instrumental case `shows cause or reason' and can `therefore be translated by such expressions as: by means of; on account of; through; by reason of; owing to'. Duroiselle gives the example: \textit{kammuna vasalo hoti}, he is pariah by reason of [i.e. on account of] his work (paragraph 599, ii).

\item The instrumental and ablative cases are interchangeable. Duroiselle says that for an ablative meaning, the instrumental case `may be used as well' (599, xi); and that `the ablative is very frequently used instead of the instrumental' (599, xv).

\end{enumerate}

Norman has translated verse 839 as `purity is not by view, by learning, by knowledge, or even by virtuous conduct and vows'. This leaves the dilemma unresolved; but if it is taken to mean `purity is not achieved by means of view \ldots{} etc.' it would contradict much of Buddhist teaching; for instance, that right view (\textit{sammādiṭṭhi}) assisted by virtue/precepts (\textit{sīlānuggahitā}) and wide learning (\textit{sutānuggahitā}) has enlightenment as its fruit (A.3.20); that good conduct leads step by step to the summit (\textit{kusalāni sīlāni anupubbena aggāya parentīti}). And there would be no point in the Buddha answering Venerable Sāriputta's question in v.961, about what a monk's precepts and practices should be.

What purpose is served by having the Buddha contradict himself in the Discourse with Māgandiya, apart from undermining the rest of his Dhamma teachings? We must dare to admit that although final liberation means liberation even from the path, and that nonetheless the path is the means to liberation. I describe this further in the section `Goal and path: what is the relationship?' on page \pageref{goal-and-the-path}.

One can translate the instrumental case in several ways. For example `by means of', `by reason of', `on account of'.

However, none of these phrases really settles the confusion between

\begin{itemize}
\item the means by which purity can be spoken of, and
\item the means by which purity can be achieved.
\end{itemize}

In this present translation, for clarity, for both the instrumental and ablative cases I use the word `intrinsic'. Therefore I translate the Buddha's reply to Māgandiya as follows (v.839):

\begin{verse}
They do not say that purification is intrinsic to views\\
Learning, knowledge, or precepts and practices;\\
Nor intrinsic to a lack of views,\\
Learning, knowledge, precepts and practices.
\end{verse}

\section*{Miscellaneous translations}

\begin{description}
\item[chanda:] popularly called `desire'. But in the Octads, as with \textit{taṇha}, its basis is ignorance, so I call it `longing'. 

\item[dhamma/ā:] I have called it `religious teaching/s' or `teachings' or the `Buddha's teaching' or `Truth'.

\item[pakappitā:] I have called `concocted' (v.784) (PED says: arranged, planned, attended to designed, made). Norman has `formed'. The word \textit{kappayanti} is a synonym (vv.794; 803). At v.784 \textit{saṅkhatā} is a near synonym: `conjured up'. Norman calls it `constructed'.

\label{transl-blindly-follow}
\item[purakkharoti:] PED calls it `follow'. But in the Octads it seems to mean `blindly follow'. Norman calls it `prefer'.

\item[brāhmaṇo:] means Brahman i.e. arahant.

\item[sacca/saccaṃ:] I follow the PED in calling \textit{sacca} `true' (i.e. adjective); \textit{saccaṃ} `Truth' (i.e. noun).

\end{description}
