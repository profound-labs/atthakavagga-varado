
\section*{Summary}

The Octads presents different statements on Truth and Truth-finding in which the words of arahants are contrasted with those of `so-called experts'. In summary the advice of the Octads is this:

\begin{itemize}
\item detach from everything
\item realise Truth
\item purify yourself
\item find peace
\item abide in solitude
\item attain the fruits of Truth-realisation
\item dispel Darkness
\item desist from arguments
\item develop good qualities -- the `shoulds'
\end{itemize}

\section*{Detaching from everything}

One of the prominent themes of the Group of Octads is attachment -- attachment to pleasure and to views. Attachment to pleasure means that `if his pleasures diminish, he suffers as if pierced with an arrow' (v.767) and, for those so attached `when drawn into difficulty, they lament: ``What will become of us in the hereafter?''\thinspace' (v.773).

Attachment to views means thinking one's own view is supreme in the world (v.796), and asserting that it alone is Truth (v.832), the highest Goal (vv.788; 796). This leads to disputes (v.796); to hard heartedness (v.780); to seeing other people as inferior (vv.798); and calling them fools (v.887).

As for detachment, in the Octads, it seems that one can only detach from a path of virtue; one cannot detach from a path of non-virtue. One cannot cross over the flood of sorrow without having bailed out one's boat (v.771). This is difficult to do (v.772). It involves freeing oneself from defilement, not just living physically detached in a cave (v.772). If one does not detach, one is not easily liberated (v.773).

Detachment means abandoning merit and evil (v.790) and detaching from fictitious perceptions (v.792), from knowledge (v.800), from aspiration for every form of existence (v.801); from precepts and practices, and all conduct, whether flawed or not (v.900). In the Octads, these are called auxiliary bases of attachment (vv.789, 813, 908, 790).

Detaching from precepts does not mean immoral behaviour, because the Buddha describes the supreme person as one who is restrained in speech (v.850); not arousing contempt (v.852); who abstains from initiating new kamma (v.900). This aspect of practice is discussed in: Should monks forsake precepts and practices? on page \pageref{should-monks-forsake}.

\section*{Realising Truth}

Truth is single (v.884) and eternal (v.884); there is not another Truth about which mankind should contend (v.884). Truth is realised through one's own insight (\pali{sakkhidhammamanītihamadassī}) (v.934). One who has realised Truth has done so by detaching from everything (v.946), both merit and evil, (v.790) and precepts and practices (v.900). He hopes for nothing in the world (v.794). Therefore he is peaceful (v.946).

But so-called experts think their own religious teachings, views and opinions are Truth, and call them `sanctity' (\pali{subhaṃ}) (vv.824; 832; 904; 910) or the highest Goal (vv.833; 904). Other people's views they call `contemptible'; the sophists call them `Falsehood' (\pali{musāti}) (v.886).

\section*{Purifying oneself}

The wise say that purification of heart is the summit (\pali{agga}) of practice. Purification means freedom from fictitious perceptions (v.874-6). It is to be found here in this world (v.876) - though some so-called experts think it is found only at the final passing away of the khandhas (v.876).

Some so-called experts think that purification is intrinsic to self-restraint (v.898), or to ascetic practices (v.901). They think that only in their own dogmatic teachings is there purification (v.892) and accuse people with other religious views of straying from purification (v.891). But the good say that these things proclaimed by so-called experts are merely auxiliary bases of attachment (v.908). As such, they cannot purify other bases of attachment (v.790). If one dedicates oneself to a basis of attachment, one is led on to further existence (v.898).

Various ascetic practices are described in: Should monks forsake precepts and practices? on page \pageref{should-monks-forsake}.

\section*{Finding peace}

Sometimes the goal is described as peace (\pali{santo}). Peace is found by scrutinising religious views without grasping them (v.837). This peace comes from within, not from some auxiliary basis of attachment (v.919). It comes from not clinging (v.912), from forsaking everything (vv.946; 949), from having nothing further to relinquish (v.919), from extinguishing the illusion of Self (v.933).

\section*{Abiding in solitude}\label{abiding-in-solitude}

Sometimes the goal is said to be solitude (\pali{vivekā}). Solitude does not simply mean physical solitude (v.772) or faring alone (\pali{ekacariyaṃ}. It means freedom from attachment, defilement and delusion (v.772) even in the midst of sense contact (v.851). It means freedom from passion, clinging to nothing in the world (v.915), and seeing nothing in the world as one's own (v.861). But solitude also means physical solitude; for instance, not pursuing sexual intercourse (vv.814, 820, 821), and having no children, cattle fields or property (v.858).

At S.4.37 a `solitary person' (\pali{ekavihārī'ti}) is one who, even when crowded round, dwells without clinging (\pali{taṇhā}). At S.2.283, the Buddha said that spending the whole day alone is only a partial fulfillment of solitude (\pali{ekavihārī}). For complete fulfilment (\pali{vitthāreṇa paripuṇṇo hoti}), one must abandon the past and the future and thoroughly remove longing and greed for the present forms of individual existence (\pali{yaṃ atītaṃ taṃ pahīnaṃ yaṃ anāgataṃ taṃ paṭinissaṭṭhaṃ; paccuppannesu ca attabhāvapaṭilābhesu chandarāgo suppaṭivinīto}).

\section*{Attaining the fruits of Truth-realisation}

The good qualities that Truth-realisation brings are these: no illusion of Self (v.783); no boasting of one's virtue (v.783); no conceit (v.783); seeing things as they are (v.793); conducting oneself openly (v.793); not hungering for existence (v.839); being untethered, unattached, not possessive (vv.839; 849; 851); being free of false desire and yearning (vv.849; 856); being free of strong emotions (vv.850; 852); being well behaved in body, speech and mind (vv.850; 852; 853); being free of sorrow (v.851); being not guided by fixed views (v.851); not getting involved in arguments (vv.859; 912); not comparing oneself with others (vv.855; 860); being indifferent to pleasure (v.857); being free of time (v.860); not concocting religious teachings nor blindly following them (v.861).

\section*{Dispelling Darkness}

Darkness is that which should be dispelled (\pali{vinodayeyya}), or put an end to (\pali{vihane}). Having done this, one attains delight (\pali{ratimajjhagā}). Darkness has many aspects, one of which is a disturbed mind (\pali{āvilattaṃ manaso}). A disturbed mind should be dispelled, by recognising that it is part of Darkness. But some aspects of Darkness must be dispelled with a composed mind (\pali{ekodibhūto}), by examining the Buddha's teachings at suitable times, in suitable ways (vv.956, 967, 975).

\section*{Desisting from arguments}

Sages do not enter arguments or speak to people contentiously (vv.780; 844) because they do not cling to any view (v.787). They do not take sides in a dispute (v.800). They do not pit one view against another, or grasp any view as the highest Goal (\pali{paramuggahītaṃ}). Having abandoned fixed opinions, they create no more trouble in the world (v.894). They regard non-dispute as the grounds for peace (v.896).

But people who say that their own view is the highest Goal, and call other views are contemptible, have not gone beyond disputes (v.796). Such people proclaim that purity is intrinsic to their religious teachings alone (v.824). They go looking for arguments, seeking praise, considering other people fools (v.825). If they lose an argument, they are shaken by the criticism (v.826) and wail about their defeat (v.827). The victor, however, gets puffed up with pride. This is the basis of later distress (v.830). Seeing this, one should desist from arguments because it does not lead to purity (v.830).

\section*{Developing good qualities -- the `shoulds'}

Of the Octads' one hundred and twenty-two `shoulds', eighty-two are packed into the last three discourses, giving these discourses a distinctive tone. These `shoulds' are the views that v.837 says should be scrutinised without grasping. They occur as:

\begin{itemize}
\item an answer to a request to `speak about the path of practice, about monastic discipline, and also about samadhi' (v.921);
\item a recitation of the training rules (v.940);
\item as an answer to Venerable Sāriputta's question: `For a monk going where he never before has gone \ldots{} What should be the manner of his speech? What should be his field of conduct? What should be that energetic monk's precepts and practices?'
\end{itemize}

Although these `shoulds' are to be pursued, they are not to be made objects of pride (v.822, v.846). If they are grasped, one will be simply led onto further existence (v.898).
