
The Group of Octads (the 'Octads') means ‘the group of discourses with eight verses’. It is the fourth chapter of the Sutta Nipāta and consists of sixteen discourses in metrical form. In fact only four of the discourses are octads; a few of them have even twenty verses. Some of the discourses, where there are breaks in logical flow, appear to be composites. For example, the conversation with Magandiya seems to suddenly finish at verse 841. 

The Buddha must have composed the Octads relatively early his teaching career because discussions concerning it are mentioned in other parts of the Pali canon. How important the Octads must have been to the early monks is illustrated in a story concerning Venerable Soṇa who recited all sixteen discourses to the Buddha one early morning. This incident is described in Appendix 8.

My main references for this translation have been the Pali from the Vipassana Research Institute CD-ROM; K.R.Norman’s Group of Discourses (published by the Pali Text Society, 2006); and the Pali-English Dictionary (the ‘PED’) edited by T.W.Rhys Davids and William Stede. I consulted other works too – translations by Ven. H. Saddhātissa, Bhikkhu Ṭhānissaro, Paññobhāsa Bhikkhu, Max Muller, E.M. Hare, and John Ireland. For Pali grammar I used C.Duroiselle’s A Practical Grammar of the Pali Language (1997). To all these authors and editors, and to everyone who helped make these works available, I express my profound appreciation.

This present work seemed necessary because although Norman’s scholarly approach is impressive, his translation is sometimes unclear. One of the goals of this work was to solve the puzzles of his text. I have made copious explanatory footnotes - divided into Notes for Readers and Notes on Translation as well as extensive notes in the Appendices.
