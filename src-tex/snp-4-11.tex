
\begin{verse}

(Questioner)
\vspace*{-0.5\baselineskip}

\verseref{862} Where do quarrels,\\
Disputes, lamentation and sorrow come from,\\
Together with selfishness,\\
Pride, arrogance and malicious speech?\\
From where do they come?\\
Please tell me.

\pali{kutopahūtā kalahā vivādā,\\
paridevasokā sahamaccharā ca.\\
mānātimānā sahapesuṇā ca,\\
kutopahūtā te tadiṅgha brūhi.}

(The Buddha)
\vspace*{-0.5\baselineskip}

\verseref{863} From what is loved\\
Come quarrels, disputes, lamentation and sorrow,\\
Together with selfishness, pride,\newline arrogance and malicious speech.\\
Quarrels and disputes are linked to selfishness;\\
From disputes comes malicious speech.

\pali{piyappahūtā kalahā vivādā,\\
paridevasokā sahamaccharā ca.\\
mānātimānā sahapesuṇā ca,\\
maccherayuttā kalahā vivādā.\\
vivādajātesu ca pesuṇāni.}

(Questioner)

\verseref{864} What is the source of love and greed in the world?\\
And what is the source of expectation and hope\newline that a man has for the next life?\pagenote{`hope': PED calls niṭṭhā `aim'. This would give `aim that a man has for the next life'. I call this `hope'. Norman calls it `fulfillment [of hope]' which does not fit the context.}

\pali{piyā su lokasmiṃ kutonidānā,\\
ye cāpi lobhā vicaranti loke.\\
āsā ca niṭṭhā ca kutonidānā,\\
ye samparāyāya narassa honti.}

(The Buddha)

\verseref{865} Longing\\
Is the source of love and greed in the world,\\
And also the source of expectation and hope\newline that a man has for the next life.

\pali{chandānidānāni piyāni loke,\\
ye cāpi lobhā vicaranti loke.\\
āsā ca niṭṭhā ca itonidānā,\\
ye samparāyāya narassa honti.}

(Questioner)

\verseref{866} What is the source of longing?\\
And from where do fixed opinions come from,\newline anger, lies, perplexity,\\
And other such things spoken of by the Ascetic?

\pali{chando nu lokasmiṃ kutonidāno,\\
vinicchayā cāpi kutopahūtā.\\
kodho mosavajjañca kathaṃkathā ca,\\
ye vāpi dhammā samaṇena vuttā.}

(The Buddha)

\verseref{867}\verseref{868} Longing arises in the world\\
Dependent on what is called\newline the `pleasing-displeasing duality'.\pagenote{`duality’: I add the word `duality' in v.867 because in v.868 pleasing-displeasing are called \pali{dvayameva}.}\\
Anger, lies, perplexity and other such things\\
Also arise when this duality exists.\\
A person develops fixed opinions\\
From watching the apparent annihilation\newline and existence of material phenomena.\pagenote{`apparent': v.870 says that sense contact is the source of the annihilation and existence of material phenomena. Because this annihilation and existence depends on sense contact, it seems better to call it `apparent' annihilation and existence, rather than simply `annihilation and existence'. Another reason for calling them `apparent annihilation' and `apparent existence' is that the Buddha veered away from two extreme views: the view `All exists' (\pali{sabbamatthī'ti}) and the view `All does not exist' (\pali{sabbaṃ natthī'ti}) (S.2.17). Therefore it seems more appropriate to talk in terms of apparent existence and non-existence, rather than wrongly speak of actual existence and non-existence.}

One who is perplexed\\
Should train in the path of knowledge,\\
For it is in having \emph{known}\\
That the Ascetic has spoken of all these things.

\pali{sātaṃ asātanti yamāhu loke,\\
tamūpanissāya pahoti chando.\\
rūpesu disvā vibhavaṃ bhavañca,\\
vinicchayaṃ kubbati jantu loke.}

\pali{kodho mosavajjañca kathaṃkathā ca,\\
etepi dhammā dvayameva sante.\\
kathaṃkathī ñāṇapathāya sikkhe,\\
ñatvā pavuttā samaṇena dhammā.}

(Questioner)

\verseref{869} What is the source of pleasure and pain?\\
When what is not do they not exist?\\
And apparent annihilation and existence\newline -- which you mentioned --\pagenote{\pali{yametamatthaṃ}: `which you mentioned', see Translation Notes, page \pageref{transl-which-you-mentioned}.}\\
Tell me, too, what is their source?

\pali{sātaṃ asātañca kutonidānā,\\
kismiṃ asante na bhavanti hete.\\
vibhavaṃ bhavañcāpi yametamatthaṃ,\\
etaṃ me pabrūhi yatonidānaṃ.}

(The Buddha)

\verseref{870} Sense contact is the source of pleasure and pain.\\
When there is no sense contact\newline pleasure and pain do not exist.\\
And apparent annihilation and existence\newline -- which I mentioned --\\
Sense contact too is their source.

\pali{phassanidānaṃ sātaṃ asātaṃ,\\
phasse asante na bhavanti hete.\\
vibhavaṃ bhavañcāpi yametamatthaṃ,\\
etaṃ te pabrūmi itonidānaṃ.}

(Questioner)

\verseref{871} What is the source of sense contact?\\
And where does grasping arise from?\\
When what is not,\newline is there then no possessiveness?\\
When what is annihilated,\newline do sense contacts stop contacting?

\pali{phasso nu lokasmi kutonidāno,\\
pariggahā cāpi kutopahūtā.\\
kismiṃ asante na mamattamatthi,\\
kismiṃ vibhūte na phusanti phassā.}

\clearpage

(The Buddha)

\verseref{872} Sense contact is dependent\newline on the body-mind complex.\\
Desire is the source of grasping.\\
When desire is not,\newline there is no possessiveness.\\
When material form is annihilated,\newline sense contacts stop contacting.\pagenote{When material form is annihilated: this is arahantship, because the arahant lays down one body without taking up another (S.4.55). I call this the annihilation of material form, not the annihilation of the body, because this annihilation involves more than just the present body, but the potential for future bodies too. Of his own body, the Buddha said that after his death, devas and humans would see him no more (D.1.46).}

\pali{nāmañca rūpañca paṭicca phasso,\\
icchānidānāni pariggahāni.\\
icchāyasantyā na mamattamatthi,\\
rūpe vibhūte na phusanti phassā.}

(Questioner)

\verseref{873} For one arriving at what,\newline is material form annihilated\\
Together with its pleasure and pain.\pagenote{`material form together with its pleasure and pain': Norman phrases this `How does happiness or misery disappear also?' But the Buddha, in his answer to this question, makes no reference to \pali{sukhaṃ dukhañcāpi}. Therefore the two words are obviously adjuncts to the question on material form; they are not meant as a separate question. Because at v.875 the questioner says `You have explained what we asked'.}\\
How is it annihilated?\\
Tell me this.\\
My heart is set on knowing how it is annihilated.

\pali{kathaṃsametassa vibhoti rūpaṃ,\\
sukhaṃ dukhañcāpi kathaṃ vibhoti.\\
etaṃ me pabrūhi yathā vibhoti,\\
taṃ jāniyāmāti me mano ahu.}

(The Buddha)

\verseref{874} For one who is not aware\newline of fictitious perceptions,\pagenote{`fictitious perceptions': see Translation Notes, page \pageref{transl-fictitious-perceptions}.}\pagenote{The Buddha said he taught a doctrine such that, for the arahant, fictitious perceptions do not linger in him (\pali{saññā nānusenti}) (M.1.108). Freedom from fictitious perceptions therefore implies arahantship.}\\
And not aware of perverted perceptions;\pagenote{\pali{na visaññasaññī}: not aware of perverted perceptions. At A.2.52, \pali{visaññino} is used as a synonym of \pali{saññāvipallāso}, perverted perception. This is defined as seeing permanence, happiness, Self and beauty where there is none (\pali{anicce niccan'ti; dukkhe sukhan'ti; anattani attā'ti; asubhe subhan'ti}).}\\
And not unaware,\pagenote{For material form to be annihilated, a person must be not unaware (\pali{nopi asaññī}). The line implies that for arahantship, awareness must be functioning, as well as the undistorted perception alluded to in the first two lines. \pali{Asaññī} perhaps refers to the Unaware Gods (\pali{asaññasattā nāma devā}) (D.1.28). According to this verse, they cannot annihilate their material form. The Buddha may have mentioned this in case people assume that the goal of the holy life is to achieve a state of unawareness. \pali{Asaññī} cannot here mean the exalted state of \pali{saññāvedayitanirodhaṃ} (ending of perception and sensation: S.2.151), in which mental activities have ceased (\pali{cittasaṅkhārā niruddhā}), which is a meditation state that always results in arahantship (M.3.45; M.3.28), and where the word \pali{saññā} means perception, not awareness.}\\
And not with awareness destroyed:\pagenote{And not with awareness destroyed (\pali{na vibhūtasaññī}). \pali{Saññī} must again mean awareness not perception. Because destruction of perception is a meditation state that only arahants attain (Sn.v.1113; A.5.325).}\\
For one arriving at this,\newline material form is annihilated.\pagenote{For one arriving at this, material form is annihilated. This seems to contradict D.1.223 where the Buddha said one should not ask where the material elements are annihilated (\pali{aparisesā nirujjhanti}). Rather, one should ask where they find no footing (\pali{na gādhati}). Perhaps `annihilation of material form' means the death of the arahant where no further body is taken up (S.4.60). `Finding no footing' in fact refers to \pali{viññāṇaṃ anidassanaṃ anantaṃ sabbato pahaṃ}, which is the mind of the living arahant. See: What is the consciousness of an arahant? page \pageref{arahant-consciousness}.}\\
Fictitious perceptions are indeed\newline the source of the conception of `me'.\pagenote{Fictitious perceptions are indeed the source of the notion of `me': perhaps the Buddha added this comment to emphasise that the notion of `me' is what sustains material form, because the notion of `me' is linked to re-birth. One who overcomes the notion of `me' is free of birth, ageing and death (M.3.246) and realises Nibbāna here and now (\pali{anattasaññi asmimānasamugghātaṃ pāpuṇāti diṭṭheva dhamme nibbānaṃ'ti}) (Ud.37).}

\pali{na saññasaññī na visaññasaññī,\\
nopi asaññī na vibhūtasaññī.\\
evaṃsametassa vibhoti rūpaṃ,\\
saññānidānā hi papañcasaṅkhā.}

(Questioner)

\verseref{875} You have explained what we asked.\\
We ask one more thing. Please say!\\
Do the wise say\newline that just this much is the summit,\\
That purity of spirit\newline is to be found in this world?\pagenote{\pali{idha}: means `in this world' at v.801 also.}\\
Or do they say\newline that it is found somewhere other than this?

\pali{yaṃ taṃ apucchimha akittayī no,\\
aññaṃ taṃ pucchāma tadiṅgha brūhi.\\
ettāvataggaṃ nu vadanti heke,\\
yakkhassa suddhiṃ idha paṇḍitāse.\\
udāhu aññampi vadanti etto.}

(The Buddha)

\verseref{876} Some of the wise say\newline that just this much is the summit,\\
That purity of spirit is found here in this world.\\
But some so-called experts say that it is only\newline at time of the arahant's passing away.\pagenote{`the arahant's passing away': the suttas distinguish two elements of Nibbāna: Nibbāna with residue, and Nibbāna without residue. The Nibbāna-element with residue (\pali{saupādisesā nibbānadhātu}) means the destruction of attachment, hatred and delusion by the arahant (\pali{tassa yo rāgakkhayo dosakkhayo mohakkhayo, ayaṃ vuccati saupādisesā nibbānadhātu}). Being `with residue' means the arahant continues to experience pleasure and pain, because of his unimpaired sense faculties -- these sense faculties being called the `residue'. The Nibbāna-element without residue (\pali{an\-upā\-disesā nibbāna\-dhātu}) refers to the final passing of the arahant, who utterly abandons all modes of being and attains the heart of the Teaching (\pali{dhammasārādhigamā}) (It.38-9).}

\pali{ettāvataggampi vadanti heke,\\
yakkhassa suddhiṃ idha paṇḍitāse.\\
tesaṃ paneke samayaṃ vadanti,\\
anupādisese kusalā vadānā.}

\verseref{877} The investigating sage knows\newline that these so-called experts are tethered\\
And he knows what they are tethered to.\\
Knowing, liberated, he does not dispute.\\
The wise man does not return\newline to any form of existence.

\pali{ete ca ñatvā upanissitāti,\\
ñatvā munī nissaye so vimaṃsī.\\
ñatvā vimutto na vivādameti,\\
bhavābhavāya na sameti dhīro'ti.}

\end{verse}
