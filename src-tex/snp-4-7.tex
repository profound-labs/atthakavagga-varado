
\begin{verse}

(Tissa Metteyya)

\verseref{814} Tell us the trouble, sir,\newline that befalls one given to sexual intercourse.\\
Having heard your teaching\newline we will train ourselves in solitude.

\pali{methunamanuyuttassa,\\
vighātaṃ brūhi mārisa.\\
sutvāna tava sāsanaṃ,\\
viveke sikkhissāmase.}

(The Buddha)

\verseref{815} In one given to sexual intercourse\\
The teaching is forgotten\newline and he conducts himself wrongly.\\
This is dishonourable of him.

\pali{methunamanuyuttassa,\\
mussate vāpi sāsanaṃ.\\
micchā ca paṭipajjati,\\
etaṃ tasmiṃ anāriyaṃ.}

\verseref{816} Whoever formerly fared alone,\\
Who then indulges in sexual intercourse,\\
In the world is called\\
A `lurching vehicle',\\
`Contemptible',\\
A `common worldling'.

\pali{eko pubbe caritvāna,\\
methunaṃ yo nisevati.\\
yānaṃ bhantaṃ va taṃ loke,\\
hīnamāhu puthujjanaṃ.}

\verseref{817} His earlier glory and reputation is lost.\\
Seeing this,\\
One should train oneself\newline to forsake one's sexual inclinations.

\pali{yaso kitti ca yā pubbe,\\
hāyate vāpi tassa sā.\\
etampi disvā sikkhetha,\\
methunaṃ vippahātave.}

\verseref{818} Overcome by thought,\newline he broods like a miserable wretch.\\
On hearing others' criticism,\newline he becomes downcast.

\pali{saṅkappehi pareto so,\\
kapaṇo viya jhāyati.\\
sutvā paresaṃ nigghosaṃ,\\
maṅku hoti tathāvidho.}

\verseref{819} Or, provoked by the rumours against him,\\
He retaliates,\\
Or sinks to false speech.\\
Such, indeed, is his great worldly attachment.\pagenote{Worldly attachment: although \pali{gedha} means greed, I call it `worldly attachment' because the PED says the two words are closely linked. Norman calls it `great entanglement'. It seems that blame, one of the eight worldly states, is what the disrobed monk is attached to.}

\pali{atha satthāni kurute,\\
paravādehi codito.\\
esa khvassa mahāgedho,\\
mosavajjaṃ pagāhati.}

\verseref{820} They called him wise\newline when he was committed to faring alone.\\
But now that he is given to sexual intercourse\\
He is harassed as a fool.

\pali{paṇḍitoti samaññāto,\\
ekacariyaṃ adhiṭṭhito.\\
athāpi methune yutto,\\
mandova parikissati.}

\verseref{821} Having realised the wretchedness of all this,\\
The sage for his whole life\newline remains firmly committed to faring alone.\\
He does not pursue sexual intercourse.

\pali{etamādīnavaṃ ñatvā,\\
muni pubbāpare idha.\\
ekacariyaṃ daḷhaṃ kayirā,\\
na nisevetha methunaṃ.}

\verseref{822} He should indeed train himself in solitude.\\
For noble ones this is the supreme training.\\
But he should not suppose\newline that he is therefore `the best'.\\
He is indeed at freedom's threshold.

\pali{vivekaññeva sikkhetha,\\
etadariyānamuttamaṃ.\\
na tena seṭṭho maññetha,\\
sa ve nibbānasantike.}

\verseref{823} The emancipated sage\\
Abides indifferent to sensual pleasure.\\
People enslaved by sensual pleasure envy him,\\
The flood-crosser.

\pali{rittassa munino carato,\\
kāmesu anapekkhino.\\
oghatiṇṇassa pihayanti,\\
kāmesu gadhitā pajā'ti.}

\end{verse}
