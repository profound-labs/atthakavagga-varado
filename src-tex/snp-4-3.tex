
\begin{verse}

\verseref{780} Those who are evil-minded dispute\newline -- of course.\\
But some whose hearts are set on Truth\newline also dispute.\\
However, a sage does not enter a dispute\newline that has arisen.\\
Therefore, he nowhere inclines\newline to hard heartedness.\pagenote{hard heartedness: \pali{khilo}, see Translation Notes, page \pageref{transl-hard-heartedness}.}

\pali{vadanti ve duṭṭhamanāpi eke,\\
athopi ve saccamanā vadanti.\\
vādañca jātaṃ muni no upeti,\\
tasmā munī natthi khilo kuhiñci.}

\verseref{781} How could someone motivated by longing,\\
Bent on pleasure,\\
Overcome the fixed views\newline which he has himself concocted?\\
Having come to his own conclusions,\\
Then, just as he sees things,\\
So would he speak.

\pali{sakañhi diṭṭhiṃ kathamaccayeyya,\\
chandānunīto ruciyā niviṭṭho.\\
sayaṃ samattāni pakubbamāno,\\
yathā hi jāneyya tathā vadeyya.}

\verseref{782} Whoever unasked,\\
Boasts to others of his precepts and practices,\\
Speaking of himself of his own accord,\\
Is ignoble, say the good.

\pali{yo attano sīlavatāni jantu,\\
anānupuṭṭhova paresa pāva.\\
anariyadhammaṃ kusalā tamāhu,\\
yo ātumānaṃ sayameva pāva.}

\verseref{783} But a monk who is peaceful,\\
Having completely extinguished the illusion of Self,\\
Who does not boast about his virtue, `I am like this',\\
Who is not conceited about anything in the world\\
Is noble, say the good.

\pali{santo ca bhikkhu abhinibbutatto,\\
itihanti sīlesu akatthamāno.\\
tamariyadhammaṃ kusalā vadanti,\\
yassussadā natthi kuhiñci loke.}

\verseref{784} He whose religious teachings have been\\
Concocted,\\
Conjured up,\\
And blindly followed\\
Is not cleansed.\pagenote{\pali{vīvadātā}: derived from \pali{odāta}, which PED says is an adjective and a past participle.}\\
Whatever good result from them\newline he might see in himself,\\
If he is tethered to that result,\\
Any satisfaction he feels is dependent\newline on what is unstable.

\clearpage

\pali{pakappitā saṅkhatā yassa dhammā,\\
purakkhatā santi avīvadātā.\\
yadattani passati ānisaṃsaṃ,\\
taṃ nissito kuppapaṭicca santiṃ.}

\verseref{785} It is not easy to transcend opinionatedness\pagenote{Opinionatedness: \pali{diṭṭhīnivesā}, attachment to views.}\newline in regards to dogmatic religious teachings.\pagenote{in regards to dogmatic religious teachings: \pali{dhammesu niccheyya samuggahītaṃ}, see Translation Notes, \pageref{transl-dogmatic-religious-teachings}.}\\
Because of this, men reject or accept religious\newline teachings in accordance with their opinions.

\pali{diṭṭhīnivesā na hi svātivattā,\\
dhammesu niccheyya samuggahītaṃ.\\
tasmā naro tesu nivesanesu,\\
nirassatī ādiyatī ca dhammaṃ.}

\verseref{786} One who is purified\\
Concocts no fixed view about anything in existence.\\
Having forsaken deceit and pride,\\
By what attachment would such a person go?\\
He is without attachment.

\pali{dhonassa hi natthi kuhiñci loke,\\
pakappitā diṭṭhi bhavābhavesu.\\
māyañca mānañca pahāya dhono,\\
sa kena gaccheyya anūpayo so.}

\verseref{787} One who is attached argues\newline over religious teachings.\\
But how, and about what, can you argue\newline with one who is without attachment?\\
For him there is nothing clung to,\newline and nothing to relinquish.\\
He has shaken off every fixed view\newline in this very world.

\pali{upayo hi dhammesu upeti vādaṃ,\\
anūpayaṃ kena kathaṃ vadeyya.\\
attā nirattā na hi tassa atthi,\\
adhosi so diṭṭhimidheva sabbanti.}

\end{verse}

