
Preface

by Bhante Varado 
Preface to the Third Edition

I revised several verses in the Second Edition.
Bhante Varado
Polgahawela
22 April 2008
Preface to the Second Edition

I dedicate this edition to William Stede, to whom, in 1916, T.W. Rhys Davids entrusted the work of editing the Pali Text Society's Pali-English Dictionary (referred to as ‘PED'). I have made good use of Stede's interpretation of Pali vocabulary. This has been useful when commentaries have dominated the interpretation of some Dhammapada verses, and hence many modern translations. Perhaps the most well-known instance is Verse 6, in which the word yamāmase occurs. I have written footnotes to some of these cases.
Having excluded a hundred verses from the last edition, this edition, using a different approach, I have excluded none. In cases where the Pali cannot be gracefully put as poetry, I have used prose. The benefit of poetry, with its pronounced rhyme and rhythm is seen most clearly in Chapter 22, The Abyss, which is otherwise rather solemn.
Some Dhammapada verses are obviously compilations of two shorter verses. This can lead to strange breaks in logical flow. For instance, if verse 146 was really a single verse, it would read: ‘For what the mirth and jubilation in this endless conflagration? Blind in the black of the night: won’t you endeavour to seek for a light?’ I have split such verses into more logical units, named ‘a’ or ‘b’.
Appropriately, this edition was prepared in Sri Lanka, where Samanera Bodhesako conceived the project. I would like to thank Bhikkhu Dhammajoti, also in Sri Lanka, for his interest in publishing this on the web. Both of us are associated with the Mahamevnawa monasteries.
Bhante Varado
Polgahawela
28 February 2008
 
Preface to the First Edition

Translating the Dhammapada into verse has been a challenge that several people have undertaken. This present translation was inspired by the attempt of Samanera Bodhesako, who was an American man, ordained as a Buddhist novice monk from 1966-71, and then again from 1980 until his death in 1988. Most of his monastic life was spent in Sri Lanka. It seems that he composed verses on his travels, often by foot, around that island. He was planning to incorporate his translation into a large coffee-table book with cross references to other parts of the Pali canon. It was going to be a comprehensive guide to Buddhism. However, he died suddenly in Nepal, aged 49, and the work was left incomplete.
When I was in Thailand in 1999, Bhikkhu Nirodho, an Australian monk, showed me the original manuscript. Being intrigued by the work, I tried to see whether it was possible to improve on it, hoping thereby to copy the style of the original Pali text which is also in verse. I was not, however, trying to produce refined poetry - a task best left to genuine poets. Some of the verses in this present translation are more or less Samanera Bodhesako's. I have marked these with an asterisk. In addition, the last line in Verse 33 is Francis Story's.
My translation is incomplete for two reasons. Firstly, having translated all of the Pali text, I decided to exclude approximately a hundred verses. Secondly, although the Dhammapada is directed primarily at men, the Buddha said that the Sangha was complete only if it included accomplished laywomen and ordained women (evam tam brahmacariyam paripuram hoti) (D.3.124-5). I therefore did my best to reflect a less male-orientated attitude to Dhamma in this translation.
I have not included any notes. This is partly a matter of personal taste, but also because commentaries to the Dhammapada are widely available elsewhere. For notes on the Dhammapada and Buddhism, I recommend the Access to Insight website.
I have consulted many other translations, but I would like to especially thank Miroslav Rozehnal for his word-by-word analysis.
I would like to thank Bhikkhu Nirodho for his sustained interest in the project and recommend his website where Samanera Bodhesako’s autobiography can be found. [More of Bodhesako's writings are here].
I would like to thank the peoples of Thailand, Malaysia and U.K. in whose happy company I undertook most of this work. I would also like to thank all the people who offered me suggestions to improve the text, and my mother for buying me necessary books.
Bhante Varado
   23 March 2005