
\begin{verse}

\verseref{766} If a person craving for sensual pleasure is satisfied,\\
He's delighted, yes,\\
The mortal who gets what he wants.

\pali{kāmaṃ kāmayamānassa,\\
tassa ce taṃ samijjhati.\\
addhā pītimano hoti,\\
laddhā macco yadicchati.}

\verseref{767} But that person,\\
Craving and longing,\\
If his pleasures diminish,\\
He suffers as if pierced with an arrow.

\pali{tassa ce kāmayānassa,\\
chandajātassa jantuno.\\
te kāmā parihāyanti,\\
sallaviddhova ruppati.}

\verseref{768} Whoever, attentive, avoids sensual pleasure\\
As he might, with his foot, the head of a snake,\\
Leaves behind this attachment to the world.

\pali{yo kāme parivajjeti,\\
sappasseva padā siro.\\
somaṃ visattikaṃ loke,\\
sato samativattati.}

\verseref{769}\verseref{770} A man who is greedy\\
For fields, property and gold,\\
Cattle and horses,\\
Slaves, servants, maids and relatives,\\
And many sensual pleasures\\
Is overpowered by what is weak\\
And is crushed by troubles.\\
Sorrow invades him like water into a leaky boat.

\pali{khettaṃ vatthuṃ hiraññaṃ vā,\\
gavassaṃ dāsaporisaṃ.\\
thiyo bandhū puthu kāme,\\
yo naro anugijjhati.\\
abalā naṃ balīyanti,\\
maddantenaṃ parissayā.\\
tato naṃ dukkhamanveti,\\
nāvaṃ bhinnamivodakaṃ.}

\verseref{771} So a person, ever attentive,\\
Should avoid the objects of desire.\\
Having forsaken them\\
He will cross the flood of sorrow\pagenote{Water leaks into a boat like sorrow into man. Therefore I have called \pali{oghaṃ} the `flood of sorrow'.}\\
Like one, having bailed out a boat,\\
Who reaches the further shore.

\pali{tasmā jantu sadā sato,\\
kāmāni parivajjaye.\\
te pahāya tare oghaṃ,\\
nāvaṃ sitvāva pāragūti.}

\end{verse}
