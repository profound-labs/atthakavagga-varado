
\begin{verse}

(Questioner)

\verseref{895} For those who dispute, maintaining a fixed view,\\
Saying `This is very Truth',\\
Is criticism all that they experience?\\
Do they not indeed also receive praise?

\pali{ye kecime diṭṭhiparibbasānā,\\
idameva saccanti vivādayanti.\\
sabbeva te nindamanvānayanti,\\
atho pasaṃsampi labhanti tattha.}

(The Buddha)

\verseref{896} What praise they receive is trifling,\\
Not enough to bring them any consolation.\\
I say that disputes have only two fruits:\\
Praise and criticism.\\
Seeing this, you should not dispute.\\
Regard instead non-dispute\newline as the grounds for peace.

\pali{appañhi etaṃ na alaṃ samāya,\\
duve vivādassa phalāni brūmi.\\
etampi disvā na vivādayetha,\\
khemābhipassaṃ avivādabhūmiṃ.}

\verseref{897} Those who are wise do not involve themselves\newline with commonplace opinions.\\
If someone is without attachment,\\
Why would he then become involved?\\
He is someone who does not relish\pagenote{Relish (\pali{khantimakubbamāno}):\newline Norman calls it `preference' though his notes say that \pali{khantimakubbamāno} is synonymous with \pali{pemaṃ akaronto}. \pali{Khanti} occurs in v.944 in the parallel phrase: `He should not be nostalgic (\pali{nābhi\-nan\-deyya}) about the past / Nor relish (\pali{khantiṃ na kubbaye}) what is new'. Here \pali{khanti} obviously correlates with \pali{abhinandati}.}\newline what is seen or heard.

\pali{yā kācimā sammutiyo puthujjā,\\
sabbāva etā na upeti vidvā.\\
anūpayo so upayaṃ kimeyya,\\
diṭṭhe sute khantimakubbamāno.}

\verseref{898} Those for whom virtuous conduct\newline is the supreme practice\\
Say that purification is intrinsic\newline to self-restraint.\\
Having undertaken such a practice,\newline they dedicate themselves to it.\\
They think:\\
`We should train ourselves in just this,\newline for it is purification'.\pagenote{`We should train ourselves in just this, for it is purification': This is an example of v.794's concocting a religious teaching and blindly following it.}\\
These so-called experts are thus led on\newline to further existence.

\pali{sīluttamā saññamenāhu suddhiṃ,\\
vataṃ samādāya upaṭṭhitāse.\\
idheva sikkhema athassa suddhiṃ,\\
bhavūpanītā kusalā vadānā.}

\verseref{899} But if someone like this\newline falls from his precepts and practices\\
He is agitated, having failed in conduct.\\
He hungers and longs for purity\\
Like a wretched merchant\newline living far away, for his home.

\pali{sace cuto sīlavatato hoti,\\
pavedhatī kamma virādhayitvā.\\
pajappatī patthayatī ca suddhiṃ,\\
satthāva hīno pavasaṃ gharamhā.}

\verseref{900} But one who has forsaken precepts and practices,\pagenote{Forsaken precepts and practices: see Should monks forsake precepts and practices? page \pageref{should-monks-forsake}.}\\
And all conduct, both flawed and not flawed,\\
Not wishing for either purity or impurity,\\
Would abide abstaining\newline from initiating new kamma,\pagenote{Abstaining from initiating new kamma, see: Is the arahant free of kamma? page \pageref{arahant-free-of-kamma}. In this sentence, \pali{virato care} has no object. Because the verse is about kamma, I take it to mean `abstaining from initiating new kamma', because the same concept also occurs at v.953 (\pali{virato so viyārambhā}). Norman translates virato care as `he would dwell detached', whereas in v.943 and v.953, where there is an object, he calls \pali{virato} `abstaining'.}\\
Peaceful, free of grasping.\pagenote{`Peaceful, free of grasping': I take \pali{santimanuggahāya} to be synonymous with \pali{anuggahāya santo} of v.839.}

\pali{sīlabbataṃ vāpi pahāya sabbaṃ,\\
kammañca sāvajjanavajjametaṃ.\\
suddhiṃ asuddhinti apatthayāno,\\
virato care santimanuggahāya.}

\verseref{901} Tethered to ascetic practices and self-mortification\\
Or to what is seen, heard or cognised,\\
With raised voices they wail for purification,\\
Not free of clinging to existence.

\pali{tamūpanissāya jigucchitaṃ vā,\\
athavāpi diṭṭhaṃ va sutaṃ mutaṃ vā.\\
uddhaṃsarā suddhimanutthunanti,\\
avītataṇhāse bhavābhavesu.}

\verseref{902} One with wishes is indeed hungering.\\
With regards to his own concocted views\newline about existence,\pagenote{`Concocted views about existence': \pali{pakappitā}. `Concoct' is related in the Octads to a variety of subjects: religious teachings (\pali{dhammā}) (v.784); views about existence (\pali{diṭṭhi bhavābhavesu})\linebreak (v.786); concepts about what is seen, heard or cognised (v.802); opinions (\pali{vinicchayā}) (v.838); sophistry (v.886); views (\pali{diṭṭhi}) (v.910). I have followed the subject suggested in the last line of the previous verse.} there is anxiety.\\
But one for whom\newline there is neither death nor rearising,\\
Why would he be anxious?\\
For what would he hunger?

\pali{patthayamānassa hi jappitāni,\\
pavedhitaṃ vāpi pakappitesu.\\
cutūpapāto idha yassa natthi,\\
sa kena vedheyya kuhiṃva jappe.}

(Questioner)

\verseref{903} The teachings that some call the highest Goal,\newline others call contemptible.\\
Which statement\newline of all of these so-called experts is true?\pagenote{`True': \pali{sacca} is an adjective; \pali{saccaṃ} (next verse) a noun, Truth.}

\pali{yamāhu dhammaṃ paramanti eke,\\
tameva hīnanti panāhu aññe.\\
sacco nu vādo katamo imesaṃ,\\
sabbeva hīme kusalā vadānā.}

\verseref{904} They each say their own teachings are perfect,\\
While the teachings of others they call contemptible.\\
Thus contentious, they squabble.\\
Each one says their own opinion is Truth.

\pali{sakañhi dhammaṃ paripuṇṇamāhu,\\
aññassa dhammaṃ pana hīnamāhu.\\
evampi viggayha vivādayanti,\\
sakaṃ sakaṃ sammutimāhu saccaṃ.}

(The Buddha)

\verseref{905} If a teaching becomes contemptible\newline because an opponent reviles it\\
Then none of the teachings have any merit,\\
For each person says\newline that the others' teachings are contemptible\\
Whilst steadfastly asserting their own.

\pali{parassa ce vambhayitena hīno,\\
na koci dhammesu visesi assa.\\
puthū hi aññassa vadanti dhammaṃ,\\
nihīnato samhi daḷhaṃ vadānā.}

\verseref{906} Just as they honour their own teachings,\\
So they praise their own paths.\\
If all their statements were true,\\
Purity would, of course, be individually theirs.

\pali{saddhammapūjāpi nesaṃ tatheva,\\
yathā pasaṃsanti sakāyanāni.\\
sabbeva vādā tathiyā bhaveyyuṃ,\\
suddhī hi nesaṃ paccattameva.}

\verseref{907} In regards to dogmatic teachings,\pagenote{In regards to dogmatic teachings, a Brahman has no attachment (\pali{na \ldots{} atthi dhammesu niccheyya samuggahītaṃ} -- `not anything in regards to dogmatic religious teachings'): I take the `anything' to mean attachment (i.e. the Brahman has no attachment) because the phrase \pali{dhammesu niccheyya samuggahītaṃ} is linked to \pali{nivesanā} (attachment) at v.801: `no attachment to dogmatic religious teachings'.}\\
A Brahman has no attachment\newline that could be inferred in him by others.\\
Therefore he has gone beyond disputes.\\
He does not regard the mere knowledge\newline of a religious teaching as best.

\pali{na brāhmaṇassa paraneyyamatthi,\\
dhammesu niccheyya samuggahītaṃ.\\
tasmā vivādāni upātivatto,\\
na hi seṭṭhato passati dhammamaññaṃ.}

\verseref{908} Some say:\\
`I know. I see. This is precisely how it is:\\
Some attain purity by means of views'.\\
Even if someone has seen something,\newline what use is it to him?\\
He has gone too far:\\
He speaks of purification as intrinsic\newline to an auxiliary basis of attachment.\pagenote{An auxiliary basis of attachment: \pali{aññena}, see Translation Notes, page \pageref{transl-auxiliary-basis}.}

\pali{jānāmi passāmi tatheva etaṃ,\\
diṭṭhiyā eke paccenti suddhiṃ.\\
addakkhi ce kiñhi tumassa tena,\\
atisitvā aññena vadanti suddhiṃ.}

\verseref{909} A person, in seeing,\newline sees only physical and mental phenomena.\\
Having seen, he will know just that much.\\
Whether he sees a little or a lot,\\
The good do not say\newline that purification is intrinsic to that.

\pali{passaṃ naro dakkhati nāmarūpaṃ,\\
disvāna vā ñassati tānimeva.\\
kāmaṃ bahuṃ passatu appakaṃ vā,\\
na hi tena suddhiṃ kusalā vadanti.}

\verseref{910} A person with rigid views\newline does not easily understand this.\\
He blindly follows the views he has concocted.\\
Wherever he is tethered is his so-called `sanctity'.\\
He calls it `purification'.\\
It is there that he sees Truth.

\pali{nivissavādī na hi subbināyo,\\
pakappitaṃ diṭṭhi purekkharāno.\\
yaṃ nissito tattha subhaṃ vadāno,\\
suddhiṃvado tattha tathaddasā so.}

\verseref{911} The Brahman cannot be reckoned in terms of time.\\
He does not blindly follow views.\pagenote{`Blindly follow': I take \pali{sāreti} (\pali{sārī}) to be a synonym of \pali{purakkharoti}.}\\
He is not bound even to knowledge.\\
And having recognised commonplace opinions\newline which other people grasp,\\
He remains indifferent to them.

\pali{na brāhmaṇo kappamupeti saṅkhā,\\
na diṭṭhisārī napi ñāṇabandhu.\\
ñatvā ca so sammutiyo puthujjā,\\
upekkhatī uggahaṇanti maññe.}

\verseref{912} Having loosened his bonds in the world,\\
The sage does not take sides\newline when disputes have arisen.\\
Amongst those not at peace, he is at peace.\\
He remains equanimous,\\
Not grasping what other people grasp.

\pali{vissajja ganthāni munīdha loke,\\
vivādajātesu na vaggasārī.\\
santo asantesu upekkhako so,\\
anuggaho uggahaṇanti maññe.}

\verseref{913} Forsaking old blinding tendencies,\pagenote{Blinding tendencies: \pali{āsavā}, see Translation Notes, page \pageref{transl-blinding-tendencies}.}\\
Not cultivating any new ones,\\
He is not governed by longing.\\
He is not dogmatic.\\
He is totally liberated from opinionatedness.\\
He is wise.\\
He is not stained by the world.\pagenote{What is the stain of the world? Verses 778-9 say possessiveness (\pali{pariggahesu}) is the stain (\pali{lippatī}). Verses 811-2 say lamentation and selfishness (\pali{paridevamaccharaṃ}) are the stain (\pali{limpati}).}\pagenote{Stained by the world (\pali{loke}). Duroiselle (para 601, xiv) says the locative case is extensively used instead of other cases. Here I take \pali{loke} as instrumental case.}\\
He does not blame himself.\pagenote{`He does not blame himself', see Do arahants blame themselves? page \pageref{arahant-blame-themselves}.}

\pali{pubbāsave hitvā nave akubbaṃ,\\
na chandagū nopi nivissavādī.\\
sa vippamutto diṭṭhigatehi dhīro,\\
na limpati loke anattagarahī.}

\verseref{914} He is peaceful amidst all things,\newline whether seen, heard or cognised.\\
His burden is laid down.\\
The sage is totally liberated.\\
He neither restrains himself from what is temporal\\
Nor yearns for it.

\pali{sa sabbadhammesu visenibhūto,\\
yaṃ kiñci diṭṭhaṃ va sutaṃ mutaṃ vā.\\
sa pannabhāro muni vippamutto,\\
na kappiyo nūparato na patthiyoti.}

\end{verse}
