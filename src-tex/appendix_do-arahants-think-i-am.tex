
It is by taking possession of the five khandhas that the presumption `I am' arises (\pali{upādāya asmi'ti hoti}) (S.3.105). When one perceives not-Self one removes the presumption of a `me'. This is Nibbāna here and now (\pali{anattasaññi asmimānasamugghātaṃ pāpuṇāti diṭṭheva dhamme nibbānaṃ'ti}) (Ud.37). So, for instance, when Venerable Sāriputta entered first jhāna he did not think `I attained first jhāna' (\pali{ahaṃ paṭhamaṃ jhānaṃ samāpannoti}) or `I emerged from first jhāna' (\pali{ahaṃ paṭhamā jhānā vuṭṭhitoti}) (S.3.235).

Nonetheless, an arahant might say `I speak' (\pali{ahaṃ vadāmī'ti}) or `They speak to me' (\pali{mamaṃ vadantī'ti}). Skilful, knowing the world's expressions, he expresses himself using everyday language (\pali{vohāramattena so vohareyyāti}) (S.1.14). Therefore, the Buddha would ask `What does the Sa\.ngha expect of me?' (\pali{kimpanānanda bhikkhusaṅgho mayi paccāsiṃsati}) (D.2.100); or he would exclaim `Ānanda, I am now old, worn out, venerable, one who has travelled life's path, I have reached the term of life, which is eighty' (\pali{ahaṃ kho panānanda etarahi jiṇṇo vuddho mahallako addhagato vayo anuppatto; āsītiko me vayo vattati}) (D.2.100). He even said `All-conquering, am I: all things do I know' (\pali{sabbābhibhu sabbavidu'hamasmi}) (Dhp.353). And he claimed `I am perfected in the world; I am the supreme teacher; I alone am completely awakened; I am become cool, and attained Nibbāna' (\pali{ahaṃ hi arahā loke ahaṃ satthā anuttaro, eko'mhi sammāsambuddho sītibhutosmi nibbuto}) (Vin.1.8). But this are expressions in everyday language, which the Tathāgata uses `without grasping' (\pali{lokasamaññā lokaniruttiyo lokavohārā lokapaññattiyo yāhi tathāgato voharati aparāmasanti}) (D.1.202); and \pali{aparāmasanti} implies that one does not see things as `me' or `mine' or `my Self' (S.2.94). These statements are not the expression of conceit or ignorance. A monk whose mind is liberated employs the usual way of speech in the world without adhering to it (\pali{yañca loke vuttaṃ teneva voharati aparāmasanti}) (M.1.501).
