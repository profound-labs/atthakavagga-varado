by Ajahn Jāgaro

Somebody gave me a book the other day - a book by the followers of another religious group - which analyses every other religion in a very critical way. The person asked me, ‘Is what they have written here about Buddhism correct?’ Without having looked at the book, I said ‘No, it's not right’. And I said, ‘If what they had written was right, those people would be Buddhists by now!’ It's true also that if one really understands Christianity, one would be a Christian.

One of the last sentences in the book was something like: ‘The search for enlightenment without God is impossible,’ or ‘meaningless’, or something to that effect. I actually agree with that, but I would use different terminology. I would say that any religion is meaningless unless it refers to an ultimate reality, an ultimate truth beyond duality, and offers the means to realise it, or to be with it. So, yes, the search for enlightenment would be absolutely meaningless if there were not the Unconditioned, the Immortal, the Uncreated - Nibbāna, if you wish to use Buddhist terminology. Because that's what enlightenment means: coming to realise that which is beyond mortality, beyond duality, beyond concept and thought, beyond creation. It there wasn't that, there would be no enlightenment. There would be nothing to be enlightened to.

So this is why in Buddhism we stress very much the need for the human being to realise Truth, and not to think about Truth; to realise the Unconditioned, not to think about the Unconditioned; because in Buddhism, not only are we very aware of the limitation of the material, physical idols and symbols, we are also critically aware of the limitations of the mental symbols, in other words, of thought itself. Thought is a very, very limited area of human experience. If you want to know the taste of honey, you don't think about it. You could spend a lifetime thinking about the taste of honey and not know what it is. You've got to get beyond thinking in order to know the taste of honey.

What is the point of a religion if it does not point to something beyond concept and thought? Because all concept and all thought is a creation of the mind; it is mortal and limited. And if one begins to understand this, one then begins to understand how foolish are some of the ideas that religions present, because all those descriptions and ideas are mortal, all are limited, all are dual.

The first images or symbols used for God were very crude ones; but as the human intellect developed, then there was more abstract terminology, all of it still created by man. That's why there is such a variety of symbols and terminologies. That's why there will never be agreement between religions when they just talk and think and try to understand by thinking. Then, of course, the differences remain and those differences will always be potential cause for conflict.

In Buddhism we say that for insight or realisation of Truth to happen, the mind has to be trained in some degree of concentration and tranquillity, because the realisation of the Unconditioned is not a thought. It can never be reached by thought. It's got nothing to do with belief. It is when the mind has attained to a particular level of preparedness, which here, primarily, means the degree of concentration, silence and awareness, that the mind can experience things directly. This is possible for every human being to do, but they must leave behind all that clutters the mind.

You can see why there is so little understanding between people of different religions because they are all bringing along their whole storehouse of idols. I don't mean they bring along their Buddha statue or other statues and compare which one is best, but they bring their ideas and beliefs. This is what they compare. And obviously there's going to be differences.

If we speak of a Christian God, it's obviously a limited God. A Christian God is obviously not a Buddhist one. Or if we consider the Buddhist Truth - well that's limited too. The Buddhist Truth is obviously not the Christian Truth, is it? But the Unconditioned cannot be anybody; it cannot have any limitation, nor have any shades or colouring. This is why we call it the Unconditioned. It means it has no conditions, no limitations, no colour, no form. Therefore it is not Christian, Moslem or Hindu, but it is something that human beings can realise. And the realisation of this brings true peace, because once this is realised, then all those things that normally cause conflict become meaningless. The purpose of religion is to help facilitate this realisation. And whether a religion is good or not depends on how well it points. But that's for each person to discover for themselves. I don't think it is for one religion to judge another. It's up to the followers of any particular religion to validate their own religion.

If I want to validate whether Buddhism is a true Path to enlightenment, how can I do it? By becoming enlightened. I can't validate the Islamic path because I don't know much about it. If someone following that path wants to prove that that is the path to enlightenment, or liberation, or whatever they want to call it, they have to follow it to its limits. It is up to each individual to validate their own path by taking it to the limit and realising. Unless we have some direct experience of the Unconditioned, then all we have are conditioned things: all those things that are so variable, in which are so many areas of differences, problems and conflicts.

So for us it is important to train the mind, to become aloof or secluded, to withdraw beyond experience of the six senses and realise that there is awareness, peace and clarity. That mind is then fit to realise something beyond conditions and mortal experiences. And that is how we validate religion - through realising the Immortal.

First published: Forest Sangha Newsletter, April 1994
Ajahn Jāgaro was previously senior incumbent of Bodhinyana Monastery, Western Australia
