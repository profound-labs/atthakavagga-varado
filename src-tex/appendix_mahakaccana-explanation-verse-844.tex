
\begin{verse}
Having forsaken the home-life: \textit{okaṃ pahāya}\\
Not living in company: \textit{aniketasārī}\\
The sage does not create intimate relationships in the village: \textit{gāme akubbaṃ muni satthavāni}\\
Rid of sensual passion: \textit{kāmehi ritto}\\
Free of yearning: \textit{apurekkharāno}\\
He would not speak to people contentiously: \textit{kathaṃ na viggayha janena kayirāti}

(Verse 844)
\end{verse}

Venerable Mahākaccāna's explanation of this verse is as follows:

\begin{itemize}

\item `living the home-life' (\textit{okasārī hoti}) means one's consciousness (\textit{viññāṇaṃ}) is bound by attachment (\textit{rāgavinibaddha}) to the five khandhas -- because here, the khandhas are called the home of consciousness (\textit{viññāṇassa oko}).

\item `living homeless' (\textit{anokasārī hoti}) means overcoming one's desire and attachment for the five khandhas (\textit{yo chando yo rāgo yā nandi yā taṇhā ye upayūpādānā cetaso adhiṭṭhānābhinivesānusayā \ldots{} anabhāvakatā āyatiṃ anuppādadhammā}). This explanation includes a reference to the practice of the Tathāgata; therefore `living homeless' implies arahantship.

\item `living in company' (\textit{niketasārī hoti}) means the scattering and bondage (\textit{visāravinibandhā}) [of one's mind] in the company of sights, sounds, smells, tastes, touches and mental objects.

\item `not living in company' (\textit{aniketasārī hoti}) means overcoming the scattering and bondage [of one's mind] in the company of sights, sounds, smells, tastes, touches and mental objects. This explanation includes a reference to the practice of the Tathāgata; therefore `not living in company' implies arahantship.

\item `being intimate in the village' (\textit{gāme santhavajāto hoti}) means living in association with laypeople, rejoicing at their happiness, sorrowing at their sorrow, and involving oneself in their affairs and duties.

\item not being intimate in the village (\textit{gāme na santhavajāto hoti}) means avoiding such intimacy.

\item `being not rid of sensual passion' (\textit{kāmehi aritto hoti}) means one is not devoid of lust, longing, affection, thirst, passion, and clinging in regard to sensual pleasures (\textit{kāmesu avigatarāgo hoti avigatachando avigatapemo avigatapipāso avigatapariḷāho avigatataṇho}).

\item `being rid of sensual passion' means being rid of such desires for sensual pleasure.

\item `yearning' (\textit{purekkharāno hoti}) means wishing for the future: May I have such a body in the future! May I have such sensations in the future!

\item `free of yearning' means not having such wishes.

\item `speaking with people contentiously' means telling people `You don't understand this Dhamma and Discipline; but I understand it. you're practising incorrectly; I'm practising correctly' and other such speech .

\item `not speaking with people contentiously' means not talking to people like this (S.3.9-12).

\end{itemize}
