
The Buddha said one cannot find the `one whose knots are cut' (S.1.23). He said that one who sees things as `not me or mine' (\pali{neso hamasmi netaṃ me}) is detached (\pali{virajjati}); and being thus detached cannot be found by Māra and his army, and is secure, gone beyond all fetters (S.1.112). He said his own body stood with the link that bound it to becoming cut, and that as long as his present body subsisted, devas and humans would see him, but at death, devas and humans would see him no more (D.1.46). He would not take possession of a new body (\pali{natthi aññañca kāyaṃ upādiyati}) (S.4.60).

Whether, after death, he continued to exist or not, the Buddha left unexplained. When asked about it, he said `Consider what is unexplained as unexplained' -- because `it is not connected with the goal; it is not fundamental to the holy life; it does not lead to liberation' (M.1.427-432). He said that as he was not apprehended as real and actual in this very life, it would be unfitting to discuss what happens to him after death (S.3.118). He said that because he was liberated from the five aggregates, he was unfathomable (S.4.377).

The suttas distinguish two elements of Nibbāna: Nibbāna with residue, and Nibbāna without residue. The Nibbāna-element with residue (\pali{saupādisesā nibbānadhātu}) means the destruction of greed, hatred and delusion by the arahant (\pali{tassa yo rāgakkhayo dosakkhayo mohakkhayo ayaṃ vuccati saupādisesā nibbānadhātu}). Being `with residue' means the arahant has unimpaired sense faculties, and therefore continues to experience pleasure and pain. The Nibbāna-element without residue (\pali{anupādisesā nibbānadhātu}) refers to the final passing away of the arahant, who utterly abandons all modes of being and attains `the heart of the Teaching' (\pali{dhammasārādhigamā}) (It.38-9). But whether it is Nibbāna with residue, or without, Truth is single (v.884). Therefore the Nibbāna-element, with or without residue, must be single.

But what is the Nibbāna-element? Venerable Mahākotthita put the question like this: Is something or nothing left after the final destruction of the six spheres of sense contact (\pali{channaṃ phassāyatanānaṃ asesavirāganirodhā atthaññaṃ kiñcī}). Because this question concerns the cessation of suffering (\pali{evametassa kevalassa dukkhakkhandhassa nirodho hotī'ti}: S.2.14) it is therefore about Nibbāna. Venerable Sāriputta told Venerable Mahākotthita that to involve oneself with such questions is to form fictitious perceptions about what is beyond fictitious perceptions (\pali{appapañcaṃ papañceti}) (A.2.161). He said `However far the six spheres of contact go, that is how far fictitious perceptions go. However far fictitious perceptions go, that is how far the six spheres of contact go. With the complete destruction of the six spheres of contact there is the complete destruction all fictitious perceptions (\pali{channaṃ āvuso phassāyatanānaṃ asesavirāganirodhā papañcanirodho papañcavūpasamo'ti}) (A.2.161). So when Venerable Mahākotthita repeatedly asked him about the matter, he repeatedly replied `Don't ask that, friend (\pali{māhevaṃ āvuso})'.

In contrast to Venerable Sāriputta's apparent reluctance to discuss the matter, the Buddha was more forthcoming. He described Nibbāna as being \pali{viññāṇa} without attributes, everlasting, completely without a source (\pali{viññāṇaṃ anidassanaṃ, anantaṃ sabbatopabhaṃ}). This is probably synonymous with the `unbroken stream of \pali{viññāṇa} not established either in this world or the next (\pali{idha loke appatiṭṭhitañca paraloke appatiṭṭhitañca}) (D.3.104-5). (See: What is the consciousness of an arahant? page \pageref{arahant-consciousness}.). He said that people entering Nibbāna were like streams and showers that enter the ocean without affecting the ocean's fullness or depletion. In the same way, even if many monks attain the Nibbāna-element (\pali{anupādisesāya nibbānadhātuyā parinibbāyanti}), it in no way affects the fullness or depletion of the Nibbāna element (\pali{na tena nibbānadhātuyā ūnattaṃ vā pūrattaṃ vā paññāyati}) (Vin.2.239). He also described Nibbāna in poetry:

\begin{verse}
Where neither water nor yet earth\\
Nor fire nor air gain a foothold,\\
There gleam no stars, no sun sheds light,\\
There shines no moon, yet there no darkness reigns.

When a sage, a brahman, has come to know this\\
For himself through his own wisdom,\\
Then he is freed from form and formless.\\
Freed from pleasure and from pain.\\
(Tr.Ireland; Ud.9)
\end{verse}

On a different occasion, Venerable Sāriputta, though aloof with Venerable Mahākotthita, chose a more encouraging attitude with Venerable Udāyin. He told him that Nibbāna is happiness (\pali{sukhamidaṃ āvuso nibbānan'ti}) and said that by `Nibbāna', he meant the ending of perception and feeling (\pali{saññāvedayitanirodhaṃ}). This is a meditation state achieved after attaining the sphere of neither perception nor non-perception and is a state consistently associated with the destruction of the \pali{āsavās} by seeing them with wisdom (\pali{paññāya cassa disvā āsavā parikkhīṇā honti}) which is arahantship. And the happiness found there is the happiness where nothing is sensed (\pali{sukhaṃ yadettha natthi vedayitaṃ}) (A.4.414). The Buddha confirmed this, describing the state as a happiness more excellent and sublime than any other state (S.4.228). And this highest happiness, he also called Nibbāna (\pali{nibbāṇaparamaṃ sukhaṃ}) (Dh.v.203). Venerable Anuruddha's statements support this. He said he saw no abiding higher or more sublime than the ending of perception and feeling, and the Buddha supported this statement (M.1.209).

Given that it involves absence of sensation, it is perplexing that one can make any statement at all about the ending of perception and feeling, that one can retain any memory of it, that one would know it is everlasting, and recognise that it is completely without a source. It is even surprising that arahants attaining the ending of perception and feeling are able to recognise and remember that, during the experience, perception and feeling ended (M.1.302). Although this implies some retention of consciousness, the matter is admittedly unfathomable.

Most ancient and modern commentators consider the ending of perception and feeling to be a state that is accessible to non-returners (e.g. Visuddhimagga 702-9). By accessible, they seem to mean that one can emerge from that state having failed to achieve arahantship. Thus, even though the Anupada Sutta says Venerable Sāriputta attained arahantship within that state (M.3.28), the ancient commentary says his arahantship occurred after emerging (MLDB note1052). But there is no evidence in the suttas themselves that non-returners emerge from that state, except in the Gradual Sayings III p.141, where Hare's translation suggests that one who has attained the ending of perception and feeling may in fact be reborn, and attain a mind made body. In fact this sutta quotes Venerable S\=ariputta as saying that for a monk endowed with virtue, sam\=adhi and wisdom, it is possible that he might enter and emerge from the cessation of perception and feeling. If he does not attain final knowledge in his lifetime (i.e. if he does not attain the cessation of perception and feeling in his lifetime), he might transcend the realm of devas that are nourished on gross food and, having attained a mind-made body, it is possible that he might enter and emerge from the cessation of perception and feeling. This statement implies two things:

\begin{enumerate}
\item that a monk endowed with virtue, sam\=adhi and wisdom may not attain the cessation of perception and feeling in his lifetime. In which case, he will, if he is a non-returner, attain it having gained a mind-made body. If he is not a non-returner, he will not attain it in that mind-made body.
\item entering and emerging from the cessation of perception and feeling is synonymous with attaining final knowledge.
\end{enumerate}

The Pāli is this: \pali{Idhāvuso, bhikkhu sīlasampanno samādhisampanno paññāsampanno saññāvedayitanirodhaṃ samāpajjeyya pi vuṭṭhaheyya' pi atthetaṃ ṭhānaṃ. No ce diṭṭheva dhamme aññaṃ ārādheyya, atikkammeva kabaliṅkārāhārabhakkhānaṃ devānaṃ sahavyataṃ aññataraṃ manomayaṃ kāyaṃ upapanno saññāvedayitanirodhaṃ samāpajjeyya'pi vuṭṭhaheyya'pi atthetaṃ ṭhānanti} (A.3.193).

For the living arahant, although arahantship is permanent, and the destruction of greed, hatred and delusion is permanent, attaining the state of Nibbāna is not permanent (A.4.423-6; M.1.302). And neither attaining it nor leaving it involve conscious decisions. Rather, one's mind must be developed in such a way that it leads one into it and out of it (M.1.302). But, contradicting this, Venerable Anuruddha said he could enter it at will (M.1.209) -- as did the Buddha and Venerable Mahākassapa (S.2.212). The mental development that leads one into this state is presumably the destruction of greed, hatred and delusion (\pali{tassa yo rāgakkhayo dosakkhayo mohakkhayo}) -- the usual definition of Nibbāna. It is perplexing that the destruction of these three states is itself called Nibbāna (It.38-9), when Nibbāna seems in fact to be the cessation of perception and feeling.

By Venerable Sāriputta's description, attaining the ending of perception and feeling would be rare for arahants, because many arahants cannot even attain the preliminary step: the sphere of neither perception nor non-perception. The Buddha was once in a group of five hundred arahants, and said that only sixty of them had attained the formless spheres (S.1.191). So, experiencing the ending of perception and feeling would seem rare for arahants. Nonetheless, because Venerable Sāriputta called this state `Nibbāna', most monks must attain it without needing to attain any of the formless spheres. It is strictly speaking incorrect to say `a non-returner gains that state', because non-returnership can only be judged at death. If someone attains arahantship in his lifetime, then during that period before arahantship he cannot strictly be called a non-returner. However, the term is used in this way as a matter of convenience. The same principle would apply to stream-enterers, and other stages of sainthood.

But even if an arahant attains Nibbāna, as long as he remains alive, he continues to experience pleasure and pain (\pali{sukhadukkhaṃ paṭisaṃvedeti}). Perhaps he periodically re-enters Nibbāna, either at will or when his mind leads him into it. When he finally passes away, he attains what the Buddha called `the heart of the Teaching' (\pali{dhammasārādhigamā}) (It.38-9). Because Truth is single, one assumes that `the heart of the Teaching' is that same unparallelled happiness -- the happiness that is everlasting.
