
\begin{verse}

(The Buddha)

\verseref{935} Violence breeds fear.\\
Seeing people in conflict,\newline I will tell you of my dismay,\\
How moved I was.

\pali{attadaṇḍā bhayaṃ jātaṃ,\\
janaṃ passatha medhagaṃ.\\
saṃvegaṃ kittayissāmi,\\
yathā saṃvijitaṃ mayā.}

\verseref{936} I saw people writhing,\\
Feuding with each another like fish\newline in a small pool.\\
When I realised this, dread arose in me.

\pali{phandamānaṃ pajaṃ disvā,\\
macche appodake yathā.\\
aññamaññehi byāruddhe,\\
disvā maṃ bhayamāvisi.}

\verseref{937} The world is entirely worthless.\\
Every quarter is in turmoil.\\
Wanting somewhere for myself,\\
I saw nowhere that wasn't taken.

\pali{samantamasāro loko,\\
disā sabbā sameritā.\\
icchaṃ bhavanamattano,\\
nāddasāsiṃ anositaṃ.}

\verseref{938} Seeing nothing in the end but competition,\newline I became disgusted.\\
Then I saw a spike -- hard to see --\\
Embedded in people's hearts.

\pali{osānetveva byāruddhe,\\
disvā me aratī ahu.\\
athettha sallamaddakkhiṃ,\\
duddasaṃ hadayanissitaṃ.}

\verseref{939} A person affected by this spike\newline rushes about in all directions.\\
But on pulling it out\newline he neither rushes about nor falls away.\pagenote{`Nor falls away': \pali{Sīdati} seems synonymous with \pali{avokkamma} in v.946, meaning `not falling away from Truth'.}

\pali{yena sallena otiṇṇo,\\
disā sabbā vidhāvati.\\
tameva sallamabbuyha,\\
na dhāvati na sīdati.}

(Now follows the recitation of the training rules)

\verseref{940} Whatever is binding in the world\newline you should not pursue it.\\
Having wholly destroyed sense desire,\\
You should train yourself for Nibbāna.

\pali{tattha sikkhānugīyanti,\\
yāni loke gadhitāni\\
na tesu pasuto siyā,\\
nibbijjha sabbaso kāme,\\
sikkhe nibbānamattano.}

\verseref{941} A sage should be truthful,\\
Unassuming, and not deceitful;\\
He should be rid of malicious speech\\
And free of anger.\\
He should overcome greed and acquisitiveness.

\pali{sacco siyā appagabbho,\\
amāyo rittapesuṇo.\\
akkodhano lobhapāpaṃ,\\
vevicchaṃ vitare muni.}

\verseref{942} He should conquer sleepiness, weariness and sloth.\\
He should not live negligently.\\
The man whose heart is set on Nibbāna\newline should not be arrogant.

\pali{niddaṃ tandiṃ sahe thīnaṃ,\\
pamādena na saṃvase.\\
atimāne na tiṭṭheyya,\\
nibbānamanaso naro.}

\verseref{943} He should not sink to false speech,\\
Nor should he cultivate lust for physical forms.\\
He should comprehend pride\\
And should abstain from impetuous behaviour.

\pali{mosavajje na nīyetha,\\
rūpe snehaṃ na kubbaye.\\
mānañca parijāneyya,\\
sāhasā virato care.}

\verseref{944} He should not be nostalgic about the past.\\
Nor relish what is new.\\
He should not grieve for what is lost\\
Nor be bound to whatever comes forth.

\pali{purāṇaṃ nābhinandeyya,\\
nave khantiṃ na kubbaye.\\
hiyyamāne na soceyya,\\
ākāsaṃ na sito siyā.}

\verseref{945} I call greed the `great deluge'.\\
Hunger I call the `torrent'.\\
Concocted religious teachings are the `foothold'.\pagenote{`Concocted religious teachings' (\pali{pakappanaṃ}): `concoct' is related in the Octads to a variety of objects: views about existence (\pali{diṭṭhi bhavābhavesu}) (v.786); religious teachings (\pali{dhammā}) (v.784); fictitious perceptions (v.802); opinions (\pali{vinicchayā}) (v.838); views (\pali{diṭṭhi}) (v.910). The simile here indicates a difficult attempt to cross over greed, lust and sense pleasures. A difficult foothold indicates that concocted religious teachings are being used to achieve this.}\\
Sense pleasure is `hard-to-cross mud'.

\pali{gedhaṃ brūmi mahoghoti,\\
ājavaṃ brūmi jappanaṃ.\\
ārammaṇaṃ pakappanaṃ,\\
kāmapaṅko duraccayo.}

\verseref{946} Not falling away from Truth,\\
The sage, the Brahman, stands on high ground.\\
Having forsaken everything\\
He is truly called peaceful.

\pali{saccā avokkamma muni,\\
thale tiṭṭhati brāhmaṇo.\\
sabbaṃ so paṭinissajja,\\
sa ve santoti vuccati.}

\verseref{947} He indeed is wise.\\
He has perfect insight.\\
Having found Truth, he is untethered.\\
Wandering through the world in the right way\\
He does not envy anyone here.

\pali{sa ve vidvā sa vedagū,\\
ñatvā dhammaṃ anissito.\\
sammā so loke iriyāno,\\
na pihetīdha kassaci.}

\verseref{948} Whoever here transcends sense desire --\\
A bond hard to transcend --\\
Is free of sorrow and anxiety.\\
He has cut the stream of false desire.\pagenote{`Stream': called the stream of \pali{taṇhā} at S.4.292.}\\
He is free of bonds.

\pali{yodha kāme accatari,\\
saṅgaṃ loke duraccayaṃ.\\
na so socati nājjheti,\\
chinnasoto abandhano.}

\verseref{949} Let wither what is gone.\\
Let there not be for you anything to come.\\
If you do not grasp at what is in between\\
You will live at peace.

\pali{yaṃ pubbe taṃ visosehi,\\
pacchā te māhu kiñcanaṃ.\\
majjhe ce no gahessasi,\\
upasanto carissasi.}

\verseref{950} For whom there is nothing beloved\newline in this body / mind complex\\
And who does not grieve\newline because of what does not exist,\pagenote{`Does not grieve for what does not exist', See The Octads in a Nutshell: Abiding in solitude, page \pageref{abiding-in-solitude}.}\\
He suffers no loss in the world.

\pali{sabbaso nāmarūpasmiṃ,\\
yassa natthi mamāyitaṃ.\\
asatā ca na socati,\\
sa ve loke na jīyati.}

\verseref{951} For whoever there is no thought `This is mine'\\
Or `This belongs to others',\\
Who has no feelings of possessiveness,\\
He does not grieve for anything, thinking:\\
`It is not mine'.

\pali{yassa natthi idaṃ meti,\\
paresaṃ vāpi kiñcanaṃ.\\
mamattaṃ so asaṃvindaṃ,\\
natthi meti na socati.}

\verseref{952} Being free of cruelty, greed and lust,\\
And being everywhere tranquil:\\
When asked,\\
I say that these are the blessings\newline for those who are unshakeable.

\pali{aniṭṭhurī ananugiddho,\\
anejo sabbadhī samo.\\
tamānisaṃsaṃ pabrūmi,\\
pucchito avikampinaṃ.}

\verseref{953} For a person free of inner turbulence,\pagenote{Free of inner turbulence: \pali{anejassa} also occurs at v.920 where it is compared to the depths of the ocean where no waves swell up. Its synonym there is \pali{ṭhito}, stable. PED says aneja means `free from desires or lust'.}\\
One of discernment,\\
There is no accumulation of kamma.\\
He abstains from initiating new kamma.\\
He sees safety everywhere.

\clearpage

\pali{anejassa vijānato,\\
natthi kāci nisaṅkhati.\\
virato so viyārabbhā,\\
khemaṃ passati sabbadhi.}

\verseref{954} The sage does not speak of himself\newline as someone equal, inferior or superior.\\
At peace, unselfish,\\
He neither clings nor relinquishes.\pagenote{`Clings nor relinquishes': I take the meaning of \pali{nādeti na nirassatī ti from attā vāpi nirattā} at v.858.}

\pali{na samesu na omesu,\\
na ussesu vadate muni.\\
santo so vītamaccharo,\\
nādeti na nirassatī'ti.}

\end{verse}
