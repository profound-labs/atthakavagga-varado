
\begin{verse}

(Questioner)

\verseref{915} I ask the Kinsman of the Sun, the Great Master,\\
About solitude and the state of peace.\\
Seeing in what way is a monk freed from passion,\\
Possessing nothing in the world?

\pali{pucchāmi taṃ ādiccabandhu,\\
vivekaṃ santipadañca mahesi.\\
kathaṃ disvā nibbāti bhikkhu,\\
anupādiyāno lokasmiṃ kiñci.}

(The Buddha)

\verseref{916} A sage should completely restrain\newline the basis of the conception of `me':\\
The thought `I am'.\pagenote{The basis of the conception of `me' (\pali{mūlaṃ papañcasaṅkhāya}), the thought `I am'. I take the conception of `me' to be the `eighteen thoughts associated with clinging' (\pali{aṭṭhārasa taṇhā\-vicari\-tāni}). The Buddha said when there is the thought `I am' (\pali{asmīti sati}), there comes seventeen other thoughts: `I am in this world' (\pali{ittha\-smīti hoti}); `I am thus'; `I am otherwise'; `I am bad'; `I am good'; `I might be'; `I might be in this world'; `I might be thus'; `I might be otherwise'; `may I be'; `may I be in this world'; `may I be thus'; `may I be otherwise'; `I will be'; `I will be in this world'; `I will be thus'; `I will be otherwise' (A.2.212-3).}\\
Ever attentive, he should train himself\\
To abolish whatever clinging he finds within.\pagenote{See Translation Notes: \pali{papañcasaṅkhāya}, page \pageref{transl-conception-of-me}; \pali{taṇhā}, page \pageref{transl-clinging}; \pali{sato}, page \pageref{transl-attentiveness}.}

\pali{mūlaṃ papañcasaṅkhāya,\\
mantā asmīti sabbamuparundhe.\\
yā kāci taṇhā ajjhattaṃ,\\
tāsaṃ vinayā sadā sato sikkhe.}

\verseref{917} Whatever religious teaching he knows,\\
Either his own or that of others,\\
He should not allow it to be a cause of obstinacy,\pagenote{`A cause of obstinacy': The Buddha said that obstinately holding onto one's own views (\pali{thāmasā parāmassa abhinivissa}) leads to clashes with people of different views. Forseeing this trouble for oneself, one would forsake whatever views one is obstinately attached to and not cling to any others (M.1.498).}\\
For this is not called `peaceful' by the good.

\pali{yaṃ kiñci dhammamabhijaññā,\\
ajjhattaṃ athavāpi bahiddhā.\\
na tena thāmaṃ kubbetha,\\
na hi sā nibbuti sataṃ vuttā.}

\verseref{918} He should not think himself as better,\newline inferior or equal on account of anything.\\
Although affected by a variety of experiences\\
He should not acquiesce in the thought of Self.

\pali{seyyo na tena maññeyya,\\
nīceyyo athavāpi sarikkho.\\
phuṭṭho anekarūpehi,\\
nātumānaṃ vikappayaṃ tiṭṭhe.}

\verseref{919} A monk should find peace within.\\
He should not seek it\newline from some auxiliary basis of attachment.\\
For one who is peaceful within,\\
Having clung to nothing,\\
How could he relinquish anything?

\pali{ajjhattamevupasame,\\
na aññato bhikkhu santimeseyya.\\
ajjhattaṃ upasantassa,\\
natthi attā kuto nirattā vā.}

\verseref{920} Just as the depths of the ocean is stable\\
And waves do not swell up,\\
So, the monk who is stable, free of inner turbulence\\
Would have no swellings of conceit about anything.

\pali{majjhe yathā samuddassa,\\
ūmi no jāyatī ṭhito hoti.\\
evaṃ ṭhito anejassa,\\
ussadaṃ bhikkhu na kareyya kuhiñci.}

(Questioner)

\verseref{921} The Seer, the Witness of Truth,\newline has proclaimed the removal of danger.\\
Now, venerable sir,\newline speak about the path of practice,\\
About monastic discipline,\\
And also about sam\=adhi.

\pali{akittayī vivaṭacakkhu,\\
sakkhidhammaṃ parissayavinayaṃ.\\
paṭipadaṃ vadehi bhaddante,\\
pātimokkhaṃ athavāpi samādhiṃ.}

(The Buddha)

\verseref{922} A person should not have covetous eyes.\\
He should close his ears to ordinary chatter.\\
He should not be greedy for flavours.\\
He should not cherish anything in the world.

\pali{cakkhūhi neva lolassa,\\
gāmakathāya āvaraye sotaṃ.\\
rase ca nānugijjheyya,\\
na ca mamāyetha kiñci lokasmiṃ.}

\verseref{923} In whatever way he is affected by sense contact\\
He should not lament over anything.\\
He should not hunger for states of existence.\\
He should not tremble amidst danger.

\pali{phassena yadā phuṭṭhassa,\\
paridevaṃ bhikkhu na kareyya kuhiñcñcci.\\
bhavañca nābhijappeyya,\\
bheravesu ca na sampavedheyya.}

\verseref{924} He should not store up what is given to him\\
Whether it is food or snacks, drinks or clothing;\\
Nor should he be concerned if he gets nothing.

\pali{annānamatho pānānaṃ,\\
khādanīyānaṃ athopi vatthānaṃ.\\
laddhā na sannidhiṃ kayirā,\\
na ca parittase tāni alabhamāno.}

\verseref{925} He should be meditative, not foot-loose.\\
He should desist from worry.\\
He should not be indolent.\\
He should live in lodgings where there is little noise.

\pali{jhāyī na pādalolassa,\\
virame kukkuccā nappamajjeyya.\\
athāsanesu sayanesu,\\
appasaddesu bhikkhu vihareyya.}

\verseref{926} He should not sleep too much.\\
He should be devoted to wakefulness\newline and keen endeavour.\\
He should forsake laziness, deception, merriment,\\
Various kinds of amusements, sexual matters,\newline and anything else like it.

\clearpage

\pali{niddaṃ na bahulīkareyya,\\
jāgariyaṃ bhajeyya ātāpī.\\
tandiṃ māyaṃ hassaṃ khiḍḍaṃ,\\
methunaṃ vippajahe savibhūsaṃ.}

\verseref{927} A disciple of mine should not practise sorcery\\
Nor interpret dreams, tell fortunes,\newline practise astrology, or interpret animal cries.\\
Neither should he treat infertility,\newline nor practise medicine.

\pali{āthabbaṇaṃ supinaṃ lakkhaṇaṃ,\\
no vidahe athopi nakkhattaṃ.\\
virutañca gabbhakaraṇaṃ,\\
tikicchaṃ māmako na seveyya.}

\verseref{928} A monk should not fear blame,\\
Nor should he be conceited when praised.\\
He should drive out greed, selfishness,\newline anger and malicious speech.

\pali{nindāya nappavedheyya,\\
na uṇṇameyya pasaṃsito bhikkhu.\\
lobhaṃ saha macchariyena,\\
kodhaṃ pesuṇiyañca panudeyya.}

\verseref{929} A monk should not engage in buying and selling.\\
He should not abuse anyone for any reason.\\
He should not linger in the village.\\
He should not chatter with people\newline in the hope of gain.

\pali{kayavikkaye na tiṭṭheyya,\\
upavādaṃ bhikkhu na kareyya kuhiñci.\\
gāme ca nābhisajjeyya,\\
lābhakamyā janaṃ na lapayeyya.}

\verseref{930} A monk should not be a boaster.\\
He should not speak scheming words.\\
He should not cultivate impudence.\\
He should not utter contentious speech.

\pali{na ca katthitā siyā bhikkhu,\\
na ca vācaṃ payuttaṃ bhāseyya.\\
pāgabbhiyaṃ na sikkheyya,\\
kathaṃ viggāhikaṃ na kathayeyya.}

\verseref{931} He should not be drawn into telling lies.\\
He should not be deliberately treacherous.\\
He should not despise others\newline for their lowly way of life,\\
Or wisdom, or precepts and practices.

\pali{mosavajje na nīyetha,\\
sampajāno saṭhāni na kayirā.\\
atha jīvitena paññāya,\\
sīlabbatena nāññamatimaññe.}

\verseref{932} If ascetics or ordinary people\newline irritate him with their talkativeness\\
He should not respond harshly.\\
For the peaceful do not retaliate.

\pali{sutvā rusito bahuṃ vācaṃ,\\
samaṇānaṃ vā puthujanānaṃ.\\
pharusena ne na paṭivajjā,\\
na hi santo paṭisenikaronti.}

\verseref{933} Knowing the Buddha's teaching,\\
An ever attentive monk who investigates it\newline should train himself in it.\\
Knowing the extinguishing\pagenote{`Extinguishing': I take \pali{nibbutiṃ} as referring to the illusion of Self, because v.783 says that a peaceful monk has completely extinguished the illusion of Self (\pali{abhinibbutatto})}\newline of the illusion of Self as Peace,\\
He should not be negligent\newline in applying Gotama's teaching.

\pali{etañca dhammamaññāya,\\
vicinaṃ bhikkhu sadā sato sikkhe.\\
santīti nibbutiṃ ñatvā,\\
sāsane gotamassa na pamajjeyya.}

\verseref{934} The unconquered Conqueror realised Truth\newline through his own insight,\\
Not through hearsay.\\
So, with regards to the Sublime One's teaching,\\
One who is diligent should constantly venerate it\newline by following his example.

\pali{abhibhū hi so anabhibhūto,\\
sakkhidhammamanītihamadassī.\\
tasmā hi tassa bhagavato sāsane,\\
appamatto sadā namassamanusikkhe'ti.}

\end{verse}
