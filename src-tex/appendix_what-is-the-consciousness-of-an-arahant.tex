
\section*{The three vi\~n\~n\=a\d{n}as}

The scriptures describe three types of \pali{viññāṇa}:

\begin{itemize}
\item \pali{viññāṇakkhandha}
\item a \pali{viññāṇa} that is dependent
\item a \pali{viññāṇa} that is not dependent
\end{itemize}

These three viññāṇas were all included when Venerable Sāriputta told Anāthapindika to train himself by reflecting on \pali{viññāṇakkhandha}: `I will not take possession of viññāṇa[kkhanda]; my [dependent] viññāṇa will become not dependent on viññāṇakkhandha' (\pali{na viññāṇaṃ upādiyissāmi; na ca me viññāṇanissitaṃ viññāṇaṃ bhavissatī'ti}) (M.3.260). From this it seems that not-dependent \pali{viññāṇa} arises from dependent \pali{viññāṇa}.

\section*{Vi\~n\~n\=a\d{n}akkhandha}

The \pali{viññāṇa} of \pali{viññāṇakkhandha} is called `\pali{viññāṇa}' because, due to it, `one knows' (\pali{vijānāti}) different sensations, for example: sour, bitter, pungent, sweet, sharp, mild, salty, bland (S.3.87). This \pali{viññāṇa} is sixfold, where each of the six viññāṇas are named according to the sense-base from which it arises: eye-viññāṇa, ear-viññāṇa, nose-viññāṇa, tongue-viññāṇa, body-viññāṇa and mind-viññāṇa (M.1.259; S.3.61). Because this type of \pali{viññāṇa} is abandoned at arahantship, an arahant cannot be reckoned in terms of it (\pali{viññāṇasaṅkhāvimutto}) (M.1.488). This \pali{viññāṇa} therefore seems to be the `making known' that arises when internal sense bases (e.g. the eye) interacts with the external sense-bases (e.g. visible forms). If so, then this \pali{viññāṇa} could be called the `conscious function'; or simply called `making known'.

\section*{Dependent-type vi\~n\~n\=a\d{n}a: the `stream of attachment' or the `stream of me'}

Venerable Sāriputta's advice above assumed the existence of a dependent-type \pali{viññāṇa} (\pali{nissitaṃ viññāṇaṃ}). Venerable Mahākaccāna describes a similar \pali{viññāṇa} in a discussion with the householder Hāliddakāni, where he used the term `consciousness passionately bound to the khandhas' as a synonym of desire and clinging for the khandhas (S.3.10).\footnote{\pali{Rūpadhāturāgavinibaddhañca pana viññāṇaṃ} (one whose consciousness is passionately bound to the form element) is a synonym of \pali{rūpadhātuyā kho yo chando yo rāgo yā nandi yā taṇhā ye upayūpādānā cetaso adhiṭṭhānābhinivesānusayā} (the desire, passion, delight, clinging, the engagement and attachment, the mental standpoints, clinging, and inclination regarding the form element).} This dependent-type \pali{viññāṇa} may be equivalent to the rarely mentioned `unbroken stream of \pali{viññāṇa} that is established in both this world and the next' (\pali{viññāṇasotaṃ abbocchinnaṃ idha loke patiṭṭhitaṃ ca paraloke patiṭṭhitaṃ ca}) (D.3.105). As we have seen, Venerable Sāriputta called it simply \pali{viññāṇa}. But this `\pali{viññāṇa}' may be a term for attachment itself, for instance when it is used as Venerable Mahākaccāna used it. This would mean that the term `unbroken stream of \pali{viññāṇa}' simply means the `unbroken stream of attachment'. If this is the case, it might also be called `unbroken stream of `me' and `mine'', a stream that ends at arahantship (\pali{asmimānasamugghātaṃ pāpuṇāti diṭṭheva dhamme nibbānaṃ'ti}: Ud.37). This type of \pali{viññāṇa} is obviously unknown by arahants. 

In the course of rebirth (\pali{punabbhavābhinibbatti}) the dependent-type \pali{viññāṇa} (or, the unbroken stream of attachment) is grounded (\pali{patiṭṭhitaṃ}) in the worlds of sense desire (\pali{kāmadhātu}), or in the worlds of form (\pali{rūpadhātu}), or in the excellent worlds (\pali{paṇītāya dhātuyā}) -- a process called `becoming' (\pali{bhavo}) (A.1.222). This dependent-type \pali{viññāṇa} is one of the four nutriments, and it is a condition for the production of future renewed existence (\pali{viññāṇāhāro āyatiṃ punabbhavābhinibbattiyā paccayo}) (S.2.13). The source of this stream of \pali{viññāṇa} is said to be \pali{taṇhā} (\pali{taṇhānidānā, taṇhāsamudayā}) (S.2.12); but sometimes it is said to be one's intentions, plans and tendencies (\pali{yañca ceteti yañca pakappeti yañca anuseti}) (S.2.65). If `watered with delight' (\pali{nandūpasecanaṃ}) the stream of \pali{viññāṇa} grows and proliferates (\pali{vuddhiṃ virūḷhiṃ vepullaṃ āpajjeyya}) (S.3.53). When it is established and come to growth, the body and mind arises (\pali{nāmarūpassa avakkanti}). This leads to formation of kamma (\pali{saṅkhārānaṃ vuddhi}) and the production of future renewed existence (\pali{āyatiṃ punabbhavābhinibbatti}) (S.2.101-4). If one abandons attachment to the five khandhas (\pali{rāgo pahīno hoti}) there is no basis for the dependent-type \pali{viññāna} (\pali{patiṭṭhā viññāṇassa na hoti}). This may be equivalent to what the suttas call `cutting the stream' (\pali{chinnasotaṃ}) (S.4.292).

\section*{Not-dependent-type vi\~n\~n\=a\d{n}a: freedom from attachment}

At arahantship, knowledge of the second type of \pali{viññāṇa} is replaced by knowledge of the third type, which could therefore be termed `not-dependent-type \pali{viññāṇa}'. This not-dependent-type \pali{viññāṇa} may equal the rarely mentioned `unbroken stream of \pali{viññāṇa} not established either in this world or the next' (\pali{viññāṇasotaṃ pajānāti ubhayato abbocchinnaṃ idha loke appatiṭṭhitañca paraloke appatiṭṭhitañca}) (D.3.104-5). Venerable Sāriputta also seemed to call this simply \pali{viññāṇa}; but it may be simply a term for non-attachment itself. In which case, it could be called `unbroken stream of non-attachment not established either in this world or the next' or perhaps `an unbroken stream of not me-or-mine not established either in this world or the next'.

The scriptures mention what is possibly another synonym: \pali{viññāṇaṃ anidassanaṃ anantaṃ sabbatopabhaṃ}: \pali{viññāna} without attribute, everlasting, completely without a source (M.1.329; D.1.223). This is associated with arahantship, and is discussed below (S.1.120-1; S.3.122-3). It seems that it is realised with the cessation of the dependent-type \pali{viññāṇa} (i.e. it is realised with the cessation of attachment, or with the cessation of the stream of `me' and `mine') (\pali{viññāṇassa nirodhena}: D.1.223). This realisation occurs when one does not intend, have plans or tendencies (S.2.66).

The liberation of \pali{viññāṇa} is frequently referred to in the suttas.

\begin{quote}
Here, it is the \pali{citta} that is said to be liberated:

``Having known viññāṇa[kkhanda] to be feeble, fading away, and comfortless, with the destruction, fading away, cessation, giving up, and relinquishing of attraction and clinging regarding viññāṇa[kkhanda], of mental standpoints, adherences, and underlying tendencies regarding viññāṇa[kkhanda], I have understood that my mind is liberated.''

\pali{viññāṇaṃ kho ahaṃ āvuso abalaṃ virāgaṃ anassāsika'nti viditvā ye viññāṇe upāyūpādānā cetaso adhiṭṭhānābhinivesānusayā, tesaṃ khayā virāgā nirodhā cāgā paṭinissaggā vimuttaṃ me citta'nti pajānāmi) (M.3.31).}

``When that \pali{viññāṇaṃ} is unestablished, not coming to growth, nongenerative, it is liberated. By being liberated, it is steady; by being steady, it is content; by being content, he is not agitated. Being unagitated, he personally attains Nibbāna.''

\pali{tadappatiṭṭhitaṃ viññāṇaṃ avirūḷhaṃ anabhisaṅkhacca vimuttaṃ vimuttattā ṭhitaṃ, ṭhitattā santusitaṃ, santusitattā na paritassati, aparitassaṃ paccattaṃ yeva parinibbāyati (S.3.54).}

With viññāṇa unestablished, the clansman Godhika has attained final Nibbāna: \pali{appatiṭṭhitena ca bhikkhave viññāṇena godhiko kulaputto parinibbutoti (S.1.120-1).}
\end{quote}

When dependent-type \pali{viññāṇa} becomes not-dependent-type it does not make kamma (\pali{anabhisaṅkhacca}) (i.e. freedom-from-attachment does not make kamma) and the \pali{viññāṇa} is called liberated (\pali{vimuttaṃ}). This means the \pali{viññāṇa} is steady (\pali{ṭhitaṃ}), contented (\pali{santusitaṃ}), not agitated (\pali{na paritassati}) (i.e. freedom-from-attachment is steady, content, not agitated). Therefore the monk attains Nibbāna (S.3.53-4). This has been compared to a flame going out (`The liberation of my mind was like the dying of a flame': Thi116). This not-dependent-type \pali{viññāṇa} is, simply speaking, the end of the stream of `me' and `mine'. It is equivalent to the destruction of greed, hatred and delusion (S.4.251). It is the consciousness of the arahant.

\section*{Vi\~n\~n\=a\d{n}a\d{m} anidassana\d{m} ananta\d{m} sabbatopabha\d{m}}

\pali{`Viññāṇaṃ anidassanaṃ anantaṃ sabbatopabhaṃ'} means \pali{viññāṇa} without attributes, everlasting, completely without a source.

I translate \pali{anantaṃ} as `everlasting' because v.886 says that Truth is eternal (`There are not many and various eternal Truths in the world': \pali{saccāni niccāni}). Nibbāna should not be called `infinite' because Nibbāna is free of the sphere of infinite space (\pali{na ākāsānañcāyatanaṃ}) (Ud.80).

I take \pali{sabbatopabhaṃ} to be \pali{sabbato apabhaṃ} because Duroiselle says that after `o', vowels are usually elided -- para26. \pali{Apabhaṃ} is derived from \pali{apabhavati} or \pali{apahoti}. \pali{Pabhava} means `production, origin, source, cause'. The \pali{pabhava/pahoti} alternative spelling may explain the different readings \pali{apabhaṃ} (M.1.329) and \pali{apahaṃ} (D.1.223).

Sometimes the word \pali{sabbatopabhaṃ} has been taken to mean \pali{sabbato pabhaṃ} -- `shining everywhere'. But it is unlikely that this \pali{viññānam} would shine everywhere because luminescence is not an attribute of Nibbāna. Nibbāna has been called the Sublime (\pali{paṇītaṃ}), the Auspicious (\pali{sivaṃ}), the Wonderful (\pali{acchariyaṃ}), the Amazing (\pali{abbhutaṃ}) but not the Luminescent (S.4.360-373). In fact Nibbāna is called `neither dark nor bright' (D.3.251). It is said of Nibbāna: `There gleam no stars, no sun sheds light, there shines no moon, yet there no darkness reigns' (Ud.p9).

\pali{Viññāṇaṃ anidassanaṃ} (Nibbāna) is not reached by

\begin{itemize}
\item the solidity of solids -- \pali{paṭhaviyā paṭhavittena ananubhūtaṃ}
\item the fluidity of fluids -- \pali{āpassa āpattena ananubhūtaṃ}
\item the hotness of heat -- \pali{tejassa tejattena ananubhūtaṃ}
\item the gaseousness of gas -- \pali{vāyassa vāyattena ananubhūtaṃ}
\item the beingness of beings -- \pali{bhūtānaṃ bhūtattena ananubhūtaṃ}
\item the godness of gods -- \pali{devānaṃ devattena ananubhūtaṃ}
\item the allness of all -- \pali{sabbassa sabbattena ananubhūtaṃ} (M.1.329).
\end{itemize}

Because \pali{viññāṇaṃ anidassanaṃ} is not even reached by the four great elements, the Buddha said it would be wrong to ask if that is where `the four great elements cease without remainder' (\pali{aparisesā nirujjhanti}). Instead he said one should ask where they have no footing (\pali{na gādhati}). And because this \pali{viññāṇa} is without attributes, `long and short, coarse and fine, fair and foul, Name and Form are all completely blocked' (\pali{ettha dīghañca rassañca, aṇuṃ thūlaṃ subhāsubhaṃ; ettha nāmañca rūpañca asesaṃ uparujjhati}) (D.1.223).

So where `the four great elements have no footing' is where Name and Form are completely blocked. `Completely blocked' is a synonym for `have no footing'. Unfortunately, \pali{asesaṃ uparujjhati} is often translated as `wholly destroyed' -- which makes nonsense of the Buddha's statement. Because he then would be reported as saying that where the four great elements have no footing, in that place (\pali{ettha}) Form (\pali{rūpa}) is wholly destroyed. And he would thus apparently answer the very question he said should not be asked.
