
The Buddha saw his own teaching as endowed with all good qualities (\pali{sabbākārasampannaṃ}) and said that if anyone thought it could be improved by adding or subtracting from it, it would because of not truly seeing it (D.3.127). He said the schools of other sects were devoid of true ascetics (\pali{suññā parappavādā samaṇehi aññe}) (D.2.151). He said a monk should train himself in solitude because it is the supreme training (\pali{etadariyānamuttamaṃ}) (Octads, v.822). Regarding people who speculate about the future and the past, the Buddha did not consider their theories equal to his own, still less superior. He said `I am their superior in regard to the higher exposition' (\pali{atha kho ahameva tattha bhiyyo yadidaṃ adhipaññatti}) (D.3.139). He described the Sa\.ngha of his disciples as the unsurpassed field of merit for the world (\pali{bhagavato sāvakasaṅgho \ldots{} anuttaraṃ puññakkhettaṃ lokassāti}) (M.1.37). Venerable Sāriputta agreed; he said that the Buddha's way of teaching Dhamma in regards to wholesome states is unsurpassed (\pali{etadānuttariyaṃ yathā bhagavā dhammaṃ deseti kusalesu dhammesu}) (D.3.102).

Of course these are indeed views; and, as such, they were not grasped by the Buddha. This is a common theme of the Octads, for instance in the Māgandiya Sutta: `[The wise] do not say that purity is intrinsic to views, learning, knowledge, or precepts and practices; nor intrinsic to a lack of views, learning, knowledge, precepts and practices. But by detaching from these, not grasping them, at peace, untethered, one no longer hungers for existence' (v.839).

Once, the Buddha gave three reflections, which he called views:

\begin{itemize}
\item This has come to be (\pali{bhūtamidanti})
\item Its origin occurs with that as nutriment (\pali{tadāhārasam\-bha\-vanti})
\item With the cessation of that nutriment, what has come to be is subject to cessation (\pali{tadāhāranirodhā yaṃ bhūtaṃ taṃ nirodhadhammanti})
\end{itemize}

Then he told the monks that, as purified and bright as these views (\pali{diṭṭhiṃ}) are, if the monks adhered to them, cherished them, treated them as a possession, they would have failed to understand that Dhamma is like a raft, for the purpose of crossing over, not for the purpose of grasping (M.1.260).

Similarly, the Brahmajāla Sutta says that although the Buddha knew (\pali{pajānāti}) what the result of grasping views (\pali{diṭṭhiṭṭhānā evaṃgahitā evaṃparāmaṭṭhā}) is, he was not attached to that knowledge (\pali{pajānanaṃ na parāmasati}) and therefore he experienced for himself perfect peace (\pali{paccattaññeva nibbuti viditā}) (D.1.16-17).

So even the view or knowledge that `Buddhism is best' should not become an object of grasping or of pride. If one is attached to it, one will argue over religious teachings (v.787). If one thinks of oneself as better than others because of one's views, one will contend with others because of it (v.842).

But if one abandons one's fixed opinions, one creates no more trouble in the world (v.894). One remains equanimous, not grasping what other people grasp (v.912). One becomes someone who has found peace within (v.919). One could rightfully tell Pasūra (vv.832-834):

\clearpage

\begin{verse}
\verseref{832} They who argue,\\
Grasping a view,\\
Asserting that `This is very Truth',\\
You can talk to those people.\\
But \emph{here}\\
There is no opponent for you to battle with\newline when a dispute has arisen.

\verseref{833} Amongst those who have abandoned confrontation,\\
Who do not pit one view against another,\\
Amongst those who have not grasped any view\newline as the highest Goal,\\
Who would you gain as opponent, Pasūra?

\verseref{834} So here you come,\\
Speculating,\\
Mulling over various theories in your mind.\\
But you are paired off with a purified man.\\
With him you will not be able to proceed.
\end{verse}
