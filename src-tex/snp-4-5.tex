
\begin{verse}

\verseref{796} If a person maintains that of views,\newline his view is the highest Goal,\pagenote{A person who maintains that of views, his view is the highest Goal can be compared with the sage who does `not proclaim of any teaching ``This itself is final purification''\thinspace' (v.794).}\pagenote{The highest Goal: \pali{paramaṃ}, see: Translation Notes, page \pageref{transl-highest-goal}. }\\
Holding it as supreme in the world,\\
And says that all other views are contemptible,\\
Then he has not gone beyond disputes.

\pali{paramanti diṭṭhīsu paribbasāno,\\
yaduttari kurute jantu loke.\\
hīnāti aññe tato sabbamāha,\\
tasmā vivādāni avītivatto.}

\verseref{797} When a person sees an advantage for himself\\
In what is seen, heard, or cognised,\\
Or in precepts and practices,\\
He grasps such things,\\
Regarding everything else as contemptible.

\pali{yadattanī passati ānisaṃsaṃ,\\
diṭṭhe sute sīlavate mute vā.\\
tadeva so tattha samuggahāya,\\
nihīnato passati sabbamaññaṃ.}

\verseref{798} The good call that thing a bond,\\
If, tethered to it,\newline one regards other people as inferior.\\
Therefore a monk should not be tethered\newline to what is seen, heard, or cognised,\\
Nor to precepts and practices.

\pali{taṃ vāpi ganthaṃ kusalā vadanti,\\
yaṃ nissito passati hīnamaññaṃ.\\
tasmā hi diṭṭhaṃ va sutaṃ mutaṃ vā,\\
sīlabbataṃ bhikkhu na nissayeyya.}

\verseref{799} He should not concoct fixed views about others\pagenote{concoct fixed views about others (i.e. other people): PED's alternative translation of \pali{loka}.}\\
Based upon his knowledge of either them\\
Or of their precepts and practices.\\
He should neither suggest that he is equal,\\
Nor suppose that he is either inferior or superior.

\pali{diṭṭhimpi lokasmiṃ na kappayeyya,\\
ñāṇena vā sīlavatena vāpi.\\
samoti attānamanūpaneyya,\\
hīno na maññetha visesi vāpi.}

\verseref{800} Forsaking whatever he was clinging to --\\
And taking possession of nothing further --\\
He should not be tethered even to knowledge.\\
Amongst those in dispute\newline he should not take sides.\\
He should not revert\newline to fixed views whatsoever.\pagenote{The `should not': the whole verse is apparently meant in the optative case.}

\pali{attaṃ pahāya anupādiyāno,\\
ñāṇepi so nissayaṃ no karoti.\\
sa ve viyattesu na vaggasārī,\\
diṭṭhimpi so na pacceti kiñci.}

\clearpage

\verseref{801} One with no aspiration for any form of existence\\
Either in this world or the world beyond,\\
Has no attachment to dogmatic religious teachings.

\pali{yassūbhayante paṇidhīdha natthi,\\
bhavābhavāya idha vā huraṃ vā.\\
nivesanā tassa na santi keci,\\
dhammesu niccheyya samuggahītaṃ.}

\verseref{802} Whoever does not concoct\newline the slightest fictitious perception\pagenote{fictitious perception: \pali{saññā}, see Translation Notes, page \pageref{transl-fictitious-perceptions}.}\\
Regarding what is seen, heard or cognised,\\
This Brahman\newline who has grasped no view about anything,\pagenote{The sage has grasped no view about anything: namely, views about Truth and purity (v.824), dogmatic religious teachings\linebreak (vv.785, 837, 910), existence (v.786), or about the world (v.799). Grasping a view means thinking it is the highest Goal (v.833); or that other people's views are contemptible (v.797).}\\
How could anyone have any doubts about him?

\pali{tassīdha diṭṭhe va sute mute vā,\\
pakappitā natthi aṇūpi saññā.\\
taṃ brāhmaṇaṃ diṭṭhimanādiyānaṃ,\\
kenīdha lokasmiṃ vikappayeyya.}

\verseref{803} He does not concoct religious teachings,\\
Nor does he blindly follow them.\pagenote{concoct and blindly follow: I adopt the sense of v.784, where the terms refer to religious teachings.}\\
He does not hold on\newline even to the Buddha's teachings.\pagenote{The Buddha's teachings: \pali{dhammā}. In the Octads, the Buddha's teachings are sometimes called \pali{sāsanaṃ} (vv.814, 815, 933, 944).}\\
He is a Brahman,\\
Not to be inferred by precepts and practices.\\
Gone to the further shore,\\
One of such quality does not return.

\pali{na kappayanti na purekkharonti,\\
dhammāpi tesaṃ na paṭicchitāse.\\
na brāhmaṇo sīlavatena neyyo,\\
pāraṅgato na pacceti tādīti.}

\end{verse}
