
The Buddha described three modes of thought (\pali{tisso vidhā}):

\begin{itemize}
\item `I am superior'-mode \pali{seyyohamasmī'ti vidhā}
\item `I am equal'-mode \pali{sadisohamasmī'ti vidhā}
\item `I am inferior'-mode \pali{hīnohamasmī'ti vidhā}
\end{itemize}

He said the noble eightfold path is to be developed for the realisation of these three modes, for the full understanding of them, for their destruction and abandonment (S.5.56).

He said, if anyone regards himself as superior, equal or inferior (\pali{seyyohamasmīti vā samanupassanti, sadisohamasmīti vā samanupassanti, hīnohamasmīti vā samanupassanti}) on the basis of the impermanent, sorrowful, changing five khandhas, that it is due to not seeing things as they really are (\pali{yathābhūtassa adassanā}). Not regarding oneself thus (\pali{na samanupassanti}) is due to seeing things as they really are (\pali{yathābhutassa dassanā}) (S.3.48).

The Octads says that the Buddha did not `suppose that he [was] either inferior or superior' (v.799). If he supposed himself to be equal, superior or inferior, he `would contend with others because of it. But for one who is untroubled by these three modes of thought there is nobody equal, superior or inferior' (vv.842; 918). Nor did the Buddha regard other religious teachers as inferior, because `the good call that thing a bond, if, tethered to it, one regards other people as inferior. Therefore a monk should not be tethered to what is seen, heard, or cognised, nor to precepts and practices' (v.797).

The arahant Venerable Khema agreed with this. He said that an arahant would not think `Someone is better than me; someone is equal; someone is worse' (\pali{tassa na evaṃ hoti atthi me seyyoti vā atthi me sadisoti vā atthi me hīnoti vā'ti}). Similarly, the arahant Venerable Sumana said that an arahant would not think `No one is better than me; no one is equal; no one is worse' (\pali{tassa na evaṃ hoti `natthi me seyyoti vā, natthi me sadisoti vā, natthi me hīnoti vā'ti}) (A.3.359).

Nonetheless, the Buddha called himself `unique, without peer, without counterpart; incomparable; unequalled; matchless; unrivalled; best of men (\pali{adutiyo asahāyo, appaṭimo, appaṭisamo, appaṭibhāgo, appaṭipuggalo, asamo asamasamo, dipadānaṃ aggo}) (A.1.22) and said he was the most excellent of speakers (\pali{buddho pavadataṃ varoti}) (S.1.42). He said: `I do not see (\pali{na kho panāhaṃ passāmi}) any other recluse or brahman (\pali{aññaṃ samaṇaṃ vā brāhmaṇaṃ}) more perfect in virtue than myself, nor in concentration, wisdom, release, whom I could dwell reverencing, obeying and serving' (\pali{sīlasampannataraṃ yamahaṃ \ldots{} samādhi sampannataraṃ \ldots{} paññāsampannataraṃ \ldots{} vimuttisampannataraṃ}) (A.2.20; D.1.174). He said: `Among all the teachers now existing in the world, I see none who has attained to such a position of fame and following as I have' (\pali{yāvatā kho cunda etarahi satthāro loke uppannā, nāhaṃ cunda aññaṃ ekasatthārampi samanussami evaṃ lābhaggayasaggappattaṃ yatharivāhaṃ}) (D.3.126). He told Brahma Baka that `in regard to special knowledge, I am not equal to you, nor lower; rather, I know more than you (\pali{neva te samasamo abhiññāya, kuto nīceyyaṃ, atha kho ahameva tayā bhiyyo}) (M.1.329).

The Buddha also compared others. He called Venerable Sāriputta `chief (aggaṃ) amongst those of great wisdom' and Venerable Mahāmoggallāna `chief among those of supernormal powers' (A.1.23). He said, `I know not of any other person who so perfectly keeps rolling the wheel of Dhamma set rolling by the Tathāgata as does Sāriputta' (A.1.23). Conversely, he said `Monks, I know not of any other single person fraught with such loss to many folk \ldots{} as Makkhali, that foolish man' (\pali{makkhalī moghapuriso}) (A.1.33).

Venerable Sāriputta likewise favoured the Buddha. He said that it was clear to him that there never has been, never will be, and is not now, another ascetic or brahman who is better or more enlightened than the Buddha (\pali{bhiyyobhiññataro yadidaṃ sambodhiyanti}) (D.3.103).

So we must understand Venerables Khema's and Sumana's statements to mean that although arahants compare themselves, it is not in personal terms. Even though they might talk in personal terms, it would not be from ignorance. No Self-view would be involved. The Buddha's claims were not from arrogance or conceit. Therefore he would never ignorantly exalt himself and disparage others (\pali{na attānukkaṃseti na paraṃ vambheti}). He would never say `I live knowing and seeing, but these other people live unknowing and unseeing' (\pali{ahamasmi jānaṃ passaṃ viharāmi, ime panaññe bhikkhū ajānaṃ apassaṃ viharantīti}) (M.1.195). Nor would he think `I am well known and famous; but these other people are unknown and powerless' (\pali{ahaṃ khomhi ñāto yasassasī. Ime panaññe bhikkhū appaññātā appesakkhā'ti}) (M.3.38).
