
Arahants do indeed have differing views, but they do not fall into conflict about it. For instance, in an attempt to put his own enlightenment into perspective, the Buddha claimed that in the past and the future, too, there have been/will be other Buddhas equal in enlightenment to himself (D.3.114). But Venerable S\=ariputta said that in his view, no one ever has been, or will be, more enlightened than the Buddha. The Buddha did not argue. He simply asked Venerable S\=ariputta whether he had comprehensive knowledge of all Buddhas, past, present and future. Had he not spoken boldly with a bull's voice? Roared the lions' roar of certainty? Venerable S\=ariputta admitted that although he did not have such knowledge, in his opinion, whatever it is possible for someone to achieve through effort, that the Sublime One had achieved (D.3.113).

Arahants can sometimes be criticised by other arahants for their views. Once, Venerable Kappina the Great wondered: ``Should I go to an Observance or not? Should I go to Sa\.nghakamma or not? In either case, I am purified with the highest purification''. The Buddha told him ``If you brahmans (arahants) do not honour the Observance, who will? You go along to the Observance and Sa\.nghakamma. Do not not go''. ``Yes, Lord'' Venerable Kappina replied (Vin.1.105).

On another occasion, when the Buddha dismissed the Sa\.ngha, intending never to teach again, Venerable S\=ariputta decided to follow him into retirement to practise \textit{jh\=ana}. Later the Buddha told him: ``Stop, S\=ariputta! Never let such a thought arise in you again!'' But Venerable Mah\=amoggall\=ana had decided to help lead the Sa\.ngha, together with Venerable S\=ariputta. The Buddha told him: ``Very good, Moggall\=ana. Either I could lead the order, or S\=ariputta and Moggall\=ana could do so'' (M.1.459).

Sometimes arahant monks offer competing answers to riddles -- for instance at A.3.401 and M.1.212. Having collected the answers, when the monks asked the Buddha ``Who of us spoke well?'' (\textit{kassa nu kho bhante subhasitanti}), the Buddha replied ``You have all spoken well, each in his own way'' (\textit{sabbesaṃ vo bhikkhave subhāsitaṃ pariyāyena}); then he added ``Hear also from me [how I would answer the riddle]''. And he gave his own solution. 

On another occasion three arahants were discussing the three types of monks called \textit{kāyasakkhi, diṭṭhappatto, and saddhāvimutto}. Each arahant had a different opinion on which of the three types of monk is most excellent and choice (\textit{abhikkantataro ca paṇītataro cā}) (A.1.118). The Buddha said the issue could not be decided without knowing the spiritual attainments of each type of monk.

So arahants do indeed have differences of opinion, but without falling into conflict.
