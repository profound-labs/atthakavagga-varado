
\begin{verse}

\verseref{772} Abiding in a cave,\\
Attached to existence,\pagenote{Verse 777 says attachment (\pali{satto}) is in relation to existence (\pali{bha\-vesu}), therefore I call \pali{satto} `attached to existence'.}\\
Covered in defilement and immersed in delusion,\\
A man is truly far from solitude.\pagenote{See The Octads in a Nutshell: Abiding in solitude, page \pageref{abiding-in-solitude}.}\\
Forsaking the pleasures of the world is truly difficult.

\pali{satto guhāyaṃ bahunābhichanno,\\
tiṭṭhaṃ naro mohanasmiṃ pagāḷho.\\
dūre vivekā hi tathāvidho so,\\
kāmā hi loke na hi suppahāyā.}

\verseref{773}\verseref{774} Those fettered by desire,\\
Bound to the pleasures of existence,\\
Are not easily liberated.\\
Indeed, there is no liberation\\
Except in regard to such ties.

Wishing for the future and the past,\\
Hungering for present and former pleasures:\\
Those who are greedy for sensual enjoyment,\\
Hunting for it,\\
Stupified by it,\\
Become selfish about it,\\
Have entered upon a difficult path.\\
When drawn into difficulty, they lament:\\
`What will become of us in the hereafter?'

\pali{icchānidānā bhavasātabaddhā,\\
te duppamuñcā na hi aññamokkhā.\\
pacchā pure vāpi apekkhamānā,\\
imeva kāme purimeva jappaṃ.\\
kāmesu giddhā pasutā pamūḷhā,\\
avadāniyā te visame niviṭṭhā.\\
dukkhūpanītā paridevayanti,\\
kiṃsū bhavissāma ito cutāse.}

\verseref{775} So, people should indeed\newline train themselves in this world.\\
Whatever one knows to be wrong,\\
Do not for its sake engage in misconduct.\\
For the wise say that life is short.

\pali{tasmā hi sikkhetha idheva jantu,\\
yaṃ kiñci jaññā visamanti loke.\\
na tassa hetū visamaṃ careyya,\\
appañhidaṃ jīvitamāhu dhīrā.}

\verseref{776} I see people in this world writhing,\\
Oppressed by clinging to existence,\\
Wretched characters wailing in the face of death\\
With their clinging\newline to various forms of existence unallayed.

\pali{passāmi loke pariphandamānaṃ,\\
pajaṃ imaṃ taṇhagataṃ bhavesu.\\
hīnā narā maccumukhe lapanti,\\
avītataṇhāse bhavābhavesu.}

\verseref{777} Look at them,\\
Writhing amidst their beloved possessions\\
Like fish in a dwindling stream.\\
Having seen this,\\
You should live without possessiveness\\
And not get attached to existence.

\pali{mamāyite passatha phandamāne,\\
maccheva appodake khīṇasote.\\
etampi disvā amamo careyya,\\
bhavesu āsattimakubbamāno.}

\verseref{778} You should subdue longing\newline for all that is past or to come.\\
Having understood sense contact,\\
Being free of greed,\\
A wise person does nothing for which\newline he would blame himself.\\
For whatever he sees or hears\\
He is not stained by possessiveness.\pagenote{Verse 779 says the `stain' is that of possessiveness, so I call it that here.}

\pali{ubhosu antesu vineyya chandaṃ,\\
phassaṃ pariññāya anānugiddho.\\
yadattagarahī tadakubbamāno,\\
na lippatī diṭṭhasutesu dhīro.}

\verseref{779} Having understood fictitious perceptions,\pagenote{fictitious perceptions: \pali{saññaṃ}, see Translation Notes, page \pageref{transl-fictitious-perceptions}.}\\
Not stained by possessiveness,\\
The sage crosses the flood of sorrow.\pagenote{In verse 771, I called \pali{oghaṃ} `the flood of sorrow'; so here, too.}\\
With the arrow of suffering removed.\pagenote{Verse 767 says the `arrow' is the suffering experienced when pleasure diminishes in someone who craves for it. Therefore I call \pali{sallo} `the arrow of suffering'.}\\
Living diligently,\\
He longs for neither this world or the next.

\pali{saññaṃ pariññā vitareyya oghaṃ,\\
pariggahesu muni nopalitto.\\
abbūḷhasallo caramappamatto,\\
nāsīsatī lokamimaṃ parañcāti.}

\end{verse}
