
The Buddha said a sage does not blame himself (\pali{anattagarahī}) (v.913). This is probably because `a wise person does nothing for which he would blame himself' (\pali{yadattagarahī tadakubbamāno}) (v.778). The term \pali{anattagarahī} is therefore related to \pali{chinnakukkuccaṃ} (freedom from remorse), which is said to be a quality of an arahant at M.1.108. Remorse is what monks feel when they break their precepts (S.3.120). It is linked to both \pali{vippaṭisāro} (regret) and \pali{attā sīlato upavadatī'ti} (reproaching oneself in regard to virtue) and is allayed by the training in virtue.

Sages who do something blameworthy would likely blame themselves, because even the Buddha would blame them. For instance, when the arahant Venerable Piṇḍolabhāradvāja publically exhibited his supernormal powers, the Buddha blamed him (\pali{vigarahī}) (Vin.2.112). It is part of the monks' life to admonish others and accept admonishment in return (\pali{vattā ca assasi vacanakkhamo cā}) (S.2.282). And monks are indeed supposed to ask themselves whether they or their companions would blame (\pali{upavadati}) them for their conduct (A.5.88). And when being reproved, the Buddha told Venerable Sāriputta that a monk should welcome it (v.973).
