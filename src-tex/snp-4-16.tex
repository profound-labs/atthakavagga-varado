
\begin{verse}

\vspace*{-0.6\baselineskip}
(Venerable S\=ariputta)
\vspace*{-0.4\baselineskip}

\verseref{955} Never before have I seen or heard\\
Of a teacher coming from the host of Tusita heaven,\\
One having such lovely speech.

\pali{na me diṭṭho ito pubbe,\\
na suto uda kassaci.\\
evaṃ vagguvado satthā,\\
tusitā gaṇimāgato.}

\verseref{956} For the sake of the world with its gods\\
The Seer appears thus.\\
Having dispelled all Darkness,\\
He alone has attained delight.

\pali{sadevakassa lokassa,\\
yathā dissati cakkhumā.\\
sabbaṃ tamaṃ vinodetvā,\\
ekova ratimajjhagā.}

\verseref{957} To that Buddha,\\
Unentangled,\\
Of such good qualities,\\
Sincere,\\
Arrived with his following,\\
I come with a question\newline on behalf of the many here who are fettered.

\pali{taṃ buddhaṃ asitaṃ tādiṃ,\\
akuhaṃ gaṇimāgataṃ.\\
bahūnamidha baddhānaṃ,\\
atthi pañhena āgamaṃ.}

\verseref{958}\verseref{959} For a monk repelled by the world,\\
Resorting to lonely sitting places --\\
The foot of a tree, a cemetery, a mountain cave --\\
Or to various sleeping places:\\
How many fearful things are there\newline at which he need not tremble,\\
There in his quiet abode?

\pali{bhikkhuno vijigucchato,\\
bhajato rittamāsanaṃ.\\
rukkhamūlaṃ susānaṃ vā,\\
pabbatānaṃ guhāsu vā.}

\pali{uccāvacesu sayanesu,\\
kīvanto tattha bheravā.\\
yehi bhikkhu na vedheyya,\\
nigghose sayanāsane.}

\verseref{960} For the monk\newline going where he never before has gone,\\
How many are the difficulties\newline that he should bear,\\
There, in his secluded abode?

\pali{katī parissayā loke,\\
gacchato agataṃ disaṃ.\\
ye bhikkhu abhisambhave,\\
pantamhi sayanāsane.}

\verseref{961} What should be his manner of speech?\\
What should be the field of his conduct?\\
What should be that energetic monk's\newline precepts and practices?

\pali{kyāssa byappathayo assu,\\
kyāssassu idha gocarā.\\
kāni sīlabbatānāssu,\\
pahitattassa bhikkhuno.}

\verseref{962} For one composed, prudent and attentive,\\
Undertaking what training\newline could he remove his inner dross\\
Like a silversmith purifying molten silver?

\pali{kaṃ so sikkhaṃ samādāya,\\
ekodi nipako sato.\\
kammāro rajatasseva,\\
niddhame malamattano.}

(The Buddha)

\verseref{963} As one who knows,\\
I will explain to you what \emph{comfort} is\newline for someone repelled by the world,\\
For someone resorting to lonely lodgings,\\
Desiring awakening in accordance with Truth.

\pali{vijigucchamānassa yadidaṃ phāsu,\\
rittāsanaṃ sayanaṃ sevato ce.\\
sambodhikāmassa yathānudhammaṃ,\\
taṃ te pavakkhāmi yathā pajānaṃ.}

\clearpage

\verseref{964} A resolute monk,\\
One who is attentive,\\
Living a circumscribed lifestyle,\\
Need not tremble at five fears:\\
Horseflies, mosquitoes, snakes,\\
And interactions with humans and animals.

\pali{pañcannaṃ dhīro bhayānaṃ na bhāye,\\
bhikkhu sato sapariyantacārī.\\
ḍaṃsādhipātānaṃ sarīsapānaṃ,\\
manussaphassānaṃ catuppadānaṃ.}

\verseref{965} He need not fear followers\newline of other religious teachings --\\
Even on seeing their manifold threat.\\
He should bear other difficulties too,\newline as he seeks what is wholesome.

\pali{paradhammikānampi na santaseyya,\\
disvāpi tesaṃ bahubheravāni.\\
athāparāni abhisambhaveyya,\\
parissayāni kusalānuesī.}

\verseref{966} Affected by illness or hunger,\\
By cold or suffocating heat,\\
He should bear it.\\
That homeless one,\\
Affected in many ways,\\
Should make an effort,\\
Resolutely applying himself.

\pali{ātaṅkaphassena khudāya phuṭṭho,\\
sītaṃ atuṇhaṃ adhivāsayeyya.\\
so tehi phuṭṭho bahudhā anoko,\\
vīriyaṃ parakkammadaḷhaṃ kareyya.}

\verseref{967} He should not steal.\\
He should not lie.\\
He should touch beings with good-will,\\
Both the timid and the mettlesome.\\
When he is conscious that his mind is disturbed\\
He should dispel it with the thought:\\
`It is part of Darkness'.

\pali{theyyaṃ na kāre na musā bhaṇeyya,\\
mettāya phasse tasathāvarāni.\\
yadāvilattaṃ manaso vijaññā,\\
kaṇhassa pakkhoti vinodayeyya.}

\verseref{968} He should not fall under the control\newline of anger or arrogance;\\
He should abide having uprooted them.\\
Then he should master what is loved and hated.

\pali{kodhātimānassa vasaṃ na gacche,\\
mūlampi tesaṃ palikhañña tiṭṭhe.\\
athappiyaṃ vā pana appiyaṃ vā,\\
addhā bhavanto abhisambhaveyya.}

\verseref{969} Esteeming wisdom,\\
Delighted by what is morally good,\\
He should conquer his difficulties.\\
He should overcome discontent\newline in his secluded resting place.\\
He should overcome four lamentations:

\pali{paññaṃ purakkhatvā kalyāṇapīti,\\
vikkhambhaye tāni parissayāni.\\
aratiṃ sahetha sayanamhi pante,\\
caturo sahetha paridevadhamme.}

\verseref{970} `What will I eat?'\\
`Where will I eat?'\\
`How uncomfortably I slept!'\\
`Where will I sleep tonight?'\\
The person in training,\\
Wandering homeless,\\
Should subdue such wailing thoughts.

\pali{kiṃsū asissāmi kuvaṃ vā asissaṃ,\\
dukkhaṃ vata settha kvajja sessaṃ.\\
ete vitakke paridevaneyye,\\
vinayetha sekho aniketacārī.}

\verseref{971} When offered food and clothing\newline at the appropriate time\\
He should know how much\newline is enough for contentment.\\
Self-controlled in this respect,\\
Acting carefully in the village,\\
Even when provoked,\newline he should not speak a harsh word.

\pali{annañca laddhā vasanañca kāle,\\
mattaṃ so jaññā idha tosanatthaṃ.\\
so tesu gutto yatacāri gāme,\\
rusitopi vācaṃ pharusaṃ na vajjā.}

\verseref{972} He should restrain his eyes.\\
He should not be footloose.\\
He should apply himself to jhāna.\\
He should be very wakeful.\\
He should practise equanimity and composure.\\
He should cut off the tendency to doubt and worry.

\pali{okkhittacakkhu na ca pādalolo,\\
jhānānuyutto bahujāgarass.\\
upekkhamārabbha samāhitatto,\\
takkāsayaṃ kukkucciyūpachinde.}

\verseref{973} When being reproved, remaining attentive,\newline he should welcome it.\\
He should destroy the hard heartedness he might\newline have towards his fellows in the holy life.\pagenote{hard heartedness: \pali{khilaṃ}, see Translation Notes, page \pageref{transl-hard-heartedness}.}\\
He should speak words\newline that are skilful and timely.\\
He should not think about things\newline which are matters of gossip.

\pali{cudito vacībhi satimābhinande,\\
sabrahmacārīsu khilaṃ pabhinde.\\
vācaṃ pamuñce kusalaṃ nātivelaṃ,\\
janavādadhammāya na cetayeyya.}

\verseref{974} Furthermore, there are five stains in man\pagenote{`in man': PED's alternative translation of \pali{loke}.}\\
For the removal of which\newline he should attentively train himself:\\
He should overcome attachment to forms,\\
Sounds, tastes, smells, and tactile sensations.

\pali{athāparaṃ pañca rajāni loke,\\
yesaṃ satīmā vinayāya sikkhe.\\
rūpesu saddesu atho rasesu,\\
gandhesu phassesu sahetha rāgaṃ.}

\clearpage

\verseref{975} Being attentive,\\
With a well-liberated mind,\\
A monk should remove\newline his longing for these things.

Examining the Buddha's teachings\newline at suitable times,\\
In suitable ways,\\
With a composed mind,\\
He should put an end to Darkness.

\pali{etesu dhammesu vineyya chandaṃ,\\
bhikkhu satimā suvimuttacitto.\\
kālena so sammā dhammaṃ parivīmaṃsamāno,\\
ekodibhūto vihane tamaṃ so'ti.}

\end{verse}
