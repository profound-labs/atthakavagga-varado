
\noindent\textit{(Note) `The Lesser Blind Alley': a blind alley is defined in the scriptures as a road where `they depart the same way they entered' (Vin.4.271). Arguments about Truth -- the subject of this and the next discourse -- are likely called blind alleys because they lead nowhere.}

\begin{verse}

(Questioner)

\verseref{878} Maintaining their own fixed views,\\
Contentious,\\
Different experts say:\\
`Whoever knows this knows Truth.\pagenote{Truth: \pali{dhammaṃ} is here synonymous with \pali{saccaṃ} in v.882.}\\
Whoever rejects it is not perfected'.

\pali{sakaṃsakaṃdiṭṭhiparibbasānā,\\
viggayha nānā kusalā vadanti.\\
yo evaṃ jānāti sa vedi dhammaṃ,\\
idaṃ paṭikkosamakevalī so.}

\verseref{879} Thus contentious, they squabble:\\
`My opponent is a fool. He is no expert'.\\
Given that they all claim to be experts,\\
Which of these statements is true?

\pali{evampi viggayha vivādayanti,\\
bālo paro akkusaloti cāhu.\\
sacco nu vādo katamo imesaṃ,\\
sabbeva hīme kusalā vadānā.}

(The Buddha)

\verseref{880} If rejecting an opponent's teachings\newline makes one a `fool',\\
One of inferior wisdom,\\
Then all of them are fools\newline of very inferior wisdom,\pagenote{`All of them are fools': Each person says that other people's teachings are contemptible (says v.905); therefore, each person is likely to be accused by others of having a contemptible teaching, and so of being a fool. The Buddha says they are of `very little wisdom' because they are accused in the same terms by which they accuse others.}\\
All those who maintain that their own views\newline are the highest Goal.\pagenote{The text reads `maintain their own views'. From v.796, I take this to mean `maintain their own views are the highest Goal'.}

\pali{parassa ce dhammamanānujānaṃ,\\
bālomako hoti nihīnapañño.\\
sabbeva bālā sunihīnapaññā,\\
sabbevime diṭṭhiparibbasānā.}

\verseref{881} But if each is intrinsically cleansed by their views,\pagenote{\pali{vīvadātā}: derived from \pali{odāta}, which PED says is an adjective and a past participle.}\\
Of perfected wisdom,\\
An expert,\\
Intelligent,\\
Then none of them are of inferior wisdom,\\
For all of them are accomplished in their own views.

\pali{sandiṭṭhiyā ceva na vīvadātā,\\
saṃsuddhapaññā kusalā mutīmā.\\
na tesaṃ koci parihīnapañño,\\
diṭṭhī hi tesampi tathā samattā.}

\verseref{882} I definitely do not say\\
`This (my view) is Truth'\pagenote{`This (my view) is Truth': one who has realised Truth has done so by forsaking everything (v.946). Therefore no view can be called Truth. So a sage would `not proclaim of any teaching ``This itself is final purification''\thinspace' (v.794). Though \pali{tathiya} is an adjective, it is apparently a synonym of the noun \pali{saccaṃ} in the next line.}\\
As fools say to one another.\\
They each make out their own views to be Truth\\
And therefore brand their opponents as `fools'.

\clearpage

\pali{na vāhametaṃ tathiyanti brūmi,\\
yamāhu bālā mithu aññamaññaṃ.\\
sakaṃsakaṃdiṭṭhimakaṃsu saccaṃ,\\
tasmā hi bāloti paraṃ dahanti.}

(Questioner)

\verseref{883} What some say is Actuality, Truth,\newline others say is Vanity, Falsehood.\pagenote{So-called experts, with sophistry, call their own views `Truth', and their opponents' views `Falsehood' (v.886).}\\
Thus contentious, they squabble.\\
Why don't ascetics say one and the same thing?

\pali{yamāhu saccaṃ tathiyanti eke,\\
tamāhu aññe tucchaṃ musāti.\\
evampi vigayha vivādayanti,\\
kasmā na ekaṃ samaṇā vadanti.}

(The Buddha)

\verseref{884} The Truth is single.\\
There is not another Truth\newline about which mankind should quarrel.\\
Ascetics proclaim their own various `Truths';\\
That's why they don't say one and the same thing.

\pali{ekañhi saccaṃ na dutīyamatthi,\\
yasmiṃ pajā no vivade pajānaṃ.\\
nānā te saccāni sayaṃ thunanti,\\
tasmā na ekaṃ samaṇā vadanti.}

(Questioner)

\verseref{885} But why do they proclaim differing Truths,\\
These argumentative so-called experts?\\
Have they come across many differing Truths\\
Or are they merely speculating?

\pali{kasmā nu saccāni vadanti nānā,\\
pavādiyāse kusalā vadānā.\\
saccāni sutāni bahūni nānā,\\
udāhu te takkamanussaranti.}

(The Buddha)

\verseref{886} Apart from the mere notion of it\\
There are not many and various\newline eternal Truths in the world.\\
But by resorting to sophistry,\\
The so-called experts say that, in respect to views,\\
There is a fixed duality:\pagenote{Fixed duality: In this verse the Buddha says Truth is eternal. The sophists apparently agree with this. Therefore they presumably regard the duality they proclaim (\pali{dvayadhammamāhu}) to be fixed.} Truth and Falsehood.

\pali{na heva saccāni bahūni nānā,\\
aññatra saññāya niccāni loke.\\
takkañca diṭṭhīsu pakappayitvā,\\
saccaṃ musāti dvayadhammamāhu.}

\verseref{887} Tethered to what is seen, heard, or cognised,\\
Or to precepts and practices\\
A person shows contempt for others.\\
Abiding by his fixed opinions,\\
And pleased with himself,\\
He says:\\
`My opponent’s a fool. He is no expert'.

\pali{diṭṭhe sute sīlavate mute vā,\\
ete ca nissāya vimānadassī.\\
vinicchaye ṭhatvā pahassamāno,\\
bālo paro akkusaloti cāha.}

\verseref{888} Upon whatever basis\newline he regards his opponent a fool\\
Is the same upon which\newline he regards himself an expert.\\
To the extent to which he rates himself an expert\\
He despises anyone else who makes the same claim.

\pali{yeneva bāloti paraṃ dahāti,\\
tenātumānaṃ kusaloti cāha.\\
sayamattanā so kusalo vadāno,\\
aññaṃ vimāneti tadeva pāva.}

\verseref{889} In his own overestimated view he is perfected.\\
Drunk with pride,\\
He supposes he is fully accomplished.\\
In his mind he consecrates himself.\\
His views, likewise, he regards as also perfect.

\pali{atisāradiṭṭhiyāva so samatto,\\
mānena matto paripuṇṇamānī.\\
sayameva sāmaṃ manasābhisitto,\\
diṭṭhī hi sā tassa tathā samattā.}

\verseref{890} If by the word of somebody else\newline one were inferior,\\
That `somebody else'\newline would be of inferior wisdom also.\pagenote{`That ``somebody else'' would be of inferior wisdom also': because `each person says that the others' teachings are contemptible' (v.905).}\\
But if, by one's own reckoning,\newline one were knowledgeable and wise\\
Then none among ascetics would be a fool.

\pali{parassa ce hi vacasā nihīno,\\
tumo sahā hoti nihīnapañño.\\
atha ce sayaṃ vedagū hoti dhīro,\\
na koci bālo samaṇesu atthi.}

\verseref{891} `Those who proclaim religious teachings different\newline from this have strayed from purification.\\
They are not perfected'.\\
Members of other sects each say this\\
Because they are each ardent\newline about their own views.

\pali{aññaṃ ito yābhivadanti dhammaṃ,\\
aparaddhā suddhimakevalī te.\\
evampi titthyā puthuso vadanti,\\
sandiṭṭhirāgena hi tebhirattā.}

\verseref{892} `Here alone is purification' they say,\\
And say that there is no purification\newline intrinsic to other religious teachings.\\
Thus are members of other sects established\newline at odds with each other,\\
And thus are they committed\newline to their own so-called paths.

\pali{idheva suddhi iti vādayanti,\\
nāññesu dhammesu visuddhimāhu.\\
evampi titthyā puthuso niviṭṭhā,\\
sakāyane tattha daḷhaṃ vadānā.}

\verseref{893} Although someone is committed\newline to his own so-called path,\\
What person could he take to be a fool\newline in regards to it?\\
If he said that another person was a fool\newline following impure teachings\\
He would simply invite trouble on himself.

\pali{sakāyane vāpi daḷhaṃ vadāno,\\
kamettha bāloti paraṃ daheyya.\\
sayameva so medhagamāvaheyya,\\
paraṃ vadaṃ bālamasuddhidhammaṃ.}

\verseref{894} Steadfast in his fixed opinions,\\
Measuring others by his own criteria,\\
He enters ever more disputes in the world.\\
But the person who has forsaken all fixed opinions\\
Creates no more trouble in the world.

\pali{vinicchaye ṭhatvā sayaṃ pamāya,\\
uddhaṃ sa lokasmiṃ vivādameti.\\
hitvāna sabbāni vinicchayāni,\\
na medhagaṃ kubbati jantu loke'ti.}

\end{verse}
