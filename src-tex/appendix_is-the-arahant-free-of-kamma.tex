
Verse 953 says that an arahant:

\begin{itemize}
\item has no accumulation of kamma (\pali{natthi kāci nisaṅkhiti}).
\item abstains from initiating new kamma (\pali{virato so viyārambhā}).
\end{itemize}

These statements need qualifying. To understand the kamma of arahants, it would be helpful to review the Buddha's teachings on kamma.

Kamma is of four types:

\begin{itemize}
\item black kamma with black results: acts from unskilful roots
\item white kamma with white results: acts from skilful roots
\item black-and-white kamma with black-and-white results: acts from mixed roots
\item neither-black-nor-white kamma with neither-black-nor-white results: kamma that leads to the destruction of kamma (\pali{kammakkhayāya saṃvattati}) (A.2.231; A.3.338-9).
\end{itemize}

Kamma is whatever the person's aim (cetanā) that produces various acts of body speech and mind (\pali{cetanāhaṃ bhikkhave kammaṃ vadāmi; cetayitvā kammaṃ karoti kāyena vācāya manasā}). Aims arise from skilful or unskilful roots: greed/non-greed, hatred/non-hatred, delusion/non-delusion. All acts, whether skilful or unskilful, are kammically potent as long as greed, hatred and delusion are undestroyed. These kammically potent acts ripen (\pali{vipaccati}) and in due course bear fruit that must be experienced (\pali{vipākaṃ paṭisaṃvedeti}) either in this life or some future life. Such acts are compared to undamaged seeds that inevitably produce fruit (A.1.135).

The fruit of kamma is called \pali{vipāka}; or \pali{kammānaṃ upacitānaṃ} (stored-up kamma) or sometimes simply kamma. This kamma is experienced in any of the realms of existence: hell realms, animal realms, ghost realms, human realms, heavenly realms (A.3.414). Acts from unskilful roots lead to lower rebirths; acts from skilful roots lead to higher realms. An arahant does not make kamma because, even though his acts arise from skilful roots, there is no residual greed, hatred and delusion; therefore his acts are inoperative.

The kamma that leads to the destruction of kamma is variously named as:

\begin{itemize}
\item the eightfold path (A.2.237)
\item the seven factors of awakening (A.2.237)
\item the aim to forsake all kamma, whether black or white (\pali{pahānāya yā cetanā}) (A.2.232)
\end{itemize}

Old kamma is not destroyed except by experiencing the result (\pali{appaṭisaṃveditvā}), either in this life, or some future life (A.5.299). So any monk wanting to destroy his kamma, can do it in only one way, as follows:

\begin{itemize}
\item For a monk who scrupulously observes the Pātimokkha precepts: he must make no new kamma (\pali{navañca kammaṃ na karoti}), and must destroy old kamma by repeated contact with it (\pali{purāṇañca kammaṃ phussa phussa byantīkaroti}).
\item For a monk who practices the jhānas: he must make no new kamma, and must destroy old kamma by repeated contact with it.
\item For a monk with blinding tendencies destroyed (\pali{āsavānaṃ khayā}): he must make no new kamma, and must destroy old kamma by repeated contact with it. (A.1.221).
\end{itemize}

This destruction of kamma by experiencing its results is illustrated by the monk who was sitting cross-legged not far from the Buddha, attentive and fully conscious, enduring feelings that were painful and sharp, the result of former unskilful kamma (\pali{purāṇakammavipākajaṃ dukkhaṃ}) (Ud.21). In the same way, skilful kamma must presumably also be destroyed by experiencing its result.

There is no disagreement in the suttas that an arahant abstains from initiating new kamma -- either positive, negative or neutral (\pali{puññābhisaṅkhāraṃ vā abhisaṅkhareyya apuññābhisaṅkhāraṃ vā abhisaṅkhareyya āneñjābhisaṅkhāraṃ vā abhisaṅkhareyyā'ti}) (S.2.83). This is because, after the final destruction of greed, hatred and delusion at arahantship, any kamma performed with non-greed, non-hatred and non-delusion is not destined for future arising (\pali{anuppādadhammaṃ}) (A.1.135). Such acts are compared to seeds that are destroyed, incapable of producing fruit (A.1.135). Because arahants have destroyed greed, hatred and delusion (S.5.8), they therefore cannot make kamma.

Many suttas agree that an arahant has no accumulated kamma. For instance, the arahant Venerable Ugga said that whatever kamma he had created, whether small or great, `all that is destroyed' (\pali{sabbametaṃ parikkhīṇaṃ}) (Th.80). This implies, say the suttas, that he had experienced its fruit (A.5.299). The suttas say that he must have experienced all the fruit, because `there is no end to dukkha, until the fruit has been experienced' (\pali{na tvevāhaṃ bhikkhave sañcetanikānaṃ kammānaṃ katānaṃ upacitānaṃ appaṭisaṃveditvā dukkhassantakiriyaṃ vadāmi}) (A.5.299). Although `the end of dukkha' usually implies arahantship (\pali{antamakāsi dukkhassa}: A.1.134), we will review this assumption below. But, nonetheless, this phrase suggests that the arahant has indeed no accumulated kamma.

However, all arahants have certain aspects of old kamma that have not been destroyed. For instance, the support one receives as a monk is related to the alms one has given in previous lives (A.3.33-4). And the Buddha even described one's own body and senses as old kamma (\pali{purāṇam kammaṃ}) something to be felt (\pali{vedayitaṃ}) (S.2.64; S.4.132). `Something to be felt' implies that it is purified in the same way as other kamma. This idea is illustrated in the Lakkhana Sutta, where the Buddha explained the kamma which led to his own fine body, good health and bountiful support. Presumably there are similar reasons for an arahant having an unattractive body and poor health. This would seem to be kamma -- either good or bad -- that is in the process of being destroyed by an arahant.

Secondly, some suttas suggest that the arahant is still paying off his old debts. For instance -- as noted above -- an arahant must apparently destroy old kamma by repeated contact with it (A.1.221); which suggests that an arahant still has kamma to destroy.

Several suttas describe the situation in a mixed way. They say that an arahant has destroyed kamma but is nonetheless still destroying it; for instance the monk who was sitting cross-legged not far from the Buddha, attentive and fully conscious, enduring feelings that were painful and sharp, the result of former kamma. The Buddha described him as one who had `forsaken all kamma' (\pali{sabbakammajahassa bhikkhuno}) but added that he was `shaking off the dust of former kamma' (\pali{dhunamānassa pure kataṃ rajaṃ}) (Ud.21). Secondly, when Venerable Angulimāla was physically attacked on almsround, the Buddha said `Bear it, Brahman! Bear it, Brahman! You are experiencing here and now the result of kamma because of which you might have been tortured in hell for many years' (\pali{adhivāsehi tvaṃ brāhmaṇa adhivāsehi tvaṃ brāhmaṇa; yassa kho tvaṃ brāhmaṇa kammassa vipākena bahūni vassāni \ldots{} niraye pacceyyāsi tassa tvaṃ brāhmaṇa kammassa vipākaṃ diṭṭheva dhamme paṭisaṃvedesī'ti}) (M.2.106). The word `Brahman' suggests that Angulimāla was already an arahant, but still having to bear his old kamma. But later, in solitude he exclaimed that all his kamma `has touched me now' (\pali{phuṭṭho kammavipākena}) and that now ate his almsfood `free of debt' (\pali{aṇaṇo bhuñjāmi bhojanaṃ}) (M.2.104-5).

The answer to this conundrum is suggested by the arahant Venerable Samitigutta who said that whatever kamma he had done in previous lives, `that must be experienced in this world' (\pali{idheva taṃ vedanīyaṃ}) because `there is no other basis'; which means that for him there was no other lifetime (\pali{vatthu aññaṃ na vijjatī'ti}) (Th.81); because at the moment of enlightenment, the arahant knows that this is his last birth, there will be no continuation of existence (\pali{ayamantimā jāti natthi dāni punabbhavo'ti}). Therefore he knows that all his kamma must be destroyed by him within his lifetime. To that extent one can say that there is no end to dukkha until all one's accumulated kamma has been experienced, if by `the end of dukkha' one does not mean the moment of arahantship, but the final passing away of the arahant, when the body and the senses -- all of it old kamma -- is finally discarded.

A similar knowledge is known to non-returners who apparently know that whatever evil deeds (\pali{pāpakammaṃ}) they did, the results will be experienced in their present life (\pali{sabbaṃ taṃ idha vedaniyaṃ}); it will not follow them on (\pali{na taṃ anugaṃ bhavissatī'ti}) (A.5.301). That leaves them the fruit of good deeds to be experienced in their next (celestial) existence.
