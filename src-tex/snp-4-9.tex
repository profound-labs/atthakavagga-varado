
\noindent\textit{\small Dialogue on the occasion of M\=aga\d{n}\d{d}iya's offer of his daughter to the Buddha.}

\begin{verse}

(The Buddha)

\verseref{835} Seeing even Taṇhā, Arati and Ragā,\\
(The three daughters of Māra),\\
Aroused in me no longing for sexual intercourse.\pagenote{The three daughters of Māra attempted to seduce the Buddha by elaborate means to bring him back under the control of their father. Māra later told them their attempt had been like battering a mountain with lotus stalks, or digging a mountain with their nails, or chewing iron with their teeth. He told them they had been swept away by the Buddha like wind blows away a cotton tuft (S.1.124).}\\
So for what reason would I want \emph{this} woman,\\
Filled as she is with urine and excrement?\\
I would not want to touch her -- even with my foot.

\pali{disvāna taṇhaṃ aratiṃ ragañca,\\
nāhosi chando api methunasmiṃ.\\
kimevidaṃ muttakarīsapuṇṇaṃ,\\
pādāpi naṃ samphusituṃ na icche.}

(M\=aga\d{n}\d{d}iya)

\verseref{836} If you do not want such a jewel,\\
A woman sought after by many kings,\\
Then what views, precepts, practices,\newline livelihood and rebirth do you proclaim?

\pali{etādisaṃ ce ratanaṃ na icchasi,\\
nāriṃ narindehi bahūhi patthitaṃ.\\
diṭṭhigataṃ sīlavataṃ nu jīvitaṃ,\\
bhavūpapattiñca vadesi kīdisaṃ.}

(The Buddha)

\verseref{837} In regards to dogmatic religious teachings,\\
Of none of them have I said `I proclaim this'.\\
But rather, in scrutinising views, without grasping,\\
While searching, I realised inner peace.

\pali{idaṃ vadāmīti na tassa hoti,\\
dhammesu niccheyya samuggahītaṃ.\\
passañca diṭṭhīsu anuggahāya,\\
ajjhattasantiṃ pacinaṃ adassaṃ.}

(M\=aga\d{n}\d{d}iya)

\verseref{838} Of opinions that have been concocted,\\
You indeed speak without grasping.\\
This inner peace, which you mentioned,\pagenote{\pali{yametamatthaṃ}: `which you mentioned', see: Translation Notes, page \pageref{transl-which-you-mentioned}.}\\
How is it explained by the wise?

\pali{vinicchayā yāni pakappitāni,\\
te ve munī brūsi anuggahāya.\\
ajjhattasantīti yametamatthaṃ,\\
kathaṃ nu dhīrehi paveditaṃ taṃ.}

\verseref{839} They do not say that purification\newline is intrinsic to views\pagenote{For a commentary to this translation, see: Translation Notes, page \pageref{transl-intrinsic}: Instrumental and ablative cases as `intrinsic'.}\\
Learning, knowledge, or precepts and practices;\\
Nor intrinsic to a lack of views,\\
Learning, knowledge, precepts and practices.\\
But by forsaking these,\pagenote{See: Should monks forsake precepts and practices? page \pageref{should-monks-forsake}.}\\
Not grasping them,\pagenote{Not grasping views, learning, knowledge, precepts and practices. A similar description of practice is found in the \pali{Udāna}, which says that one extreme is to think that religious training is the essence -- or that precepts and practices, or celibacy or service is the essence (\pali{sikkhāsārā sīlabbatajīvitabrahmacariyaupaṭṭhānasārā ayameko anto}). The other extreme is to think that there is no flaw in sensuality (\pali{natthi kāmesu doso'ti ayaṃ dutiyo anto}). Both these extremes `cause the cemeteries to grow'. For those who fully understand (\pali{abhiññāya}) the two extremes, and abandon them, they escape from \pali{sa\d{m}s\=ara} (Ud.71-2).}\\
At peace, untethered,\\
One no longer hungers for existence.

\pali{na diṭṭhiyā na sutiyā na ñāṇena,\\
sīlabbatenāpi na suddhimāha.\\
adiṭṭhiyā assutiyā añāṇā,\\
asīlatā abbatā nopi tena.\\
ete ca nissajja anuggahāya,\\
santo anissāya bhavaṃ na jappe.}

(M\=aga\d{n}\d{d}iya)

\verseref{840} If they do not say that purification\newline is intrinsic to views,\\
Learning, knowledge, precepts and practices;\\
Nor intrinsic to a lack of views,\\
Learning, knowledge, precepts and practices,\\
It seems to me that this teaching is indeed foolish.\\
For some attain purity by means of views.\pagenote{The Buddha defines the goal as being untethered to the path. This confuses M\=aga\d{n}\d{d}iya to the extent that, whereas the Buddha gave him the \emph{definition} of purity, his complaint concerns the \emph{attainment} of purity. And whereas the Buddha describes the abstract quality of purity; M\=aga\d{n}\d{d}iya asks about purity as a personal attainment.}

\pali{no ce kira diṭṭhiyā na sutiyā na ñāṇena,\\
sīlabbatenāpi na suddhimāha.\\
adiṭṭhiyā assutiyā añāṇā,\\
asīlatā abbatā nopi tena.\\
maññāmahaṃ momuhameva dhammaṃ,\\
diṭṭhiyā eke paccenti suddhiṃ.}

(The Buddha)

\verseref{841} Enquiring,\\
Tethered to a fixed view,\\
Bewildered by what you are attached to,\\
You cannot apprehend the simplest notion.\\
Therefore you think that this teaching is foolish.\pagenote{Thus ends the conversation with M\=aga\d{n}\d{d}iya.}

\pali{diṭṭhañca nissāya anupucchamāno,\\
samuggahītesu pamohamāgā.\\
ito ca nāddakkhi aṇumpi saññaṃ,\\
tasmā tuvaṃ momuhato dahāsi.}

\verseref{842} Whoever supposes himself\newline to be equal, superior or inferior\\
Would contend with others because of it.\\
But for one who is untroubled\newline by these three modes of thought\\
There is nobody equal, superior or inferior.

\pali{samo visesī uda vā nihīno,\\
yo maññatī so vivadetha tena.\\
tīsu vidhāsu avikampamāno,\\
samo visesīti na tassa hoti.}

\verseref{843} Of what view would a Brahman say\pagenote{Of what view. Here \pali{kiṃ} seems to mean `what view?'. Also in v.832 `This is very Truth' refers to views.}\\
`It is Truth' or `It is Falsehood'?\pagenote{So-called experts call their religious teachings `Truth' (vv.824, 910). But Truth involves forsaking everything (v.946). So a sage would `not proclaim of any teaching ``This itself is final purification''\thinspace' (v.794). `Falsehood' is a term devised by sophists to label other people's `Truths' (v.886).}\\
With whom would he contend?\\
The Brahman who neither supposes\newline he is `equal' nor `unequal',\pagenote{\pali{Yasmiṃ samaṃ visamaṃ vāpi natthi}. Norman says `In whom there is no (idea of being) equal or unequal either'. A similar expression in v.799 includes the word \pali{maññetha}: one must not `suppose' one is either inferior or superior. Therefore I phrase it: The Brahman who neither supposes he is `equal' nor `unequal'.}\\
With whom would he join in dispute?

\pali{saccanti so brāhmaṇo kiṃ vadeyya,\\
musāti vā so vivadetha kena.\\
yasmiṃ samaṃ visamaṃ vāpi natthi,\\
sa kena vādaṃ paṭisaṃyujeyya.}

\verseref{844} Having forsaken the home-life,\\
Not living in company,\\
The sage does not create\newline intimate relationships in the village.\\
Rid of sensual passion,\\
Free of yearning,\\
He would not speak to people contentiously.\pagenote{See: Venerable Mah\=akacc\=ana's explanation of verse 844, page \pageref{mahakaccana-v844}. This verse, though not obviously part of the conversation with Māgandiya, even within the Buddha's lifetime was nonetheless considered part of the Questions of Māgandiya (\pali{māgandiya pañhe}) (S.3.9).}

\pali{okaṃ pahāya aniketasārī,\\
gāme akubbaṃ muni santhavāni.\\
kāmehi ritto apurekkharāno,\\
kathaṃ na viggayha janena kayirā.}

\verseref{845} Those things that a great being\newline should live aloof from,\\
He should neither acquire them\newline nor talk about them.\\
As the prickly lotus\newline is unsullied by water and mud,\\
So the sage,\\
Professing peace, and free of greed,\\
Is not stained by sensual pleasure\newline and the things of the world.

\pali{yehi vivitto vicareyya loke,\\
na tāni uggayha vadeyya nāgo.\\
jalambujaṃ kaṇḍakaṃ vārijaṃ yathā,\\
jalena paṅkena canūpalittaṃ.\\
evaṃ munī santivādo agiddho,\\
kāme ca loke ca anūpalitto.}

\verseref{846} One who has realised Truth\\
Feels no pride regarding his views or thoughts\\
Because he does not regard them\newline as part of himself;\pagenote{part of himself: \pali{tammayo}. This word also occurs at M.1.319 and A.3.444; and \pali{tammayatā} occurs at M.3.42 and M.3.220. Seeing that everything in the world is not part of oneself (\pali{sabbaloke ca atammayo bhavissāmi}) is one of the advantages of developing the notion of not-Self (\pali{anattasaññaṃ}). Two further advantages are: 
It restrains the notion of `me' (\pali{ahaṅkārā ca me uparujjhissanti}).
It restrains the notion of `mine' (\pali{mamaṅkārā ca me uparujjhissanti}) (A.3.444).}\\
Such a person is not to be inferred\\
By his precepts or practices,\pagenote{Not to be inferred by his precepts or practices (\pali{na kammunā \ldots{} neyyo}). I take \pali{kammunā} as equivalent to \pali{sīlavatena} of v.803 (\pali{na brāhmaṇo sīlavatena neyyo}).}\\
Nor by his religious knowledge.\\
He is a person not drawn into clinging.

\pali{na vedagū diṭṭhiyāyako na mutiyā,\\
sa mānameti na hi tammayo so.\\
na kammunā nopi sutena neyyo,\\
anūpanīto sa nivesanesu.}

\verseref{847} For one unattached to fictitious perceptions\\
There are no bonds.\\
For one liberated through wisdom\\
There are no illusions.\\
Those attached to fictitious perceptions\newline and to views\\
Roam the world offending people.

\pali{saññāvirattassa na santi ganthā,\\
paññāvimuttassa na santi mohā.\\
saññañca diṭṭhiñca ye aggahesuṃ,\\
te ghaṭṭayantā vicaranti loke'ti.}

\end{verse}
