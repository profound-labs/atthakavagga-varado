
\begin{verse}

\verseref{804} Short indeed is this life.\\
You die within a hundred years.\\
Indeed, if you live beyond that\\
You surely die of decrepitude.

\pali{appaṃ vata jīvitaṃ idaṃ,\\
oraṃ vassasatāpi miyyati.\\
yo cepi aticca jīvati,\\
atha kho so jarasāpi miyyati.}

\verseref{805} People grieve for their beloved possessions;\\
But nothing is possessed forever.\\
Having seen that separation\newline does indeed happen,\\
One should not lead the household life.

\pali{socanti janā mamāyite,\\
na hi santi niccā pariggahā.\\
vinābhāvasantamevidaṃ,\\
iti disvā nāgāramāvase.}

\verseref{806} At death,\\
That which a person supposes to be `mine'\newline is abandoned.\\
Realising this, my wise disciples\newline should not be inclined to possessiveness.

\pali{maraṇenapi taṃ pahīyati,\\
yaṃ puriso mamidanti maññati.\\
etampi viditvā paṇḍito,\\
na mamattāya nametha māmako.}

\verseref{807} On awakening,\\
A man does not see what he met in a dream.\\
Likewise, one does not see loved ones\newline who have passed away.

\pali{supinena yathāpi saṅgataṃ,\\
paṭibuddho puriso na passati.\\
evampi piyāyitaṃ janaṃ,\\
petaṃ kālakataṃ na passati.}

\verseref{808} When they were alive,\\
People called by this name or that\newline were both seen and heard.\\
But when dead,\newline only their names live on to be uttered.

\pali{diṭṭhāpi sutāpi te janā,\\
yesaṃ nāmamidaṃ pavuccati.\\
nāmaṃyevāvasissati,\\
akkheyyaṃ petassa jantuno.}

\verseref{809} Those greedy for beloved possessions\newline do not forsake\\
Grief, lamentation and selfishness.\\
Looking for safety,\\
Sages abandon possessions\newline and lead the homeless life.

\clearpage

\pali{sokapparidevamaccharaṃ,\\
na jahanti giddhā mamāyite.\\
tasmā munayo pariggahaṃ,\\
hitvā acariṃsu khemadassino.}

\verseref{810} For a monk living withdrawn,\\
Resorting to a secluded dwelling,\\
They say that it is fitting\pagenote{fitting: Critical Pali Dictionary calls \pali{asāmaggiya} `want of concord, disharmony'. \pali{Sāmaggiya} would thus be `harmonious' or `fitting'. Norman has `agreeable'.}\\
For him to not exhibit himself in the world.

\pali{patilīnacarassa bhikkhuno,\\
bhajamānassa vivittamāsanaṃ.\\
sāmaggiyamāhu tassa taṃ,\\
yo attānaṃ bhavane na dassaye.}

\verseref{811} The sage\\
Is not tethered in any way.\\
He does not regard anything\newline as either loved or hated.\\
Lamentation and selfishness do not stain him,\\
Just as water does not stain a lotus leaf.

\pali{sabbattha munī anissito,\\
na piyaṃ kubbati nopi appiyaṃ.\\
tasmiṃ paridevamaccharaṃ,\\
paṇṇe vāri yathā na limpati.}

\verseref{812} A lotus leaf or a red lily\\
Is not stained by a waterdrop.\\
The sage, likewise, is not stained\\
By lamentation and selfishness\newline for what is seen, heard or cognised.\pagenote{\pali{nopalimpati}: from the reference in the previous verse, I have taken as `stained by lamentation and selfishness'.}

\clearpage

\pali{udabindu yathāpi pokkhare,\\
padume vāri yathā na limpati.\\
evaṃ muni nopalimpati,\\
yadidaṃ diṭṭhasutaṃ mutesu vā.}

\verseref{813} He does not suppose\\
That he is intrinsically purified\newline by what is seen, heard or cognised.\\
Nor does he want to be thus purified\newline by some auxiliary basis of attachment.\pagenote{by some auxiliary basis of attachment: \pali{āññena}, see: Translation Notes, p. \pageref{transl-auxiliary-basis}.}\\
By nothing is he either attracted or repelled.\pagenote{neither attracted nor repelled. This continues a theme of v.811: `He does not regard anything as either loved or hated'. For \pali{virajjati}, Norman has `dispassioned'. But in the context of v.811 it would mean `repelled'.}

\pali{dhono na hi tena maññati,\\
yadidaṃ diṭṭhasutaṃ mutesu vā.\\
nāññena visuddhimicchati,\\
na hi so rajjati no virajjatīti.}

\end{verse}
