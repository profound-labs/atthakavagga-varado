
\begin{verse}

(Questioner)

\verseref{848} Having what vision,\\
Being of what character,\\
Is one called peaceful?\\
Gotama, tell me about the supreme person.

\pali{kathaṃdassī kathaṃsīlo,\\
upasantoti vuccati.\\
taṃ me gotama pabrūhi,\\
pucchito uttamaṃ naraṃ.}

(The Buddha)

\verseref{849} A person -- before the body's destruction --\\
Who is freed of clinging,\\
Who is not tethered to the past,\\
Who cannot be reckoned\newline in terms of the present,\\
And in whom\newline there are no yearnings for the future;\pagenote{No yearning for the future: \pali{natthi purakkhataṃ} is explained like this by Venerable Mahākaccāna (S.3.11).}

\pali{vītataṇho purā bhedā,\\
pubbamantamanissito.\\
vemajjhe nupasaṅkheyyo,\\
tassa natthi purakkhataṃ.}

\verseref{850} A person who is not angered,\\
Not frightened,\\
Not boastful, not fretful,\\
Who gives wise advice,\\
Who is calm,\\
Restrained in speech,\\
Who is indeed a sage;

\pali{akkodhano asantāsī,\\
avikatthī akukkuco.\\
mantabhāṇī anuddhato,\\
sa ve vācāyato muni.}

\verseref{851} A person who is not attached to the future\\
Who does not sorrow over the past,\\
Who finds solitude amidst sense contacts\pagenote{Solitude implies freedom from passion, clinging to nothing in the world (v.915). See The Octads in a Nutshell: Abiding in solitude, page \pageref{abiding-in-solitude}.}\\
And is not guided by fixed views;

\pali{nirāsatti anāgate,\\
atītaṃ nānusocati.\\
vivekadassī phassesu,\\
diṭṭhīsu ca na nīyati.}

\verseref{852} A person who is retiring,\\
Not deceitful,\\
Not covetous, not selfish,\\
Not impudent, not arousing contempt,\\
Who does not engage in malicious speech;

\pali{patilīno akuhako,\\
apihālu amaccharī.\\
appagabbho ajeguccho,\\
pesuṇeyye ca no yuto.}

\verseref{853} A person who does not relish pleasure,\\
Who is not arrogant,\\
Who is mild and of ready wit,\\
Who is not credulous,\\
Who by nothing is repelled;

\pali{sātiyesu anassāvī,\\
atimāne ca no yuto.\\
saṇho ca paṭibhānavā,\\
na saddho na virajjati.}

\verseref{854} A person who does not take on the training in hopes\newline of material gain,\\
Who is unperturbed if he gets nothing,\\
Who is unhampered by clinging,\\
And not greedy for flavours;

\pali{lābhakamyā na sikkhati,\\
alābhe ca na kuppati.\\
aviruddho ca taṇhāya,\\
rasesu nānugijjhati.}

\verseref{855} A person who is even-tempered,\\
Ever attentive,\\
Who does not suppose that in the world\newline he is equal, superior or inferior,\\
And who is free of conceit;

\pali{upekkhako sadā sato,\\
na loke maññate samaṃ.\\
na visesī na nīceyyo,\\
tassa no santi ussadā.}

\clearpage

\verseref{856} A person for whom there are no tethers,\\
Who, knowing Truth, is not tethered in any way;\pagenote{`In any way': a phrase adopted from v.811, where \pali{anissito} is said to be \pali{sabbattha}. \pali{Sabbattha} seems necessary here too.}\\
And in whom no clinging is found\newline for existence or non-existence:

\pali{yassa nissayanā natthi,\\
ñatvā dhammaṃ anissito.\\
bhavāya vibhavāya vā,\\
taṇhā yassa na vijjati.}

\verseref{857} This is someone I call peaceful.\\
He is indifferent to sensual pleasure.\\
In him, bonds are not found;\\
He has overcome attachment.

\pali{taṃ brūmi upasantoti,\\
kāmesu anapekkhinaṃ.\\
ganthā tassa na vijjanti,\\
atarī so visattikaṃ.}

\verseref{858} He has no children, cattle, fields or property.\\
For him there is nothing clung to,\\
And nothing to relinquish.

\pali{na tassa puttā pasavo,\\
khettaṃ vatthuñca vijjati.\\
attā vāpi nirattā vā,\\
na tasmiṃ upalabbhati.}

\verseref{859} He has no yearning for those things\\
Of which either ordinary people,\\
Ascetics or religious people might talk.\\
Therefore he is unmoved by their disputes.

\pali{yena naṃ vajjuṃ puthujjanā,\\
atho samaṇabrāhmaṇā.\\
taṃ tassa apurakkhataṃ,\\
tasmā vādesu nejati.}

\verseref{860} The sage,\\
Free of greed and selfishness,\\
Does not speak of himself as among those\newline who are superior, equal or inferior.\\
He does not return to the process of time;\\
He is delivered from the phenomenon of time.

\pali{vītagedho amaccharī,\\
na ussesu vadate muni.\\
na samesu na omesu,\\
kappaṃ neti akappiyo.}

\verseref{861} He regards nothing in the world as his own.\\
He does not grieve because of what does not exist.\pagenote{`What does not exist'. \pali{Socati} in v.851 and v.944 refers to the past. But in M.1.137 `what does not exist externally' (\pali{bahiddhā asati paritassanāti}) means whatever one had in the past that is lost, or whatever one wants that one has not gained; `what does not exist internally' (\pali{ajjhattaṃ asati paritassanāti}) means one's presumed Self. Any of these meanings would fit here.}\\
He does not blindly follow religious teachings.\pagenote{\pali{Dhammesu ca na gacchati}: Norman has `does not go (astray) among mental phenomena'. But in the Octads, \pali{dhamma} usually means `religious teaching/s' or the `Buddha's teaching' or `Truth'. Here I take it as `religious teachings'. And although Norman has taken \pali{dhammesu} as the locative case, it may be functionally an ablative case (Duroiselle, para.601, xvi). The phrase would therefore mean `Go by means of religious teachings' or, in other words `Follow religious teachings'. Therefore I have taken the phrase to be a synonym of \pali{dhammā purakkhatā} of v.784: not blindly follow religious teachings.}\\
He is truly called peaceful.

\pali{yassa loke sakaṃ natthi,\\
asatā ca na socati.\\
dhammesu ca na gacchati,\\
sa ve santoti vuccatī'ti.}

\end{verse}
