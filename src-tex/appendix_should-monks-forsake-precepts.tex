
The Octads warns that if one is tethered to precepts and practices, one is likely to regard other people as inferior (v.798). Therefore a monk should not be tethered to precepts and practices. He should forsake them (v.839). This `forsaking' has two possible meanings: forsaking ineffective practices or forsaking the identification with noble precepts.

Attaching to ineffective practices or identifying with noble precepts is called \pali{sīlabbataparāmāso}, which is the third of the lower fetters (\pali{orambhāgiyāni saṃyojanāni}). This fetter is abandoned at stream-entry (M.1.9); because one of the qualities of the stream-enterer is the possession of virtues dear to the noble ones, unbroken, untorn, and not grasped (\pali{aparāmaṭṭhehi}) (S.5.343) -- a word that implies not seeing them as `me' or `mine' or `my Self' (\pali{etaṃ mama esohamasmi eso me attā'ti}) (S.2.94).

But forsaking precepts does not mean immoral behaviour. What the Octads emphasises is good behaviour, as well as detachment. And detachment supports good behaviour not bad behaviour. Venerable Soṇa said the moral behaviour of an arahant is a natural expression of an enlightened mind. When he was accused of practising harmlessness `due to blind attachment to precepts and practices' (\pali{silabbataparāmāsaṃ}), he said no, the arahant is intent on non-harming because of the destruction of attachment, hate and delusion (\pali{khayā rāgassa vītarāgattā \ldots{} dosassa vītadosattā \ldots{} mohassa vītamohattā abyāpajjādhimutto hoti}) (Vin.1.183-5).

Precepts and practices recommended by the Buddha are those where skilful states (\pali{kusalā dhammā}) grow and unskilful states (\pali{akusalā dhammā}) diminish (A.1.225). Stream-enterers discard unskilful religious practices and undertake noble precepts, but without grasping these precepts (\pali{aparāmaṭṭhehi}); which means that the stream-enterer does not see precepts as `me' or `mine' (S.2.94).

The following examples from the suttas illustrate how unskilful practices are discarded, and skilful practices are undertaken:

\begin{itemize}

\item Fire-worship: When Kassapa of Uruvela and his group of matted hair ascetic disciples decided to take ordination under the Buddha, they flung into the river their hair, braids, bundles on carrying poles, and fire-worshipping implements (Vin.1.32-3).

\item Sacrifice: A brahman brought hundreds of bulls, goats and sheep to the sacrificial post for slaughter and burning, then asked the Buddha how to perform the sacrifice so it would be of the greatest benefit. The Buddha replied that even in preparing for such a sacrifice, thinking to make merit, one makes demerit; thinking to do good, one does evil; thinking one is pursuing happiness, one is pursuing pain. Then the Buddha explained that greed, hatred and delusion are three fires that should be shunned. He said that three fires should be venerated instead: one's parents, one's family, and employees, ascetics and brahmans.

\item River cleansing: The brahman Sundarika Bhāradvāja asked the Buddha if he ever went to the Bahukā River to bathe? `For in the Bahukā River many people wash away the evil deeds they have done'. The Buddha replied that a `fool may bathe there forever, yet will not purify his black deeds'. He said that someone who is pure in heart and who does good deeds is ever cleansed, and advised the brahman to bathe in this way, to make himself a refuge for all beings, and to keep the moral precepts (M.1.39).

\item River cleansing: A brahman said to the bhikkhunī Punnikā: `Whoever, young or old, does evil kamma, is, through water ablution, from evil kamma set free'. Punnikā replied: `In that case, they would all go to heaven: all the frogs, turtles, serpents, crocodiles, and anything else that lives in the water'. She said that if these rivers could carry off evil kamma, they would carry off merit as well. She advised the brahman to stop doing whatever it was that made him always need cleansing, and added `Don't let the cold hurt your skin'. The brahman said `I've been following the miserable path, good lady, and now you've brought me back to the noble' (Thi.p146).

\item Worshipping and serving: One early morning, the Buddha met a young brahman named Sigālaka, who, with clothes and hair still dripping from his ritual bath, and with joined palms, was worshipping (\pali{namassati}) the six directions out of respectful obedience to his father's dying request that he do so. The Buddha told Sigālaka that according to the noble discipline (\pali{ariyassa vinaye}) this was not the way to worship the six directions, which Sigālaka then asked the Buddha to explain. In fact, the Buddha explained, not how to `worship' the six directions, but how to `cover' them (\pali{paṭicchādī}), which he explained meant `serving' the people in one's life (\pali{paccupaṭṭhātabbā}) because it is likely that `worshipping' was a term that he felt should be used exclusively in relationship to the Buddha, Dhamma and Sa\.ngha. He told Sigālaka how to properly serve six groups of people: one's parents, teachers, spouses, friends, servants, and ascetics and brahmans, and how these can reciprocate by showing their tenderness (\pali{anukampanti}) in return. The Buddha said that if one follows this advice, then each direction is made safe, free of fear (\pali{khemā appaṭibhayā}) -- possibly because one is not cultivating danger and fear within any relationship. Instead, one is cultivating three qualities that might summarise the Buddha's advice to Sigālaka: respect, kindness and dutifulness (D.3.180).

\item Purifying rites: Cunda was a silversmith whose purifying rites involved him touching the ground, cowdung or grass, worshipping fire or the sun, and bathing three times a day. The Buddha said that this was different from noble purification (\pali{ariyassa vinaye soceyyaṃ}) which, at Cunda's request, he explained meant the four ways of right speech, the four ways of right conduct, and freedom from covetousness, ill-will and wrong views. These noble purifications result in someone who is indeed pure (\pali{suci yeva hoti}) whether or not he practises touching the ground, worshipping fire and bathing three times a day, and they lead to happy rebirths, either celestial or human (A.5.263-268).
\item Going upwards: There is a brahman practice called `going upwards' (\pali{udayagāminiṃ nāma paṭipadaṃ}) in which a disciple is told to get up early and walk facing east, and told not to avoid a pit, a precipice, a stump, a thorny place, a village pool, or a cesspool, and told to `expect death wherever you fall. Thus, good man, with the breakup of the body, after death, you will be reborn in heaven.' The Buddha said that this foolish practice does not lead to revulsion, dispassion, ending, peace, realisation, enlightenment or Nibbāna. The practice called `going upwards' in the Noble One's Discipline (\pali{ariyassa vinaye udayagāminiṃ paṭipadaṃ paññāpemi}) involves having unwavering faith in the Buddha, Dhamma, Sa\.ngha,and possession of the virtues dear to the noble ones. This leads to utter revulsion, to dispassion, to ending, to peace, to realisation, to enlightenment, to Nibbāna' (S.5.361).
\item Gruelling asceticism: Before his enlightenment, the Buddha practised various ascetic practices. For instance, he rejected social conventions by practising nakedness; by remaining standing when eating, urinating and defaecating; by licking his hands clean instead of washing them. He tormented himself by standing continuously, rejecting seats; or by maintaining the squatting position; by using a bed of spikes; by bathing in cold water three times daily including the evening. He survived on very small amounts of food, and reached a state of extreme emaciation. Yet by such conduct and austerity he admitted that he did not attain any superhuman state of knowledge and vision that was truly noble, because he did not attain noble wisdom (\pali{ariyāya paññāya}) which leads to the complete destruction of suffering (M.1.81). Later, he was to reflect: `I am indeed freed from that gruelling asceticism. It is good indeed that I am freed from that useless gruelling asceticism. It is good that, steady and mindful, I have attained enlightenment' (S.1.103).
\end{itemize}
