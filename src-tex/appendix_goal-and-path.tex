
Some religions proclaim a final goal that is lower than the Buddhist Nibbāna. Indeed, what some ascetics proclaim as the final goal, is often merely a stage on the Buddhist path. For instance, some proclaim `ultimate purification' lies in the sphere of neither perception nor non-perception (A.5.64). It was on the basis of this that the Buddha's teacher Uddaka Rāmaputta proclaimed himself a `knowledge-master' and a `universal conqueror' (S.4.83; M.1.165). Others say that Nibbāna is found in the practice of \pali{jhāna} (D.1.37-8); or proclaim that the highest austerity lies in suffusing the world with a mind of compassion and serenity (D.3.50). This confusion about what constitutes the goal is one of the sources of quarrels in the Octads.

The Buddha said that the highest Nibbāna is liberation free of ownership (\pali{anupādā vimokkho}) (A.5.64). He said that this Nibbāna exists (\pali{tiṭṭhateva nibbānaṃ}); the path to it exists (\pali{tiṭṭhati nibbānagāmimaggo}); and he said that he was the guide (\pali{samādapetā}). When his disciples have been instructed by him, some attain this ultimate goal, some do not. The Buddha explained: `What can I do about it? A Tathāgata is the proclaimer of the path' (\pali{maggakkhāyīhaṃ tathāgato'ti}) (M.3.6). So he was not proclaiming Truth. The confusion between the path and goal is another source of quarrels in the Octads.

The relationship between path and goal is explained in several suttas. When someone visited Venerable Ānanda in Ghosita's monastery Venerable Ānanda told him that the goal of the holy life is to abandon longing (\pali{chando}); and to abandon longing, one must long to do so. When one has achieved the goal, the longing ceases: the longing for arahantship ceases at arahantship. To illustrate this, Ānanda asked his visitor if he had earlier longed to visit the monastery. Of course, the man said yes. Then, when arrived, did the longing subside? Again the man said yes (S.5.272).

Similarly, the Buddha compared his teachings to a raft for crossing a stream from a danger to safety. Having crossed, one might reflect on how useful the raft has been; but one should then abandon it. It would be absurd to carry it around on one's head. The Buddha said his teachings likewise were for crossing over, not for grasping. He concluded: `When you see that religious teachings are similar to a raft, you should abandon even what is righteous, how much more so things which are unrighteous (\pali{dhammāpi vo pahātabbā pageva adhammā}) (M.1.135).

Similarly, Venerable Puṇṇa said that the stages of the spiritual path were like a relay of chariots. The only purpose of each stage is to reach the next stage.

\begin{itemize}
\item Purifying virtue
\item is to purify the mind
\item which is to purify one's views
\item which is to overcome uncertainty
\item which is to attain knowledge and vision of what is the path and what is not
\item which is to attain knowledge and vision of the practice
\item which is to attain knowledge and vision
\item which is to attain final Nibbāna without clinging.
\end{itemize}

Each of the intermediate stages is eventually abandoned. But each stage must be attained before its abandonment. As Venerable Puṇṇa explained: if final Nibbāna could be attained without the intermediate stages, then an ordinary person would attain Nibbāna, because he is without these stages (M.1.149-150).

The three similes illustrate different aspects of liberation:

\begin{itemize}
\item the simile of walking to a park shows that one's spiritual efforts are abandoned;
\item the simile of the raft shows that the Buddha's teachings are abandoned;
\item the simile of the relay chariots shows that one's attainments on the path are abandoned.
\end{itemize}

