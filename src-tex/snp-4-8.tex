
\begin{verse}

\verseref{824} `Here alone is purity found'\\
The so-called experts say.\pagenote{The so-called experts say (\pali{kusalā vadānā}): the phrase in fact occurs in the next verse. For \pali{vadānā} I follow PED's `so-called'.}\\
They deny that purity\newline is intrinsic to other religious teachings.\\
Wherever they are tethered\newline is their so-called `sanctity'.\\
They are each committed to their separate `Truths'.

\pali{idheva suddhi iti vādayanti,\\
nāññesu dhammesu visuddhimāhu.\\
yaṃ nissitā tattha subhaṃ vadānā,\\
paccekasaccesu puthū niviṭṭhā.}

\verseref{825} Looking for an argument,\newline they gather at meetings,\\
Considering each other fools.\\
Clinging to knowledge,\\
Wanting praise,\\
They argue --\\
The so-called experts.

\pali{te vādakāmā parisaṃ vigayha,\\
bālaṃ dahantī mithu aññamaññaṃ.\\
vadanti te aññasitā kathojjaṃ,\\
pasaṃsakāmā kusalā vadānā.}

\verseref{826} In the midst of gatherings,\\
Engaged in dispute,\\
A person is desirous of praise\\
But is also anxious about the outcome.\\
If his argument is refuted he becomes downcast.\\
Shaken by criticism,\newline he looks for his opponent's weak spots.

\pali{yutto kathāyaṃ parisāya majjhe,\\
pasaṃsamicchaṃ vinighāti hoti.\\
apāhatasmiṃ pana maṅku hoti,\\
nindāya so kuppati randhamesī.}

\verseref{827} If the judge declares that his argument is inferior,\\
And therefore refuted,\\
The inferior speaker laments and grieves.\\
`He defeated me' he wails.

\pali{yamassa vādaṃ parihīnamāhu,\\
apāhataṃ pañhavimaṃsakāse.\\
paridevati socati hīnavādo,\\
upaccagā manti anutthunāti.}

\verseref{828} These disputes have arisen among ascetics.\\
In them are victory and defeat.\\
Seeing this,\\
One should desist from arguments\\
For they have no other purpose\newline than the gaining of praise.

\pali{ete vivādā samaṇesu jātā,\\
etesu ugghāti nighāti hoti.\\
etampi disvā virame kathojjaṃ,\\
na haññadatthatthipasaṃsalābhā.}

\verseref{829} He who is praised for presenting his argument\newline in the midst of a gathering,\\
Having attained his heart’s desire,\\
Is mirthful and conceited on account of it.

\pali{pasaṃsito vā pana tattha hoti,\\
akkhāya vādaṃ parisāya majjhe.\\
so hassatī uṇṇamatī ca tena,\\
pappuyya tamatthaṃ yathā mano ahu.}

\verseref{830} That conceit will be the basis of later distress.\\
Moreover, he speaks with pride and arrogance.\\
Seeing this,\\
One should desist from arguments.\\
No purity is attained thereby, say the good.

\pali{yā uṇṇatī sāssa vighātabhūmi,\\
mānātimānaṃ vadate paneso.\\
etampi disvā na vivādayetha,\\
na hi tena suddhiṃ kusalā vadanti.}

\verseref{831} Like a hero nourished on royal food\\
He thunders along looking for an opponent.\\
Run wherever he is, hero.\\
There is nothing for you to fight against here.

\pali{sūro yathā rājakhādāya puṭṭho,\\
abhigajjameti paṭisūramicchaṃ.\\
yeneva so tena palehi sūra,\\
pubbeva natthi yadidaṃ yudhāya.}

\verseref{832} They who argue,\\
Grasping a view,\\
Asserting that `This is very Truth',\\
You can talk to those people.\\
But \emph{here}\\
There is no opponent for you to battle with\newline when a dispute has arisen.

\pali{ye diṭṭhimuggayha vivādayanti,\\
idameva saccanti ca vādayanti.\\
te tvaṃ vadassū na hi tedha atthi,\\
vādamhi jāte paṭisenikattā.}

\verseref{833} Amongst those who have abandoned confrontation,\\
Who do not pit one view against another,\\
Amongst those who have not grasped any view\newline as the highest Goal,\\
Who would you gain as opponent, Pasūra?

\pali{visenikatvā pana ye caranti,\\
diṭṭhīhi diṭṭhiṃ avirujjhamānā.\\
tesu tvaṃ kiṃ labhetho pasūra,\\
yesīdha natthī paramuggahītaṃ.}

\verseref{834} So here you come,\\
Speculating,\\
Mulling over various theories in your mind.\\
But you are paired off with a purified man.\\
With him you will not be able to proceed.

\pali{atha tvaṃ pavitakkamāgamā,\\
manasā diṭṭhigatāni cintayanto.\\
dhonena yugaṃ samāgamā,\\
na hi tvaṃ sakkhasi sampayātaveti.}

\end{verse}
