
\begin{verse}

\verseref{788} `I see Purity, the highest Goal, the Non-Afflicted:\pagenote{The Buddha used these same epithets for \pali{Nibbāna}: \pali{suddhiṃ} (S.4.372); \pali{paramaṃ}) (Dh v.184); \pali{ārogya} (M.1.510).}\\
A man's purification is intrinsic to his vision.'\pagenote{A `seer of purity' thinks a man's purity is intrinsic to his vision, unlike a wise person, `who does not relish what is seen or heard' (v.897).}\\
Knowing such knowledge as `the highest Goal'\\
A `Seer of Purity' reverts to knowledge.

\pali{passāmi suddhaṃ paramaṃ arogaṃ,\\
diṭṭhena saṃsuddhi narassa hoti.\\
evābhijānaṃ paramanti ñatvā,\\
suddhānupassīti pacceti ñāṇaṃ.}

\verseref{789} If a man's purification is intrinsic to his vision,\\
If forsaking sorrow is intrinsic to his knowledge,\\
Then a person with one basis of attachment\\
Is intrinsically purified by means of another.\pagenote{by means of another: \pali{aññena}, see Translation Notes, page \pageref{transl-auxiliary-basis}.}\\
The view of one who asserts purity in this way\newline is thus belied.

\pali{diṭṭhena ce suddhi narassa hoti,\\
ñāṇena vā so pajahāti dukkhaṃ.\\
aññena so sujjhati sopadhīko,\\
diṭṭhī hi naṃ pāva tathā vadānaṃ.}

\verseref{790} No Brahman says that purification\newline is intrinsic to an auxiliary basis of attachment,\pagenote{`some auxiliary basis of attachment': translation of \pali{aññato} in the context of verse 789.}\\
Either to what is seen, heard or cognised,\newline or to precepts and practices.\\
A Brahman is not stained by merit or evil.\\
Forsaking whatever he was clinging to,\\
He does not make further attachments\newline in the world.\pagenote{\pali{nayidha pakubbamāno}: Norman has `does not fashion [anything more] here'. This is because \pali{pakaroti} means `effect, perform, prepare, make, do' (PED). But the verse is about purity through non-attachment. I translate the verse accordingly: `He does not make any more attachments in this world'.}

\pali{na brāhmaṇo aññato suddhimāha,\\
diṭṭhe sute sīlavate mute vā.\\
puññe ca pāpe ca anūpalitto,\\
attañjaho nayidha pakubbamāno.}

\verseref{791} Those following craving,\\
Forsaking what they have\newline in order to grab something different,\\
Do not cross over attachment.\\
They release and catch hold --\\
Like a monkey releasing one branch\newline in order to seize another.

\pali{purimaṃ pahāya aparaṃ sitāse,\\
ejānugā te na taranti saṅgaṃ.\\
te uggahāyanti nirassajanti,\\
kapīva sākhaṃ pamuñcaṃ gahāyaṃ.}

\verseref{792} A person attached to fictitious perceptions\pagenote{fictitious perceptions: \pali{saññaṃ}, see Translation Notes, page \pageref{transl-fictitious-perceptions}.}\\
Who undertakes religious practices of his own\\
Goes high and low.\\
But one of great wisdom,\\
One knowing the Buddha's teaching, a sage,\\
Does not go high and low.

\pali{sayaṃ samādāya vatāni jantu,\\
uccāvacaṃ gacchati saññasatto.\\
vidvā ca vedehi samecca dhammaṃ,\\
na uccāvacaṃ gacchati bhūripañño.}

\verseref{793} He is peaceful towards everything\newline whether seen, heard or cognised.\\
He sees things as they are,\newline and conducts himself openly.\\
How could anyone have any doubts about him?

\pali{sa sabbadhammesu visenibhūto,\\
yaṃ kiñci diṭṭhaṃ va sutaṃ mutaṃ vā.\\
tameva dassiṃ vivaṭaṃ carantaṃ,\\
kenīdha lokasmi vikappayeyya.}

\verseref{794} Ones like him neither concoct religious teachings\\
Nor blindly follow them.\pagenote{blindly follow: \pali{purakkhatā}, see Translation Notes, page \pageref{transl-blindly-follow}. In verse 784 it refers to religious teachings (\pali{dhammā}), and is therefore rendered as such here.}\\
They do not proclaim of any teaching\newline `This itself is final purification'.\\
Having loosened the knot of grasping\newline with which they are bound\\
They do not hope for anything in the world.

\pali{na kappayanti na purekkharonti,\\
accantasuddhīti na te vadanti.\\
ādānaganthaṃ gathitaṃ visajja,\\
āsaṃ na kubbanti kuhiñci loke.}

\verseref{795} The Brahman has gone beyond\newline conventional boundaries.\\
He has grasped nothing,\newline either what is seen or known.\\
He is not overcome by lust,\newline nor overwhelmed by disgust.\\
There is nothing in the world\newline grasped by him as the highest Goal.\pagenote{ The highest Goal: \pali{paramaṃ}, see Translation Notes, page \pageref{transl-highest-goal}.}

\clearpage

\pali{sīmātigo brāhmaṇo tassa natthi,\\
ñatvā va disvā va samuggahītaṃ.\\
na rāgarāgī na virāgaratto,\\
tassīdha natthī paramuggahītanti.}

\end{verse}

